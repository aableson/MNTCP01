
\documentclass[10pt]{article}

\usepackage{graphicx,amsmath,amssymb,subfigure,enumerate,versions}
\usepackage{multicol,multirow,mdframed}
\usepackage{epstopdf}
\usepackage{pstricks,auto-pst-pdf}
\usepackage{pst-all}
\usepackage{pst-ode}
\usepackage{pst-math}
\usepackage{hyperref}
\usepackage{listings}
%\usepackage{mcode}
\lstset{language=Matlab}
\DeclareGraphicsExtensions{.png,.jpg,.pdf}

% ************ Page Margins *************
\hoffset=-1.3in
\setlength{\textwidth}{7.5in}
%%%%% MARGINS
\topmargin 0pt
\advance \topmargin by -\headheight
\advance \topmargin by -\headsep
\textheight 9.5in

% ************ Shortcuts *************
\newcommand{\Z}{\mbox{\sf Z\hspace{-1.5mm}Z}}
\newcommand{\SolutionSeparator}{ \hfill \hfill \hrule \hfill \hfill }
\newcommand{\R}{\mbox{\rm I\hspace{-0.75mm}R}}
\columnsep=0.75in
\newcommand{\vsc}{\vspace{1mm}}
\newcommand{\D}{\Delta }
\newcommand{\ifd}{f(x)~dx}
\newcommand{\dd}{\frac{dy}{dx} \,} 
\newcommand{\der}[2]{\frac{d{#1}}{d{#2}} \,}
\newcommand{\ddx}[1]{\frac{d {#1}}{dx} \,} 
\newcommand{\ddy}[1]{\frac{d {#1}}{dy} \,} 
\newcommand{\ddz}[1]{\frac{d {#1}}{dz} \,} 
\newcommand{\ddt}[1]{\frac{d {#1}}{dt} \,} 
\newcommand{\ds}{\displaystyle } 
\newcommand{\la}{\lambda } 
\newcommand{\del}{\nabla } 
\newcommand{\zx}{\frac{\partial z}{\partial x} \,}
\newcommand{\zy}{\frac{\partial z}{\partial y} \,}
\newcommand{\dx}{\frac{\partial f}{\partial x} \,}
\newcommand{\dy}{\frac{\partial f}{\partial y} \,}
\newcommand{\pp}[2]{\frac{\partial {#1}}{\partial {#2}} \,}
\newcommand{\ppx}{\frac{\partial }{\partial x} \,}
\newcommand{\ppy}{\frac{\partial }{\partial y} \,}
\renewcommand{\thesection}{\Roman{section}}
\newcommand{\vi}{\vec{i}}
\newcommand{\vj}{\vec{j}}
\newcommand{\vk}{\vec{k}}
\newcommand{\vv}{\vec{v}}
\newcommand{\lan}{\left\langle}
\newcommand{\ran}{\right\rangle}
\newcommand{\degr}{^{\circ}}

% *** Define the printed question style ***
\newcommand{\q}[1]{ {\em #1} }
% \renewcommand{\q}[1]{ {} }

\newcommand{\notice}{ \begin{center}Some problems and solutions
    selected or adapted from \\ Stewart {\em Calculus-Early
      Transcendentals} and Hughes-Hallett {\em Calculus} .\end{center}
}

% *** Overwrite, if desired, the question format
\includeversion{Question} 
\includeversion{Solution}

\newcommand{\multicolstart}{ }
\newcommand{\multicolend}{ }

\renewenvironment{Question}
{ \begin{mdframed}[nobreak=true,hidealllines=true,backgroundcolor=gray!50,innerleftmargin=5ex] }
{ \end{mdframed} }


% *** Footnoting with symbols ***
\long\def\symbolfootnote[#1]#2{\begingroup%
\def\thefootnote{\fnsymbol{footnote}}\footnote[#1]{#2}\endgroup}

\newcommand{\WeekTitleOne}{Derivatives - Foundations}
\newcommand{\WeekTitleTwo}{Derivatives - Linearization and Applications}
\newcommand{\WeekTitleThree}{Derivatives - Modeling}
\newcommand{\WeekTitleFour}{Integrals - Foundations}
\newcommand{\WeekTitleFive}{Integrals - Techniques}
\newcommand{\WeekTitleSix}{Integrals - Modeling}
\newcommand{\WeekTitleSeven}{Differential Equations - }
\newcommand{\WeekTitleEight}{Differential Equations - }
\newcommand{\WeekTitleNine}{Differential Equations - }
\newcommand{\WeekTitleTen}{Linear Algebra - }
\newcommand{\WeekTitleEleven}{Linear Algebra - }
\newcommand{\WeekTitleTwelve}{Linear Algebra - }


\usepackage{bbding} % for Checkmarkbold
\begin{document}

\newcommand{\ub}{\underbrace}

\begin{center}
\subsection*{MNTC P01 - Week \#9 - \WeekTitleNine}
\end{center}

% % ******************** PENDULUM ********************************

% \item In class, we saw the differential equation for the angular
%   motion of a pendulum.  Here it is again, with no friction.
% \begin{align*}
%   \mbox{Newton's Second Law: }   m  L^2 \theta'' & = T_g \\
%   & = - m L g \sin(\theta)  \\
%   \mbox{Solving for $\theta''$: }\theta'' & = - \frac{g}{L} \sin(\theta) 
% \end{align*}

% Without simulating the actual motion of the pendulum, we can compute
% the period, $T$, using the formula below:
% \begin{align*}
%   T = 4 \sqrt{L/g} \int_0^{\pi/2} \frac{dx}{\sqrt{1 - k^2 \sin^2 x}}
% \end{align*}
% where $k = \sin\left(\frac{1}{2} \theta_0\right)$ and $g$ is the
% acceleration due to gravity, $9.8 $ m/s.  


% For each set of values for $L$ and $\theta_0$,
% \begin{enumerate}[(a)]
% \item Use \verb#quad# to find the period of the pendulum oscillations,
%   and
% \item confirm the period by using \verb#ode45# to simulate the motion
%   pendulum for exactly that length of time, and plot a graph of the
%   angular {\bf velocity} against time.  The velocity should just reach
%   zero at the end of one cycle.
% \end{enumerate}

% \begin{enumerate}[(i)]
% \item $L = 2$ m, $\theta_0 = 40^o$, 
% \item $L =2.5$ m, $\theta_0 = 20^o$.
% \item $L =5.0$ m, $\theta_0 = 90^o$.
% \end{enumerate}

% \begin{solutions}

%   All of the problems are shown solved in \verb#q_pendulum.m#.

%   \begin{enumerate}[(i)]
%   \item $L = 2$ m, $\theta_0 = 40^o$:  {\bf T = 2.9274}
%   \item $L =2.5$ m, $\theta_0 = 20^o$: {\bf T = 3.1978}
%   \item $L =5.0$ m, $\theta_0 = 90^o$: {\bf T = 5.2974}
%   \end{enumerate}

% Plots: \\
% \includegraphics[width=2in]{q_pendulum_1} 
% \includegraphics[width=2in]{q_pendulum_2} 
% \includegraphics[width=2in]{q_pendulum_3} 

% \end{solutions}




\begin{enumerate}
\item A 150 litre tank initially contains 60 litres of water with 0.5
  kgs of salt dissolved in it.  Water enters the tank at a rate of 0.9
  litres/hr and the water entering the tank has a salt concentration
  of $\frac{1}{5}(1 + \cos (t))$ kgs/litre. If a well mixed solution
  leaves the tank at a rate of 0.6 litres/hr, how much salt is in the
  tank when it overflows?

% ******************** Fish Population ********************************

\item Differential equations are not just well-suited for physics
  applications: they are are also widely used in biology, particularly
  in population models.

  Consider the fish population model below, based on a standard
  limited-resource population growth, minus a function of harvesting.

$$\frac{dP}{dt} =\underbrace{ [(10 -P)\cdot P]}_{\mbox{natural population growth rate}} -\underbrace{h(t)}_{\mbox{harvesting rate}}$$
where 
\begin{itemize}
\item $P$  = population of fish (in thousands), and 
\item $\frac{dP}{dt}$  = rate of population change, in thousands per 
year
\item $h(t)$ is the harvesting rate (in thousands of fish per 
year)
\end{itemize}

We want to study the impact of two harvesting models:
\begin{itemize}
\item $h_1 = k_1$; constant harvesting
\item $h_2(t) = k_2 (\sin(\pi t) + 1)$; seasonal model where the
  harvesting has a yearly cycle.
\end{itemize}
\begin{enumerate}

\item Generate a prediction of the population over time, starting at
  initial populations of $P(0) = 15$ for each model.  Use $k_1 = k_2 =
  5$. Produce a graph showing the predicted population over time on
  the same graph, over a long enough time interval to show the
  long-term behaviour of both solutions.

\vspace{0.2in}

One question that arises in such harvesting models is which fishing
strategy permits a higher average harvesting rate can be maintained:
seasonal harvesting, or constant harvesting?  To decide this, we note
that the average harvest rate for $h_1$ is $k_1$, and for $h_2$ is
$k_2$, so whichever value of $k_1$ and $k_2$ is larger indicates the
strategy with the greater average harvesting rate.

We will define the {\em maximum sustainable harvest rate} for both
models as the {\em highest harvest rate for which the population is
  not driven to zero.}

\item Find and report the maximum sustainable harvest level $k_1$ for
  the constant harvesting model (to the nearest integer).  (Use trial
  and error if necessary, though more insightful DE-related ways are
  possible.)  Indicate how you found the cut-off level. 

  {\em NOTE: during this process, your model will predict a population
    of zero, which will then lead to large negative populations.  This
    clearly makes no sense, so limit your plots with the command
    \verb#ylim([0, P0])#.  This same problem will also trigger
    warnings in ode45 about error tolerances; you can safely ignore
    those warnings.}

\item Generate a plot showing the population over time, using the same
  initial value used earlier, but using both the $k_1$ value just
  above, and just below the extinction level. (One line should remain
  positive, while the other should crash to zero at some point on the
  graph.)

\item Use trial and error (theory isn't much help here) to find the
  maximum sustainable harvest level $k_2$ for the cyclic harvesting
  model (to the nearest integer).  Include a plot showing the
  population over time with this harvesting level.

\item Based on your experiments, can constant harvesting or cyclic
  harvesting sustain a greater average harvest in the long run?
  Explain your reasoning.
\end{enumerate}

\begin{solutions}
\begin{solution}
\begin{enumerate}
\item The file \verb#q_fishHarvesting.m# shows the answer to this part of
  the question.  It can be modified to answer the later parts.

\item Note that trial and error is just fine for this problem.  If the
  next paragraph doesn't make sense to you, feel free to skip it.  The
  important thing is to be able to do the simulations, and to generate
  the appropriate graphs.

  To get an analytic answer for the maximum sustainable harvest rate,
  we look more closely at the differential equation. We're worried
  about the population of the fish always decreasing, which would mean
  a derivative $\frac{dP}{dt}$ always being negative.  That would
  happen if the quadratic part was always smaller than $h_1 = k_1$.
  Looking at the quadratic $(10-P)P$, it will be largest at $P= 5$
  (halfway between the roots of $P=0$ and $P=10$, and will produce a
  maximum growth rate then of $(10 - 5)5 = 25$.  If we set $k_1 = 25$,
  we should just be on the threshold of sustainability.  Any value
  larger than that, and the population rate of change will always be
  negative, leading to population collapse.

  Plotting the solution for the constant harvest model at $k_1 = 25$
  and $k_1 = 26$ supports this hypothesis.
\begin{center}
\includegraphics[width=3in]{q_FishHarvestingConstant}
\end{center}

\item In the seasonal harvest scenario, using theory to find the
  maximum sustainable value of $k_2$ isn't straightforward.  Instead,
  we simply experiment with values of $k_2$, and find that between
  $k_2 = 16$ and $k_2 = 17$, we see our seasonal pattern stop
  repeating and start reaching extinction:

\begin{center}
\includegraphics[width=3in]{q_FishHarvestingSeasonal}
\end{center}

\item Based on these experiments, it seems that seasonal harvesting
  leads to extinction at lower average harvesting levels, because a
  lower average rate of harvest (16 thousand fish per year) leads to
  extinction, compared to the constant harvest case (where 25 thousand
  fish per year can be harvested).

\end{enumerate}

\end{document}

