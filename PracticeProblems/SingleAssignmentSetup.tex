\documentclass[10pt]{article}

\usepackage{graphicx,amsmath,amssymb,subfigure,enumerate,versions}
\usepackage{multicol,multirow,mdframed}
\usepackage{epstopdf}
\usepackage{pstricks,auto-pst-pdf}
\usepackage{pst-all}
\usepackage{pst-ode}
\usepackage{pst-math}
\DeclareGraphicsExtensions{.png,.jpg,.pdf}

% ************ Page Margins *************
\hoffset=-1.3in
\setlength{\textwidth}{7.5in}
%%%%% MARGINS
\topmargin 0pt
\advance \topmargin by -\headheight
\advance \topmargin by -\headsep
\textheight 9.5in

% ************ Shortcuts *************
\newcommand{\Z}{\mbox{\sf Z\hspace{-1.5mm}Z}}
\newcommand{\SolutionSeparator}{ \hfill \hfill \hrule \hfill \hfill }
\newcommand{\R}{\mbox{\rm I\hspace{-0.75mm}R}}
\columnsep=0.75in
\newcommand{\vsc}{\vspace{1mm}}
\newcommand{\D}{\Delta }
\newcommand{\ifd}{f(x)~dx}
\newcommand{\dd}{\frac{dy}{dx} \,} 
\newcommand{\der}[2]{\frac{d{#1}}{d{#2}} \,}
\newcommand{\ddx}[1]{\frac{d {#1}}{dx} \,} 
\newcommand{\ddy}[1]{\frac{d {#1}}{dy} \,} 
\newcommand{\ddz}[1]{\frac{d {#1}}{dz} \,} 
\newcommand{\ddt}[1]{\frac{d {#1}}{dt} \,} 
\newcommand{\ds}{\displaystyle } 
\newcommand{\la}{\lambda } 
\newcommand{\del}{\nabla } 
\newcommand{\zx}{\frac{\partial z}{\partial x} \,}
\newcommand{\zy}{\frac{\partial z}{\partial y} \,}
\newcommand{\dx}{\frac{\partial f}{\partial x} \,}
\newcommand{\dy}{\frac{\partial f}{\partial y} \,}
\newcommand{\pp}[2]{\frac{\partial {#1}}{\partial {#2}} \,}
\newcommand{\ppx}{\frac{\partial }{\partial x} \,}
\newcommand{\ppy}{\frac{\partial }{\partial y} \,}
\renewcommand{\thesection}{\Roman{section}}
\newcommand{\vi}{\vec{i}}
\newcommand{\vj}{\vec{j}}
\newcommand{\vk}{\vec{k}}
\newcommand{\vv}{\vec{v}}
\newcommand{\lan}{\left\langle}
\newcommand{\ran}{\right\rangle}
\newcommand{\degr}{^{\circ}}

% *** Define the printed question style ***
\newcommand{\q}[1]{ {\em #1} }
% \renewcommand{\q}[1]{ {} }

\newcommand{\notice}{ \begin{center}Some problems and solutions
    selected or adapted from \\ Stewart {\em Calculus-Early
      Transcendentals} and Hughes-Hallett {\em Calculus} .\end{center}
}

% *** Overwrite, if desired, the question format
\input{DocumentFormat.tex}

% *** Footnoting with symbols ***
\long\def\symbolfootnote[#1]#2{\begingroup%
\def\thefootnote{\fnsymbol{footnote}}\footnote[#1]{#2}\endgroup}
