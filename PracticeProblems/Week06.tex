
\documentclass[10pt]{article}

\usepackage{graphicx,amsmath,amssymb,subfigure,enumerate,versions}
\usepackage{multicol,multirow,mdframed}
\usepackage{epstopdf}
\usepackage{pstricks,auto-pst-pdf}
\usepackage{pst-all}
\usepackage{pst-ode}
\usepackage{pst-math}
\usepackage{hyperref}
\usepackage{listings}
%\usepackage{mcode}
\lstset{language=Matlab}
\DeclareGraphicsExtensions{.png,.jpg,.pdf}

% ************ Page Margins *************
\hoffset=-1.3in
\setlength{\textwidth}{7.5in}
%%%%% MARGINS
\topmargin 0pt
\advance \topmargin by -\headheight
\advance \topmargin by -\headsep
\textheight 9.5in

% ************ Shortcuts *************
\newcommand{\Z}{\mbox{\sf Z\hspace{-1.5mm}Z}}
\newcommand{\SolutionSeparator}{ \hfill \hfill \hrule \hfill \hfill }
\newcommand{\R}{\mbox{\rm I\hspace{-0.75mm}R}}
\columnsep=0.75in
\newcommand{\vsc}{\vspace{1mm}}
\newcommand{\D}{\Delta }
\newcommand{\ifd}{f(x)~dx}
\newcommand{\dd}{\frac{dy}{dx} \,} 
\newcommand{\der}[2]{\frac{d{#1}}{d{#2}} \,}
\newcommand{\ddx}[1]{\frac{d {#1}}{dx} \,} 
\newcommand{\ddy}[1]{\frac{d {#1}}{dy} \,} 
\newcommand{\ddz}[1]{\frac{d {#1}}{dz} \,} 
\newcommand{\ddt}[1]{\frac{d {#1}}{dt} \,} 
\newcommand{\ds}{\displaystyle } 
\newcommand{\la}{\lambda } 
\newcommand{\del}{\nabla } 
\newcommand{\zx}{\frac{\partial z}{\partial x} \,}
\newcommand{\zy}{\frac{\partial z}{\partial y} \,}
\newcommand{\dx}{\frac{\partial f}{\partial x} \,}
\newcommand{\dy}{\frac{\partial f}{\partial y} \,}
\newcommand{\pp}[2]{\frac{\partial {#1}}{\partial {#2}} \,}
\newcommand{\ppx}{\frac{\partial }{\partial x} \,}
\newcommand{\ppy}{\frac{\partial }{\partial y} \,}
\renewcommand{\thesection}{\Roman{section}}
\newcommand{\vi}{\vec{i}}
\newcommand{\vj}{\vec{j}}
\newcommand{\vk}{\vec{k}}
\newcommand{\vv}{\vec{v}}
\newcommand{\lan}{\left\langle}
\newcommand{\ran}{\right\rangle}
\newcommand{\degr}{^{\circ}}

% *** Define the printed question style ***
\newcommand{\q}[1]{ {\em #1} }
% \renewcommand{\q}[1]{ {} }

\newcommand{\notice}{ \begin{center}Some problems and solutions
    selected or adapted from \\ Stewart {\em Calculus-Early
      Transcendentals} and Hughes-Hallett {\em Calculus} .\end{center}
}

% *** Overwrite, if desired, the question format
\includeversion{Question} 
\includeversion{Solution}

\newcommand{\multicolstart}{ }
\newcommand{\multicolend}{ }

\renewenvironment{Question}
{ \begin{mdframed}[nobreak=true,hidealllines=true,backgroundcolor=gray!50,innerleftmargin=5ex] }
{ \end{mdframed} }


% *** Footnoting with symbols ***
\long\def\symbolfootnote[#1]#2{\begingroup%
\def\thefootnote{\fnsymbol{footnote}}\footnote[#1]{#2}\endgroup}

\newcommand{\WeekTitleOne}{Derivatives - Foundations}
\newcommand{\WeekTitleTwo}{Derivatives - Linearization and Applications}
\newcommand{\WeekTitleThree}{Derivatives - Modeling}
\newcommand{\WeekTitleFour}{Integrals - Foundations}
\newcommand{\WeekTitleFive}{Integrals - Techniques}
\newcommand{\WeekTitleSix}{Integrals - Modeling}
\newcommand{\WeekTitleSeven}{Differential Equations - }
\newcommand{\WeekTitleEight}{Differential Equations - }
\newcommand{\WeekTitleNine}{Differential Equations - }
\newcommand{\WeekTitleTen}{Linear Algebra - }
\newcommand{\WeekTitleEleven}{Linear Algebra - }
\newcommand{\WeekTitleTwelve}{Linear Algebra - }


\usepackage{bbding} % for Checkmarkbold
\begin{document}

\newcommand{\ub}{\underbrace}

\begin{center}
\subsection*{MNTC P01 - Week \#6 - \WeekTitleSix}
\end{center}
Chapter Contents

1. Applications of the Indefinite Integral shows how to find displacement (from velocity) and velocity (from acceleration) using the indefinite integral. There are also some electronics applications in this section.

In primary school, we learned how to find areas of shapes with straight sides (e.g. area of a triangle or rectangle). But how do you find areas when the sides are curved? We'll find out how in:

    2. Area Under a Curve and
    3. Area Between 2 Curves

wine barrel

4. Volume of Solid of Revolution explains how to use integration to find the volume of an object with curved sides, e.g. wine barrels.

5. Centroid of an Area means the centre of mass. We see how to use integration to find the centroid of an area with curved sides.

6. Moments of Inertia explains how to find the resistance of a rotating body. We use integration when the shape has curved sides.

7. Work by a Variable Force shows how to find the work done on an object when the force is not constant. This section includes Hooke's Law for springs.
Survival Tips

Before you start this section, it's a good idea to revise:

    Graph of the Quadratic Function
    Graphs of Exponential and Log Functions
    Plane Analytic Geometry
    Curve Sketching

(This chapter is easier if you can draw curves confidently.)

You may also wish to see the Introduction to Calculus.

8. Electric Charges have a force between them that varies depending on the amount of charge and the distance between the charges. We use integration to calculate the work done when charges are separated.

9. Average Value of a curve can be calculated using integration.

Head Injury Criterion is an application of average value and used in road safety research.

10. Force by Liquid Pressure varies depending on the shape of the object and its depth. We use integration to find the force. 

arc length
\hrulefill

\end{enumerate}
\end{document}

