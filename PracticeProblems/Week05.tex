
\documentclass[10pt]{article}

\usepackage{graphicx,amsmath,amssymb,subfigure,enumerate,versions}
\usepackage{multicol,multirow,mdframed}
\usepackage{epstopdf}
\usepackage{pstricks,auto-pst-pdf}
\usepackage{pst-all}
\usepackage{pst-ode}
\usepackage{pst-math}
\DeclareGraphicsExtensions{.png,.jpg,.pdf}

% ************ Page Margins *************
\hoffset=-1.3in
\setlength{\textwidth}{7.5in}
%%%%% MARGINS
\topmargin 0pt
\advance \topmargin by -\headheight
\advance \topmargin by -\headsep
\textheight 9.5in

% ************ Shortcuts *************
\newcommand{\Z}{\mbox{\sf Z\hspace{-1.5mm}Z}}
\newcommand{\SolutionSeparator}{ \hfill \hfill \hrule \hfill \hfill }
\newcommand{\R}{\mbox{\rm I\hspace{-0.75mm}R}}
\columnsep=0.75in
\newcommand{\vsc}{\vspace{1mm}}
\newcommand{\D}{\Delta }
\newcommand{\ifd}{f(x)~dx}
\newcommand{\dd}{\frac{dy}{dx} \,} 
\newcommand{\der}[2]{\frac{d{#1}}{d{#2}} \,}
\newcommand{\ddx}[1]{\frac{d {#1}}{dx} \,} 
\newcommand{\ddy}[1]{\frac{d {#1}}{dy} \,} 
\newcommand{\ddz}[1]{\frac{d {#1}}{dz} \,} 
\newcommand{\ddt}[1]{\frac{d {#1}}{dt} \,} 
\newcommand{\ds}{\displaystyle } 
\newcommand{\la}{\lambda } 
\newcommand{\del}{\nabla } 
\newcommand{\zx}{\frac{\partial z}{\partial x} \,}
\newcommand{\zy}{\frac{\partial z}{\partial y} \,}
\newcommand{\dx}{\frac{\partial f}{\partial x} \,}
\newcommand{\dy}{\frac{\partial f}{\partial y} \,}
\newcommand{\pp}[2]{\frac{\partial {#1}}{\partial {#2}} \,}
\newcommand{\ppx}{\frac{\partial }{\partial x} \,}
\newcommand{\ppy}{\frac{\partial }{\partial y} \,}
\renewcommand{\thesection}{\Roman{section}}
\newcommand{\vi}{\vec{i}}
\newcommand{\vj}{\vec{j}}
\newcommand{\vk}{\vec{k}}
\newcommand{\vv}{\vec{v}}
\newcommand{\lan}{\left\langle}
\newcommand{\ran}{\right\rangle}
\newcommand{\degr}{^{\circ}}

% *** Define the printed question style ***
\newcommand{\q}[1]{ {\em #1} }
% \renewcommand{\q}[1]{ {} }

\newcommand{\notice}{ \begin{center}Some problems and solutions
    selected or adapted from \\ Stewart {\em Calculus-Early
      Transcendentals} and Hughes-Hallett {\em Calculus} .\end{center}
}

% *** Overwrite, if desired, the question format
\input{DocumentFormat.tex}

% *** Footnoting with symbols ***
\long\def\symbolfootnote[#1]#2{\begingroup%
\def\thefootnote{\fnsymbol{footnote}}\footnote[#1]{#2}\endgroup}

\newcommand{\WeekTitleOne}{Derivatives - Foundations}
\newcommand{\WeekTitleTwo}{Derivatives - Linearization and Applications}
\newcommand{\WeekTitleThree}{Derivatives - Applications}
\newcommand{\WeekTitleFour}{Integrals - Foundations}
\newcommand{\WeekTitleFive}{Integrals - Techniques}
\newcommand{\WeekTitleSix}{Integrals - Modeling}
\newcommand{\WeekTitleSeven}{Differential Equations - }
\newcommand{\WeekTitleEight}{Differential Equations - }
\newcommand{\WeekTitleNine}{Differential Equations - }
\newcommand{\WeekTitleTen}{Linear Algebra - }
\newcommand{\WeekTitleEleven}{Linear Algebra - }
\newcommand{\WeekTitleTwelve}{Linear Algebra - }


\usepackage{bbding} % for Checkmarkbold
\begin{document}

\newcommand{\ub}{\underbrace}

\begin{center}
\subsection*{MNTC P01 - Week \#5 - \WeekTitleFive}
\end{center}


\subsection*{Substitution Integrals}

\begin{Question}
  To practice computing integrals using substitutions, do as many of
  the problems from this section as you feel you need. The problems
  trend from simple to the more complex.
  
{\bf Note}: In the solutions to these problems, we always show the
  substitution used.  On a test, if you can compute the
  antiderivative in your head, you do {\em not} need to go through
  all the steps shown here.  They are included in these solutions as
  learning \& comprehension aid.
\end{Question}

\begin{enumerate}[1.]
\begin{multicols}{2}
% **************
\item \begin{Question}$\ds \int t e^{t^2}~dt$\end{Question}
\begin{Solution}
  \begin{align*}
    \mbox{Let }w & = t^2, \mbox{ so } dw=2t~dt \mbox{ or } dt = \frac{1}{2t}~dw\\
\int t e^{t^2}~dt & = \int t e^w \left(\frac{1}{2t}~dw\right) = \int \frac{1}{2} e^w dw = \frac{1}{2} e^w + C = \frac{1}{2} e^{t^2} + C \\
\mbox{ Check: } & \ddt{} \frac{1}{2} e^{t^2} + C = \frac{1}{2} (2t) e^{t^2} = t e^{t^2} \\
& = \mbox{ original function in integral. \CheckmarkBold}
  \end{align*}
\end{Solution}
\item \begin{Question}$\ds \int e^{3x}~dx$\end{Question}
\begin{Solution}
  \begin{align*}
    \mbox{Let }w & = 3x, \mbox{ so } dw=3~dx \mbox{ or } dx = \frac{dw}{3}\\
\int e^{3x}~dx & =  \int e^w \frac{dw}{3} = \frac{1}{3} e^w + C = \frac{1}{3} e^{3x} + C \\
\mbox{ Check: } & \ddx{} \frac{1}{3} e^{3x} + C = \frac{1}{3} e^{3x}(3) =  e^{3x} \\
& = \mbox{ original function in integral. \CheckmarkBold}
\end{align*} 
\end{Solution}

{\bf NOTE: we will not show the differentiation check for
  any later questions, as the process is always the same, and you
  should be comfortable enough with the derivative rules to do the
  checking independently.  If you are uncertain about any problem,
  contact your instructor.}
\item \begin{Question}$\ds \int e^{-x}~dx$\end{Question}
\begin{Solution}
  \begin{align*}
    \mbox{Let }w & = -x, \mbox{ so } dw=-1~dx \mbox{ or } dx = (-dw) \\
    \int e^{-x}~dx & =  \int e^w (-dw) =  e^w + C = - e^{-x} + C \\
\end{align*}
\end{Solution}
%***************
\item \begin{Question}$\ds \int 25e^{-0.2t}~dt$\end{Question}
\begin{Solution}
  \begin{align*}
    \mbox{Let }w & = -0.2t, \mbox{ so } dw=-0.2~dt \\
 \mbox{ or } dt & = \frac{dw}{-0.2} = -5 dw \\
    \int 25e^{-0.2t}~dt & =  \int 25 e^w (-5 dw) = -125 e^w + C \\
&= -125 e^{-0.2t} + C \\
\end{align*}
\end{Solution}
%***************
\item \begin{Question}$\ds \int t \cos(t^2)~dt$\end{Question}
\begin{Solution}
  \begin{align*}
    \mbox{Let }w & =t^2, \mbox{ so } dw=2t~dt \\
    \mbox{ or } dt & = \frac{dw}{2t} \\
    \int t \cos(t^2)~dt & = \int  t \cos(w)~\frac{dw}{2t} = \frac{1}{2}\sin(w) + C \\
& = \frac{1}{2} \sin(t^2) + C
\end{align*}
\end{Solution}
%***************
\item \begin{Question}$\ds \int \sin(2x)~dx$\end{Question}
\begin{Solution}
\begin{align*}
  \text{Let }  w = 2x, & \text{ so } dw = 2 dx, \text{ or } dx = \frac{1}{2} dw    \\
  \int \sin(2x) dx & = \frac{1}{2} \int \sin(w) dw  = - \frac{1}{2} \cos(w) + C \\ 
& = -\frac{1}{2} \cos(2x) + C
\end{align*}
\end{Solution}
%***************
\item \begin{Question}$\ds \int \sin(3-t)~dt$\end{Question}
\begin{Solution}
  \begin{align*}
    \mbox{Let }w & =3-t, \mbox{ so } dw=-1~dt \mbox{ or } dt = -dw \\
    \int \sin(3-t)~dt & = \int \sin(w) ~(-1) dw = -( -\cos(w)) + C \\
& = \cos(3-t)  +C 
\end{align*}
\end{Solution}
%***************
\item \begin{Question}$\ds \int xe^{-x^2}~dx$\end{Question}
\begin{Solution}
\begin{align*}
  \text{Let }  w = -x^2, & \text{ so } dw = -2x ~dx, \text{ or } dx = \frac{-1}{2x} dw    \\
  \int x e^{-x^2}~dx & = \int x e^w \frac{-1}{2x} ~dw = \int \frac{-1}{2} e^w ~dw = -\frac{1}{2} e^w + C \\
& = -\frac{1}{2} e^{-x^2} + C
\end{align*}
\end{Solution}
%***************
\item \begin{Question}$\ds \int (r+1)^3~dr$\end{Question}
\begin{Solution}
\begin{align*}
  \text{Let }  w = (r+1), & \text{ so } dw = dr \\
  \int (r+1)^3~dr & = \int w^3~dw = \frac{w^4}{4} + C = \frac{(r+1)^4}{4} + C
\end{align*}
\end{Solution}
%***************
\item \begin{Question}$\ds \int y(y^2 + 5)^8~dy$\end{Question}
\begin{Solution}
\begin{align*}
  \text{Let }  w = y^2+ 5, & \text{ so } dw = 2y ~dy, \text{ or } dy = \frac{1}{2y} dw    \\
  \int y(y^2 + 5)^8~dy &  = \int y w^8 \frac{1}{2y}~dw = \frac{1}{2} \int w^8 ~dw = \frac{1}{2} \frac{w^9}{9} + C \\
& = \frac{1}{18} (y^2 + 5)^9 + C 
\end{align*}
\end{Solution}
%***************
\item \begin{Question}$\ds \int t^2(t^3-3)^{10}~dt$\end{Question}
\begin{Solution}
\begin{align*}
  \text{Let }  w = t^3 -3, & \text{ so } dw = 3t^2 ~dt, \text{ or } dt = \frac{1}{3t^2} dw    \\
  \int t^2 (t^3-3)^{10}~dt &  = \int t^2 w^{10}\frac{1}{3t^2}~dw = \frac{1}{3} \int w^{10}~dw  \\
& = \frac{1}{3} \frac{w^{11}}{11} + C 
 = \frac{1}{33} (t^3-3)^{11} + C
\end{align*}
\end{Solution}
%***************
\item \begin{Question}$\ds \int x^2(1+2x^3)^2~dx$\end{Question}
\begin{Solution}
\begin{align*}
  \text{Let }  w = 1+2x^3, & \text{ so } dw = 6x^2 ~dx, \text{ or } dx = \frac{1}{6x^2} dw    \\
  \int x^2(1+2x^3)^2~dx & = \int x^2 w^2 \frac{1}{6x^2}~dw = \frac{1}{6} \frac{w^3}{3} + C \\
& = \frac{1}{18} (1+2x^3)^3 + C
\end{align*}
\end{Solution}
%***************
\item \begin{Question}$\ds \int x(x^2 + 3)^2~dx$\end{Question}
\begin{Solution}
\begin{align*}
  \text{Let }  w = x^2+3, & \text{ so } dw = 2x ~dx, \text{ or } dx = \frac{1}{2x} dw    \\
  \int x(x^2 + 3)^2~dx & = \int x w^2 \frac{1}{2x}~dw = \frac{1}{2} \frac{w^3}{3} + C \\
& = \frac{1}{6} (x^2 + 3)^3 + C
\end{align*}

\end{Solution}
%***************
\item \begin{Question}$\ds \int x(x^2-4)^{7/2}~dx$\end{Question}
\begin{Solution}
\begin{align*}
  \text{Let }  w = x^2-4, & \text{ so } dw = 2x ~dx, \text{ or } dx = \frac{1}{2x} dw    \\
  \int x(x^2 - 4)^{7/2}~dx & = \int x w^{7/2} \frac{1}{2x}~dw = \frac{1}{2} \frac{w^{9/2}}{9/2} + C \\
& = \frac{1}{9} (x^2 - 4)^{9/2} + C
\end{align*}
\end{Solution}
%***************
\item \begin{Question}$\ds \int y^2(1+y)^2~dy$\end{Question}
\begin{Solution}

  Trick (substitution) question: substitution seems not to work well
  here, because both factors have $y^2$ in them, so neither one is the
  derivative of the other.  We're better off expanding the $(1+y)^2$
  factor and then integrating each term separately:
\begin{align*}
  \int y^2(1+y)^2 dy  & = \int y^2(1 + 2y + y^2) dy   = \int y^2 + 2y^3 + y^4\\
&    = \frac{1}{3}y^3 + \frac{2}{4}y^4 + \frac{1}{5}y^5 + C
\end{align*}
\end{Solution}
%***************
\item \begin{Question}$\ds \int (2t-7)^{73}~dt$\end{Question}
\begin{Solution}
\begin{align*}
  \text{Let }  w = 2t-7, & \text{ so } dw = 2 ~dt, \text{ or } dt = \frac{1}{2} dw    \\
  \int (2t-7)^{73}~dt & = \int  w^{73} \frac{1}{2}~dw = \frac{1}{2} \frac{w^{74}}{74} + C \\
& = \frac{1}{148} (2t-7)^{74} + C
\end{align*}
\end{Solution}
%***************
\item \begin{Question}$\ds \int \frac{1}{y+5}~dy$\end{Question}
\begin{Solution}
\begin{align*}
  \text{Let }  w = y+5, & \text{ so } dw = dy, \text{ making }  \\
  \int \frac{dy}{y+5} dy & = \int \frac{1}{w} dw = \ln | w | + C  = 
  \ln|y+5| + C
\end{align*}
\end{Solution}
%***************
\item \begin{Question}$\ds \int \frac{1}{\sqrt{4-x}}~dx$\end{Question}

\begin{Solution}
    Rewrite integral: 
  \begin{align*}
\int (4-x)^{-1/2}~dx &  \\
  \text{Let }  w = 4-x, & \text{ so } dw = -1 ~dx, \text{ or } dx = -dw    \\
  \int (4-x)^{-1/2}~dx & = \int w^{-1/2} (-1) dw = -\frac{w^{1/2}}{1/2} \\
& = -2(4-x)^{1/2} + C
\end{align*}
\end{Solution}
%***************
\item \begin{Question}$\ds \int (x^2 + 3)^2~dx$\end{Question}
\begin{Solution}

Another non-substitution integral: since there is no $x$ term outside the $(x^2 +3)$, it is easier to expand the square in this case and integrate term by term.
\begin{align*}
  \int (x^2 + 3)^2~dx & = \int x^4 + 6x^2 + 9 ~dx = \frac{x^5}{5} + \frac{6x^3}{3} + 9x + C  \\
& = \frac{x^5}{5} + 2x^3 + 9x + C  
\end{align*}
\end{Solution}
%***************
\item \begin{Question}$\ds \int x^2e^{x^3+1}~dx$\end{Question}
\begin{Solution}
\begin{align*}
  \text{Let }  w = x^3+1, & \text{ so } dw = 3x^2 ~dx, \text{ or } dx = \frac{1}{3x^2} dw    \\
  \int x^2e^{x^3 + 1}~dx & = \int x^2 e^w \frac{1}{3x^2} ~dw = \frac{1}{3} e^w + C \\
& = \frac{1}{3} e^{x^3 + 1} + C
\end{align*}
\end{Solution}
%***************
\item \begin{Question}$\ds \int \sin \theta (\cos\theta + 5)^7~d\theta$\end{Question}
\begin{Solution}
\begin{align*}
  \text{Let }  w = \cos \theta + 5, & \text{ so } dw = - \sin \theta d \theta, \text{ making }  \\
  \int \sin \theta (\cos \theta + 5)^7 d \theta & = -\int w^7 dw = -\frac{w^8}{8} + C  \\
& = -\frac{1}{8} (\cos \theta + 5)^8 + C
\end{align*}
\end{Solution}
%***************
\item \begin{Question}$\ds \int \sqrt{\cos(3t)} \sin(3t)~dt $\end{Question}
\begin{Solution}
\begin{align*}
  \text{Let }  w = \cos(3t)& \text{ so } dw = -3\sin(3t)  ~dt, \\
\text{ or } dt & =\frac{-1}{3\sin(3t)} dw    \\
  \int \sqrt{\cos(3t)} \sin(3t) ~dt 
& = \int w^{1/2} \sin(3t) \frac{-1}{3\sin(3t)}~dt \\
& = \frac{-1}{3} \frac{w^{3/2}}{3/2} + C \\
& = \frac{-2}{9} (\cos(3t))^{3/2} + C
\end{align*}
\end{Solution}
%***************
\item \begin{Question}$\ds \int \sin^6 \theta \cos \theta ~d\theta$\end{Question}
\begin{Solution}
\begin{align*}
  \text{Let }  w = \sin(\theta)& \text{ so } dw = \cos\theta  ~d\theta, \\ 
\text{ or } d\theta &= \frac{1}{\cos\theta} dw    \\
  \int (\sin\theta)^6 \cos\theta ~d\theta & =  \int w^6 \cos\theta \frac{1}{\cos\theta} dw = \frac{w^7}{7} + C \\ 
& = \frac{1}{7} \sin^7\theta  + C
\end{align*}
\end{Solution}
%***************
\item \begin{Question}$\ds \int \sin^3\alpha \cos \alpha d\alpha$\end{Question}
\begin{Solution}
\begin{align*}
  \text{Let }  w = \sin(\alpha)& \text{ so } dw = \cos\alpha  ~d\alpha, \text{ or } d\alpha = \frac{1}{\cos\alpha} dw    \\
  \int (\sin\alpha)^3 \cos\alpha ~d\alpha & =  \int w^3 \cos\alpha \frac{1}{\cos\alpha} dw = \frac{w^4}{4} + C \\
& = \frac{1}{4} \sin^4\alpha  + C
\end{align*}

\end{Solution}
%***************
\item \begin{Question}$\ds \int \sin^6(5\theta) \cos(5\theta)~d\theta$\end{Question}
\begin{Solution}
\begin{align*}
  \text{Let }  w = \sin(5\theta)& \text{ so } dw = 5\cos(5\theta)  ~d\theta, \\
 \text{ or } d\theta & = \frac{1}{5\cos(5\theta)} dw    \\
  \int (\sin(5\theta))^6 \cos(5\theta)~d\theta & =  \int w^6 \cos(5\theta) \frac{1}{5\cos(5\theta)} dw \\
& = \frac{1}{5} \frac{w^7}{7} + C = \frac{1}{35} \sin^7(5\theta)  + C
\end{align*}
\end{Solution}
%***************
\item \begin{Question}$\ds \int \tan(2x)~dx$\end{Question}

\begin{Solution}
    Rewrite integral: 
  \begin{align*}
\int \frac{\sin(2x)}{\cos{2x}}~dx &  \\
    \text{Let }  w = \cos(2x), & \text{ so } dw = -2\sin(2x) ~dx, \\
\text{ or } dx & = \frac{-1}{2\sin(2x)} ~dw    \\
    \int \frac{\sin(2x)}{\cos(2x)}~dx & = \int \frac{\sin(2x)}{w}
    \frac{-1}{2 \sin(2x)} ~dw = \frac{-1}{2} \int w^{-1}~dw \\
& = \frac{-1}{2} \ln(|w|) + C = \frac{-1}{2} \ln( |\cos(2x)|) + C
  \end{align*}

\end{Solution}
%***************
\item \begin{Question}$\ds \int \frac{(\ln z)^2}{z} ~dz$\end{Question}
\begin{Solution}
  \begin{align*}
    \text{Let }  w = \ln(z), & \text{ so } dw = \frac{1}{z} ~dz, \text{ or } dz = z ~dw    \\
\int \frac{(\ln z)^2}{z}~dx & = \int \frac{w^2}{z} (z ~dw) = \frac{w^3}{3} + C \\
& = \frac{(\ln z)^3}{3} + C
  \end{align*}
\end{Solution}
%***************
\item \begin{Question}$\ds \int \frac{e^t +1}{e^t+t}~dt$\end{Question}
\begin{Solution}
  \begin{align*}
    \text{Let }  w = e^t+t, & \text{ so } dw = (e^t + 1) ~dt, \text{ or } dt = \frac{1}{e^t + 1} ~dw    \\
\int \frac{e^t + 1}{e^t + t} ~dt & = \int \frac{e^t + 1 }{w} \frac{1}{e^t + 1}~dw = \int w^{-1}~dw \\ 
& = \ln(|w|) + C = \ln(|e^t + t|) + C
  \end{align*}
\end{Solution}
%***************
\item \begin{Question}$\ds \int \frac{y}{y^2 + 4}~dy$\end{Question}
\begin{Solution}
  \begin{align*}
    \text{Let }  w = y^2 + 4, & \text{ so } dw = 2y ~dy, \text{ or } dy = \frac{1}{2y} ~dw    \\
\int \frac{y}{y^2 + 4} ~dy & = \int \frac{y}{w} \frac{1}{2y}~dw = \frac{1}{2} \int w^{-1}~dw \\
& = \frac{1}{2}\ln(|w|) + C = \frac{1}{2}\ln(|y^2 + 4|) + C
  \end{align*}
  Note that $y^2 + 4$ is always positive, so we could remove the
  absolute values if we wished, as they are redundant in this case.
\end{Solution}
%***************
\item \begin{Question}$\ds \int \frac{\cos(\sqrt{x})}{\sqrt{x}}~dx$\end{Question}
\begin{Solution}
  \begin{align*}
    \text{Let }  w = \sqrt{x}, & \text{ so } dw =\frac{1}{2} x^{-1/2}~dx, \text{ or } dx = 2 \sqrt{x} ~dw    \\
\int \frac{\cos(\sqrt{x})}{\sqrt{x}} ~dx & = \int \frac{\cos(w)}{\sqrt{x}} (2 \sqrt{x}~dw) = 2 \int \cos(w)~dw \\
& = 2 \sin(w) + C = 2 \sin(\sqrt{x}) + C 
  \end{align*}
\end{Solution}
%***************
\item \begin{Question}$\ds \int \frac{e^{\sqrt{y}}}{\sqrt{y}}~dy$\end{Question}
\begin{Solution}
  \begin{align*}
    \text{Let }  w = \sqrt{y}, & \text{ so } dw =\frac{1}{2} y^{-1/2}~dy, \text{ or } dy = 2 \sqrt{y} ~dw    \\
\int \frac{e^{\sqrt{y}}}{\sqrt{y}} ~dy & = \int \frac{e^w}{\sqrt{y}} (2 \sqrt{y}~dw) = 2 \int e^w~dw \\
&= 2 e^w + C = 2 e^{\sqrt{y}} + C
\end{align*}
\end{Solution}
%***************
\item \begin{Question}$\ds \int \frac{1+e^x}{\sqrt{x+e^x}}~dx$\end{Question}
\begin{Solution}
  \begin{align*}
    \text{Let }  w = x + e^x, & \text{ so } dw =(1 + e^x) ~dx, \text{ or } dx = \frac{1}{1 + e^x} ~dw    \\
\int \frac{1 + e^x}{\sqrt{x+e^x}} ~dx & = \int \frac{1+e^x}{\sqrt{w}} \left(\frac{1}{1+e^x} ~dw\right) = \int w^{-1/2} ~dw \\
& = \frac{w^{1/2}}{1/2}+C = 2 \sqrt{x + e^x} + C
\end{align*}
\end{Solution}
%***************
\item \begin{Question}$\ds \int \frac{e^x}{2+e^x}~dx$\end{Question}
\begin{Solution}
  \begin{align*}
    \text{Let }  w = 2 + e^x, & \text{ so } dw =e^x ~dx, \text{ or } dx = \frac{1}{e^x} ~dw    \\
\int \frac{e^x}{2+e^x} ~dx & = \int \frac{e^x}{w} \left(\frac{1}{e^x} ~dw\right) = \int w^{-1} ~dw \\
& = \ln(|w|)+C   = \ln(|2 + e^x|)+C 
\end{align*}
\end{Solution}
%***************
\item \begin{Question}$\ds \int \frac{x+1}{x^2+2x+19}~dx$\end{Question}
\begin{Solution}
  \begin{align*}
    \text{Let }  w = x^2 + 2x + 19, & \text{ so } dw =(2x + 2) ~dx, \\
\text{ or } dx & = \frac{1}{2(x+1)} ~dw   
\end{align*}
\begin{align*}
\int \frac{x+1}{x^2 + 2x + 19} ~dx & = \int \frac{x+1}{w} \left(\frac{1}{2(x+1)} ~dw\right) \\
& = \frac{1}{2}\int w^{-1} ~dw  = \frac{1}{2} \ln(|w|)+C \\
& = \frac{1}{2}\ln(|x^2 + 2x + 19|)+C 
\end{align*}
\end{Solution}
%***************
\item \begin{Question}$\ds \int \frac{t}{1+3t^2}~dt$\end{Question}
\begin{Solution}
  \begin{align*}
    \text{Let }  w = 1 + 3t^2, & \text{ so } dw =6t ~dt, \text{ or } dt = \frac{1}{6t} ~dw    \\
\int \frac{t}{1+3t^2} ~dt & = \int \frac{t}{w} \left(\frac{1}{6t} ~dw\right) =  \frac{1}{6} \int w^{-1} ~dw \\
& = \frac{1}{6} \ln(|w|)+C = \frac{1}{6} \ln(|1 + 3t^2|)+C 
\end{align*}
In this case we can remove the absolute values because $1 + 3t^2$
will always be positive, so the absolute values are redundant.
\end{Solution}
%***************
\item \begin{Question}$\ds \int \frac{e^x-e^{-x}}{e^x + e^{-x}}~dx$\end{Question}
\begin{Solution}
  \begin{align*}
    \text{Let }  w = e^x + e^{-x}, & \text{ so } dw =(e^x - e^{-x}) ~dx, \\
\text{ or } dx & = \frac{1}{e^x - e^{-x}} ~dw    \\
\int \frac{e^x - e^{-x}}{e^x+e^{-x}} ~dx & = \int \frac{e^x-e^{-x}}{w} \left(\frac{1}{e^x-e^{-x}} ~dw\right) \\
& = \int w^{-1} ~dw = \ln(|w|)+C \\
& = \ln(|e^x+e^{-x}|)+C 
\end{align*}
In this case we can remove the absolute values because $e^x + e^{-x}$
will always be positive, so the absolute values are redundant.
\end{Solution}
%***************
\item \begin{Question}$\ds \int \frac{(t+1)^2}{t^2}~dt$ \end{Question}
\begin{Solution}
  This question is probably more easily solved by expanding than by using substitution.
  \begin{align*}
    \int\frac{(t+1)^2}{t^2}~dt&= \int \frac{t^2 +2t + 1}{t^2}~dt
= \int 1 + \frac{2}{t} + \frac{1}{t^2} ~dt 
\\
& = t + 2\ln(|t|) -t^{-1} + C \\
& = t + 2\ln(|t|) -\frac{1}{t} + C
  \end{align*}

\end{Solution}
%***************
\item \begin{Question}$\ds \int \frac{x \cos(x^2)}{\sqrt{\sin(x^2)}}~dx$\end{Question}
\begin{Solution}
  \begin{align*}
    \mbox{Let }w & =\sin(x^2), \mbox{ so } dw=2x \cos(x^2)~dx \mbox{ or } dx \\
& = \frac{dw}{2x \cos(x^2)} \\
\end{align*}
\begin{align*}
    \int \frac{x \cos(x^2)}{\sqrt{\sin(x^w)}}~dx & = \int  x \cos(x^2) (w^{-1/2} \frac{1}{2x \cos(x^2)}dw \\
& = \frac{1}{2} \int w^{-1/2}~dw = \frac{1}{2} \frac{w^{1/2}}{1/2} + C \\
& = \sqrt{\sin(x^2)} + C
\end{align*}
\end{Solution}



%*****************
\item
  \begin{Question}
$\ds \int_0^\pi \cos(x + \pi) ~dx$    
  \end{Question}

  \begin{Solution}
\begin{align*}    
\ds \int_0^\pi \cos(x + \pi) ~dx & = \sin(x + \pi)\Big|_0^\pi = \sin(2 \pi) - \sin(\pi) \\
& = 0 - 0 = 0
\end{align*}    
  \end{Solution}

%*****************
\item
  \begin{Question}
   $\ds \int_0^{1/2} \cos(\pi x)~dx$ 
  \end{Question}
  \begin{align*}
   \ds \int_0^{1/2} \cos(\pi x)~dx & = \frac{1}{\pi} \sin(\pi x) \Big|_0^{1/2}  \\
& = \frac{1}{\pi}\left[ \sin(\pi/2) - \sin(0)\right] = \frac{1}{\pi}[1-0] = \frac{1}{\pi}
  \end{align*}
  \begin{Solution}
    
  \end{Solution}

%*****************
\item
  \begin{Question}
   $\ds \int_0^{\pi/2} e^{-\cos(\theta)} \sin(\theta)~d\theta $
  \end{Question}

  \begin{Solution}
In this solution, we will use the substitution just to find the antiderivative, but
then we will switch back to the original integral variable $\theta$ to evaluate the limits.
    \begin{align*}
    \mbox{Let }w & =-\cos(\theta), \mbox{ so } dw= \sin(\theta)~d\theta \mbox{ or } d\theta 
 = \frac{dw}{\sin(\theta)} 
\end{align*}
\begin{align*}
  \int_0^{\pi/2} e^{-\cos(\theta)} \sin(\theta)~d\theta 
  & = \int_{\theta = 0 }^{\theta = \pi/2} e^{w}\sin(\theta) \left( \frac{dw}{\sin(\theta)}\right)  \\
   & = \int_{\theta = 0 }^{\theta = \pi/2} e^{w} ~dw 
   = e^{w} \Big|_{\theta = 0 }^{\theta = \pi/2}  \\
  & = e^{-\cos(\theta)} \Big|_{\theta = 0 }^{\theta = \pi/2}  \\
& = e^0 - e^{-1} = 1 - \frac{1}{e}
\end{align*}
    
  \end{Solution}

%*****************
\item \label{q:xex}
  \begin{Question}
   $\ds \int_1^2 2xe^{x^2}~dx$ 
    
  \end{Question}

  \begin{Solution}
    In this solution, we will convert the limits of integration to the substitution variable. 
    \begin{align*}
    \mbox{Let }w & =x^2, \mbox{ so } dw= 2x~dx \mbox{ or } dx = (1/2x) dw \\
\mbox{ then also } x& =1 \rightarrow w = 1^2 = 1,  \\
\mbox{and } x& =2 \rightarrow w = 2^2 = 4. 
\end{align*}
    \begin{align*}
\int_{x=1}^{x=2} 2xe^{x^2}~dx
& = \int_{w=1}^{w=4} 2xe^{w} \left( \frac{1}{2x} ~dw \right) \\
& = \int_{w=1}^{w=4} e^{w} ~dw 
 = e^{w} \Big|_{w=1}^{w=4} \\
& = e^{4} - e^{1} 
    \end{align*}
  \end{Solution}

%*****************
\item
  \begin{Question}
   $\ds \int _1^8 \frac{e^{\sqrt[3]{x}}}{\sqrt[3]{x^2}}~dx$
    
  \end{Question}

  \begin{Solution}
    In this solution, we will convert the limits of integration to the substitution variable. 
    \begin{align*}
    \mbox{Let }w & =x^{1/3}, \mbox{ so } dw= \frac{1}{3}x^{-2/3}~dx \mbox{ or } dx = 3 x^{2/3} dw \\
\mbox{ then also } x& =1 \rightarrow w = (1)^{1/3} = 1,  \\
\mbox{and } x& =8 \rightarrow w = 8^{1/3} = 2. 
\end{align*}
\begin{align*}
  \ds \int _{x=1}^{x=8} \frac{e^{\sqrt[3]{x}}}{\sqrt[3]{x^2}}~dx 
& = \int_{w=1}^{w=2} \frac{e^{w}}{x^{2/3}}~\left(  3 x^{2/3} ~dw\right)  \\
& = \int_{w=1}^{w=2} 3 e^{w}~dw 
= 3 e^{w} \Big|_{w=1}^{w=2} 
= 3 e^{2} - 3 e^{1}
\end{align*}
  \end{Solution}

%*****************
\item
  \begin{Question}
   $\ds \int_{-1}^{e-2} \frac{1}{t+2}~dt$
  \end{Question}

  \begin{Solution}
    \begin{align*}
    \mbox{Let }w & =t+2, \mbox{ so } dw= dt  \\
\mbox{ then also } t& =-1 \rightarrow w = (-1) + 2 = 1 \\
\mbox{and } t& =e-2 \rightarrow w = (e-2) + 2 = e 
\end{align*}
\begin{align*}
   \int_{t=-1}^{t=e-2} \frac{1}{t+2}~dt
   & = \int_{w=1}^{w=e} \frac{1}{w}~dw 
= \ln(|w|) \Big|_{w=1}^{w=e} \\
& = \ln(e) - \ln(1) = 1 - 0 = 1  
\end{align*}
  \end{Solution}

%*****************
\item
  \begin{Question}
   $\ds \int_1^4 \frac{\cos{\sqrt{x}}}{\sqrt{x}}~dx$ 
    
  \end{Question}

  \begin{Solution}
    \begin{align*}
    \mbox{Let }w & =x^{1/2} \mbox{ so } dw= \frac{1}{2} x^{-1/2}~dx \mbox{ or } dx = 2 x^{1/2} ~dw  \\
\mbox{ then also } x& =1 \rightarrow w = (1)^{1/2} = 1 \\
\mbox{and } x& =4 \rightarrow w = (4)^{1/2} = 2 
    \end{align*}
    \begin{align*}
    \int_{x=1}^{x=4} \frac{\cos{\sqrt{x}}}{\sqrt{x}}~dx
& =     \int_{w=1}^{w=2} \frac{\cos(w)}{x^{1/2} } (2 x^{1/2} ~dw)  \\
& =     \int_{w=1}^{w=2} 2\cos(w) ~dw 
= 2 \sin(w) \Big|_{w=1}^{w=2}  \\
& = 2 \sin(2) - 2 \sin(1)
    \end{align*}
  \end{Solution}

%*****************
\item
  \begin{Question}
   $\ds \int_0^2 \frac{x}{(1+x^2)^2}~dx$
    
  \end{Question}
  \begin{Solution}
    \begin{align*}
    \mbox{Let }w & =1 + x^2 \mbox{ so } dw= 2x ~dx \mbox{ or } dx = \frac{1}{2x} ~dw  \\
\mbox{ then also } x& =0 \rightarrow w = 1 + 0^2 = 1 \\ 
\mbox{and } x& =2 \rightarrow w = 1 + 2^2 = 5 
    \end{align*}
    \begin{align*}
      \int_{x=0}^{x=2} \frac{x}{(1+x^2)^2}~dx
& = \int_{w=1}^{w=5} \frac{x}{w^2} \left(\frac{1}{2x} ~dw \right)   \\
& =  \frac{1}{2} \int_{w=1}^{w=5} \frac{1}{w^2}  ~dw 
= \frac{1}{2} \left(\frac{-1}{w}\right) \Big|_{w=1}^{w=5}   \\
& = \frac{1}{2} \left[ \frac{-1}{5} - \frac{-1}{1} \right] 
= \frac{1}{2} \frac{4}{5} = \frac{2}{5} 
    \end{align*}
  \end{Solution}
%*********************
\item 
\begin{Question}
  If appropriate, evaluate the following integrals by substitution.
  If substitution is not appropriate, say so, and do not evaluate.

\begin{enumerate}[(a)]
\item $\ds \int x \sin(x^2)~dx$
\item $\ds \int x^2 \sin(x) ~dx$
\item $\ds \int \frac{x^2}{1+x^2} ~dx$
\item $\ds \int \frac{x}{(1+x^2)^2} ~dx$
\item $\ds \int x^3 e^{x^2} ~dx$
\item $\ds \int \frac{\sin(x)}{2 + \cos(x)}~dx$
\end{enumerate}

\end{Question}

  \begin{Solution}
    \begin{enumerate}[(a)]
    \item This integral can be evaluated using integration by
      substitution. We use $w = x^2$, $dw = 2x~dx$.
      \begin{align*}
        \int x \sin(x^2)~ dx 
        & = \frac{1}{ 2} \int \sin(w)~dw   \\
        & = - \frac{1}{ 2} \cos(w) + C 
        = \frac{-1}{ 2} \cos(x^2) + C
      \end{align*}
    \item This integral cannot be evaluated using a simple integration by substitution.
    \item This integral cannot be evaluated using a simple integration by substitution.
    \item This integral can be evaluated using integration by substitution. We use $w = 1 + x^2$, $dw = 2x~dx$.
      \begin{align*}
        \int \frac{x}{ (1 + x^2)^2} dx  
        & = \frac{1}{2} \int\frac{ 1}{ w^2}~ dw  
         = \frac{1}{2} \left(\frac{-1}{ w }\right) + C  \\
        & = \frac{-1}{2(1 + x^2)} + C.
      \end{align*}
    \item This integral cannot be evaluated using a simple integration by substitution.
    \item This integral can be evaluated using integration by substitution. We use $w = 2 + \cos x, dw = -\sin x~dx.$
      \begin{align*}
\int \frac{\sin x}{2 + \cos x }~dx  
& = - \int \frac{1}{ w}~ dw  
 = -\ln |w| + C  \\
& = -\ln |2 + \cos x| + C:   
      \end{align*}
    \end{enumerate}
  \end{Solution}


%*****************
\item \label{q:area_start} 
  \begin{Question}
    Find the exact area under the graph of $f(x) = xe^{x^2}$ between
    $x=0$ and $x=2$.
  \end{Question}

  \begin{Solution}
    Since $f(x)$ is always positive, the area under the graph is equal
    to the integral of $f(x)$ on that interval.  Thus we need to
    evaluate the definite integral
    \begin{align*}
\int^2_0 xe^{x^2}~dx.
\end{align*}
This is done in Question \ref{q:xex}, using the substitution $w = x^2$, the result being
    \begin{align*}
\int^2_0 xe^{x^2} ~dx 
= \frac{1}{2} (e^4 - 1).
\end{align*}
  \end{Solution}

%*****************
\item
  \begin{Question}
    Find the exact area under the graph of $f(x) = 1/(x+1)$ between
    $x=0$ and $x=2$.
  \end{Question}

  \begin{Solution}
   Since $f(x) = 1/(x + 1)$ is positive on the interval $x = 0$ to $x = 2$, the
area under the graph and the integral are equal to one another.
\begin{align*}
\int^2_0 \frac{1}{ x + 1}~ dx 
& = \ln(x + 1) \Big|^2_0 
= \ln 3 - \ln 1 = \ln 3.
\end{align*}

The area is $\ln 3 \approx 1.0986$. 
  \end{Solution}

%*****************
\item
  \begin{Question}
 Find $\ds \int 4x( x^2 + 1)~dx$ using two methods: 
 \begin{enumerate}[(a)]
\item Do the multiplication first, and then antidifferentiate. 
\item Use the substitution $w = x^2 + 1$. 
\item Explain how the expressions from parts ( a) and ( b) are different. Are they both correct?   
 \end{enumerate}
  \end{Question}

  \begin{Solution}
    \begin{enumerate}[(a)]
    \item $\ds \int 4x(x^2 + 1)~dx = \int (4x^3 + 4x)~dx = x^4 + 2x^2 + C.$
\item If $w = x^2 + 1$ then $dw = 2x~dx$:
  \begin{align*}
\int 4x(x^2 + 1)~dx = \int 2w ~dw = w^2 + C = (x^2 + 1)^2 + C.
  \end{align*}
\item The expressions from parts (a) and (b) look different, but they
  are both correct. Note that the answer from (b) can be expanded
  as $$(x^2 + 1)^2 + C = x^4 + 2x^2 +\underbrace{1+C}_{\mbox{ new const.}}. $$In other words, the
  expressions from parts (a) and (b) differ only by a constant, so
  they are both correct antiderivatives.
    \end{enumerate}
  \end{Solution}

%*****************
\item
  \begin{Question}
    \begin{enumerate}[(a)]
\item Find $\ds \int \sin\theta \cos \theta ~d\theta$
\item You probably solved part (a) by making the substitution $w = \sin\theta$ or $w = \cos\theta$. (If not, go back and do it that way.) Now find $\ds \int \sin \theta \cos\theta ~d\theta$ by making the other substitution. 
\item There is yet another way of finding this integral which involves
  the trigonometric identities: 
\begin{align*}
\sin(2\theta) &= 2\sin\theta \cos \theta \\
\cos(2\theta) &= \cos^2 \theta - \sin^2 \theta.
\end{align*} 
Find $\ds \int \sin \theta \cos \theta d\theta$ using one of these
identities and then the substitution $w = 2\theta$.
\item You should now have three different expressions for the
  indefinite integral $\ds \int \sin \theta \cos\theta d\theta$. Are they really different?
  Are they all correct? Explain.
    \end{enumerate}
  \end{Question}

  \begin{Solution}
    \begin{enumerate}[(a)]
    \item We first try the substitution $w = \sin \theta$, so $dw =
      \cos \theta ~d\theta$. Then
    \begin{align*}
\int \sin \theta \cos \theta ~d\theta =
\int w ~dw = \frac{w^2}{ 2} + C = \frac{\sin^2 \theta}{ 2} + C.
\end{align*}
\item If we instead try the substitution $w = \cos \theta$, $dw = -\sin \theta ~d\theta$, we get
\begin{align*}
\int \sin \theta \cos \theta ~d\theta 
= - \int w ~dw 
= - \frac{w^2}{ 2} + C 
= - \frac{\cos^2 \theta}{ 2} + C.
\end{align*}
\item Once we note that $\sin(2\theta) = 2 \sin \theta \cos \theta$ we can also say
\begin{align*}
\int \sin \theta \cos \theta ~d\theta 
= \frac{1}{ 2} \int \sin (2\theta) ~d\theta
\end{align*}
Substituting $w = 2\theta$,  $dw = 2 ~d\theta$, the above equals
\begin{align*}
\frac{1}{ 4} \int \sin w ~dw 
= - \frac{\cos w}{ 4} + C 
= - \frac{\cos 2\theta} 4 + C.
\end{align*}
\item All these answers are correct. Although they have different
  forms, they differ from each other only in terms of a constant, and
  thus they are all acceptable antiderivatives.

For example, 
\begin{align*}
1-\cos^2 \theta & = \sin^2 \theta  \\
\mbox{ so } \underbrace{\frac{\sin^2 \theta}{ 2 }}_{\mbox{Answer (a)}}
&  = -\frac{\cos^2 \theta + 1 }{2}
& = \underbrace{-\frac{\cos^2 \theta }{2}}_{\mbox{Answer (b)}} - \frac{1}{2}
\end{align*}
Thus the first two expressions differ only by a constant.

Similarly, $\cos(2\theta) = \cos^2 \theta - \sin^2 \theta = 2 \cos^2 \theta - 1$, so 
\begin{align*}
\underbrace{-\frac{\cos (2\theta)}{ 4}}_{\mbox{Answer (c)}}
= \underbrace{\frac{-\cos^2 \theta}{ 2}}_{\mbox{Answer (b)}} +\frac{ 1}{4}
\end{align*}
and thus the second and third expressions differ only by a
constant. Of course, if the first two expressions and the last two
expressions differ only in the constant C, then the first and last
only differ in the constant as well, so they are all equally valid
antiderivatives.
    \end{enumerate}
  \end{Solution}
\end{multicols}

\hrulefill
\subsection*{Substitution Integrals - Applications}

\begin{multicols}{2}
%*****************
\item
  \begin{Question}
    Let $f(t)$ be the rate of flow, in cubic meters per hour, of a
    flooding river at time $t$ in hours. Give an integral for the total
    flow of the river:
    \begin{enumerate}
    \item Over the 3-day period, $0 \le t \le 72$ (since $t$ is
      measured in hours).
    \item In terms of time $T$ in {\bf days} over the same 3-day
      period.
    \end{enumerate}
  \end{Question}

  \begin{Solution}
    \begin{enumerate}[(a)]
    \item A time period of $\D t$ hours with flow rate of $f(t)$ cubic
      meters per hour has a flow of $f(t) \D t$ cubic meters.  Summing
      the flows, we get total flow $\approx \sum f(t) \D t$, so 
      \begin{align*}
\mbox{Total flow } = \int^{72}_0 f(t) ~dt \mbox{ cubic meters} 
      \end{align*}
    \item Since 1 day is 24 hours, $t = 24T$. The constant 24 has
      units (hours/day), so 24$T$ has units (hours/day) $\times$ day =
      hours.  Applying the substitution $t = 24T$ to the integral in
      part (a), we get
      \begin{align*}
\mbox{Total flow }= \int^3_0 24 f(24T) ~dT \mbox{ cubic meters.}
      \end{align*}
    \end{enumerate}
  \end{Solution}

%*****************
\item
  \begin{Question}
    Oil is leaking out of a ruptured tanker at the rate of $r( t) =
    50e^{- 0.02t}$ thousand liters per minute.
    \begin{enumerate}[(a)]
    \item At what rate, in liters per minute, is oil leaking out at $t
      = 0$? At $t = 60$? 
    \item How many liters leak out during the first hour?
    \end{enumerate}
    
  \end{Question}

  \begin{Solution}
    \begin{enumerate}[(a)]
    \item At time $t = 0$, the rate of oil leakage = $r(0)$ = 50
      thousand liters/minute.  \\
      At $t = 60$, rate = $r(60)$ = 15.06 thousand liters/minute.

\item To find the amount of oil leaked during the first hour, we integrate the rate from $t = 0$ to $t = 60$:
  \begin{align*}
\mbox{Oil leaked } &=
\int^{60}_0 50e^{-0.02t} dt 
= \left( \frac{- 50}{ 0.02} e^{-0.02t} \right) \Big|^{60}_{0} \\
&= -2500e^{-1.2} + 2500e^0 \\
& \approx 1747 \mbox{ thousand liters}.
  \end{align*}
    \end{enumerate}
  \end{Solution}

%*****************
\item
  \begin{Question}
    If we assume that wind resistance is proportional to velocity,
    then the downward velocity, $v$, of a body of mass $m$ falling
    vertically is given by $$v = \frac{mg}{ k} (1 - e^{- kt/ m})$$
    where $g$ is the acceleration due to gravity and $k$ is a
    constant. Find the height of the body, $h$, above the surface of
    the earth as a function of time. Assume the body starts at height
    $h_0$.
  \end{Question}

  \begin{Solution}
Since $\ds v = dh/dt$, it follows that $h(t) =
\int
v(t)~dt$ and $h(0) = h_0$. Since
\begin{align*}
v(t) &= \frac{mg}{k} \left( 1 - e^{- \frac{k}{m} t} \right) \\
&= \frac{mg}{k} - \frac{mg}{k} e^{-\frac{k}{m} t}
\end{align*}
we have
\begin{align*}
h(t) & = \int v(t)~dt \\
& = \frac{mg}{k} \int dt - \frac{mg}{k} \int e^{-k/m} t~dt.
\end{align*}
The first integral is simply $\ds \frac{mg}{k} t + C$. 

To evaluate the second integral, make the substitution $\ds w = -\frac{ k}{ m}t$. Then
$\ds dw = - \frac{k}{m}~dt,$
so 
\begin{align*}
\int e^{-kt/m}~dt 
&= \int e^w \left(\frac{-m}{k} ~dw\right) \\
&= - \frac{m}{k}  e^w + C  \\
&= - \frac{m}{k}  e^{-kt/m }+ C.
\end{align*}

Thus
\begin{align*}
h(t) 
&= \int v~dt  \\
&= \frac{mg}{k} t - \frac{mg}{k} \left( - \frac{m}{k} e^{-kt/m} \right) + C \\
&= \frac{mg}{k} t + \frac{m^2g} {k^2} e^{-kt/m} + C.
\end{align*}
 Since $h(0) = h_0$,
 \begin{align*}
h_0 & = \frac{mg}{k} \cdot 0 + \frac{m^2g}{ k^2} e^0 + C;  \\
C & = h_0 - \frac{m^2g}{ k^2}. \\
\end{align*}
 Thus
\begin{align*}
h(t) &= \frac{mg}{k} t + \frac{m^2g}{k^2} e^{-kt/m}  - \frac{m^2g}{k^2} + h_0 \\
h(t) &= \frac{mg}{k} t - \frac{m^2g}{k^2} \left( 1 - e^{-kt/m}  \right) + h_0.
 \end{align*}
This formula gives the height of the object above the surface of the earth as it falls.
  \end{Solution}

%*****************
\item
  \begin{Question}
    The rate at which water is flowing into a tank is $r(t)$
    gallons/minute, with $t$ in minutes.
    \begin{enumerate}[(a)]
    \item Write an expression approximating the amount of water
      entering the tank during the interval from time $t$ to time $t+
      \D t$, where $\D t$ is small.
    \item Write a Riemann sum approximating the total amount of water
      entering the tank between $t = 0$ and $t = 5$. Then write an
      exact expression for this amount.
    \item By how much has the amount of water in the tank changed
      between $t = 0$ and $t = 5$ if $r( t) = 20e^{0.02t}$?
    \item If $r( t)$ is as in part ( c), and if the tank contains 3000
      gallons initially, find a formula for $Q( t)$, the amount of
      water in the tank at time $t$.
    \end{enumerate}
  \end{Question}

  \begin{Solution}
    \begin{enumerate}[(a)]
    \item  Amount of water entering tank in a short period of time = Rate$\times$Time = $r(t) \D t$.
    \item Summing the contribution from each of the small intervals $\D t$:
\begin{align*}
&       \mbox{Amount of water entering the tank } \\
   \approx & \sum_{i} r(t_i) \D t_i 
\end{align*}where $\D t = 5/n$.
Taking a limit as $\D t \to 0$, we get the integral form instead of the sum:
\begin{align*}
&\mbox{ Exact amount of water entering the tank} \\
& = \int^5_0 r(t) ~dt.
\end{align*}
\item If $Q(t)$ is the amount of water in the tank at time $t$, then
  $Q'(t) = r(t)$. We want to calculate net change in volume between
  $t=0$ and $t=5$, or $Q(5) - Q(0)$. By the Fundamental Theorem,
  \begin{align*}
&  \mbox{ Amount which has entered tank} \\
& = Q(5) - Q(0) \\
&  = \int^5_0 r(t)~dt  \\
& = \int^5_0 20e^{0.02t}~dt  \\
&= \frac{20 }{0.02} e^{0.02t} \Big|^5_0 \\
& = 1000(e^{(0.02)(5)} - 1) \approx 105.17 \mbox{ gallons}.
  \end{align*}
\item By the Fundamental Theorem again,
  \begin{align*}
& \mbox{ Amount which has entered tank} \\
& = Q(t) - Q(0)  \\
& = \int^t_0 r(t)~dt  \\
Q(t) - 3000 & = \int^t_0 20e^{0.02t}~dt  \\
\mbox{ so } Q(t) & = 3000 + \int^t_0 20e^{0.02u}~du \\
\end{align*}
Note: $t$ is already being used, so we put $u$ inside the integral; since
this is a definite integral, the variable inside the integral will disappear when
we sub in the limits. 
\begin{align*}
Q(t) & = 3000 + \frac{20}{ 0.02} e^{0.02t} \Big|^t_0 \\
& = 3000 + 1000(e^{0.02t} - e^0) \\
& = 1000e^{0.02t} + 2000. \\
  \end{align*}
  This is the quantity of water in the tank at time $t$.  Note that we
  can do a basic sanity check on the answer by verifying that $Q(0) =
  3000$ (given), which is true for the formula we arrived at:
$$Q(0) = 1000e^{0.02\cdot 0} + 2000 = 3000$$
    \end{enumerate}
  \end{Solution}

%*****************
\item
  \begin{Question}
    After a spill of radioactive iodine, measurements at $t=0$ showed
    the ambient radiation levels at the site of the spill to be four
    times the maximum acceptable limit. The level of radiation from an
    iodine source decreases according to the formula $$R(t) = R_0
    e^{-0.004t}$$ where $R$ is the radiation level (in millirems/
    hour) at time $t$ in hours and $R_0$ is the initial radiation
    level (at $t = $0).
    \begin{enumerate}[(a)]
  \item How long will it take for the site to reach an acceptable
    level of radiation?
  \item Engineers look up the safe limit of radiation and find it to
    be 0.6 millirems/hour.  How much total radiation (in millirems)
    will have been emitted by the time found in part (a)?
    \end{enumerate}
  \end{Question}

  \begin{Solution}
\begin{enumerate}[(a)]
\item 
    If the level first becomes acceptable at time $t_1$, then $R_0 = 4R(t_1)$, and
    \begin{align*}
\frac{1}{ 4} R_0 & = R_0e^{-0.004 t_1} \\
\frac{1}{4} & = e^{-.004t_1}
    \end{align*}
Taking natural logs on both sides yields
\begin{align*}
\ln\left( \frac{1}{ 4}\right) &= -0.004t_1 \\
t_1 & = \frac{\ln\left( \frac{1}{ 4}\right)}{-0.004} \\ 
t_1 & \approx 346.574 \mbox{ hours.}
\end{align*}
\item Since the initial radiation was four times the acceptable limit of 0.6 millirems/hour, we have $R_0 = 4(0.6) = 2.4$.
The rate at which radiation is emitted is $R(t) = R_0 e^{-.004t}$, so

Total radiation emitted = $\ds \int^{346.574}_0 2.4e^{-0.004t}~dt$.

Finding by guess-and-check or substitution that an antiderivative of
$e^{-0.004t}$ is $\ds \frac{e^{-0.004t}}{-0.004}$,
\begin{align*}
& \int^{346.574}_0 2.4e^{-0.004t}~dt \\
& = 2.4 \frac{e^{-0.004t}}{-0.004} \Bigr|_0^{346.574}
\end{align*}
\begin{align*}
& = 2.4 \left[  \frac{e^{-0.004 (346.574)}}{-0.004} - \frac{e^{0}}{-0.004}\right] \\
& = 450.00 \\
\end{align*}
We find that 450 millirems were emitted during this time.
\end{enumerate}
  \end{Solution}
% ***********
\item  
\begin{Question}David is learning about catalysts in his Chemistry course.  He has read the definition:

%\vspace*{.25cm}
\begin{quote}
Catalyst: A substance that helps a reaction to go faster without being used up in the reaction.
\end{quote}
%\vspace*{.25cm}

In today's Chemistry lab exercise, he has to add a catalyst to a
chemical mixture that produces carbon dioxide.  When there is no
catalyst, the carbon dioxide is produced at a rate of $8.37 \times
10^{-9}$ moles per second.  When $C$ moles of the catalyst are
present, the carbon dioxide is produced at a rate of $(6.15 \times
10^{-8})C + 8.37 \times 10^{-9}$ moles per second.

The reaction begins at exactly 10:00 a.m.  One minute later, at 10:01 sharp, David starts to add the catalyst at a constant rate of $0.5$ moles per second.

How much carbon dioxide is produced between 10:00 (sharp) and 10:05? 
\end{Question}
\begin{Solution}
    
In the first minute of the reaction no catalyst is present
  and the amount of carbon dioxide produced is

  \begin{align*}
    (8.37\times 10^{-9}\mbox{mol}/s)(60s)=5.02\times 10^{-7}\mbox{mol}.
  \end{align*}

  In the next four minutes, catalyst is being added. After $t$ seconds
  of adding catalyst of $0.5\mbox{mol}/s$, there is $0.5t\mbox{mol}$
  of catalyst present. Thus, at time $t$ carbon dioxide is being
  produced at a rate of

  \begin{align*}
&    (6.15\times10^{-8})(0.5t)+(8.37\times 10^{-9}\mbox{mol}/s) \\ & = 
    (3.075\times10^{-8})t+(8.37\times 10^{-9}\mbox{mol}/s).
  \end{align*}

  During the four minutes when catalyst is being added, the amount of
  carbon dioxide produced is:

  \begin{align*}
&      \int_{0}^{240}(3.076\times10^{-8})t+(8.37\times 10^{-9})dt \\ 
& = \left.(1.5375\times10^{-8})t^{2}+(8.37\times 10^{-9})t\right. \Big |_{0}^{240} \\
  & =8.88\times10^{-4}\mbox{mol}
  \end{align*}

  The total amount of $CO_{2}$ produced in the first five minutes of
  the reaction is therefore

  \begin{align*}
    5.02\times 10^{-7}+8.88\times10^{-4}=8.88\times10^{-4}\mbox{mol}.
  \end{align*}
\end{Solution}

\end{multicols}

\hrulefill

\subsection*{Integration by Parts}

\begin{multicols}{2}


%*****************
\item
  \begin{Question}
    For each of the following integrals, indicate whether integration
    by substitution or integration by parts is more appropriate.  Do
    not evaluate the integrals.

    \begin{enumerate}[(a)]
    \item $\ds \int x \sin(x)~dx$
    \item $\ds \int \frac{x^2}{1+x^3}~dx$
    \item $\ds \int x e^{x^2}~dx$
    \item $\ds \int x^2 \cos(x^3)~dx$
    \item $\ds \int \frac{1}{\sqrt{3x+1}}~dx$
    \item $\ds \int x^2 \sin x~dx$
    \item $\ds \int \ln x~dx$
    \end{enumerate}
  \end{Question}

  \begin{Solution}
    \begin{enumerate}[(a)]
    \item This integral can be evaluated using integration by parts with $u = x$, $dv = \sin x~dx$.
\item We evaluate this integral using the substitution $w = 1 + x^3$.
\item We evaluate this integral using the substitution $w = x^2$.
\item We evaluate this integral using the substitution $w = x^3$.
\item We evaluate this integral using the substitution $w = 3x + 1$.
\item This integral can be evaluated using integration by parts with $u = x^2$, $dv = \sin x~dx$.
\item This integral can be evaluated using integration by parts with $u = \ln x$, $dv = dx$.
    \end{enumerate}
  \end{Solution}


\begin{Question}
  To practice computing integrals by parts, do as many of
  the problems from this section as you feel you need. The problems
  trend from simple to the more complex. 

For Questions \#\ref{q:parts1_start} to \#\ref{q:parts1_end}, evaluate the integral.
\end{Question}
%*****************
\item  \label{q:parts1_start}
  \begin{Question}
   $\ds \int t \sin t~dt$  
  \end{Question}
  \begin{Solution}
    \begin{align*}
\mbox{ We choose }       u & = t & \mbox{ and }dv & = \sin t ~dt \\
\mbox{ so } \frac{du}{dt} & = 1 & \mbox{ and } v & = \int \sin t ~dt \\
\mbox{ or } du & = 1 ~dt& v & = -\cos(t) 
    \end{align*}
Using the integration by parts formula, 
\begin{align*}
  \int\ub{ t}_u \ub{\sin t~dt}_{dv} & = \ub{t}_u \ub{(-\cos t)}_v 
- \int \ub{(-\cos t)}_v \ub{dt}_{du} \\
& = -t \cos t + \int \cos t~dt \\
& = -t \cos t + \sin t  + C
\end{align*}
As always, we can check our integral is correct by differentiating:
\begin{align*}
  \ddt{} (-t \cos t + \sin t + C) 
& = -\cos t -t (-\sin t) + \cos t \\
& = t \sin t \\ 
& = \mbox{ original integrand.~~\CheckmarkBold}
\end{align*}
  \end{Solution}
%*****************
\item
  \begin{Question}
   $\ds \int t e^{5t}~dt$
  \end{Question}

  \begin{Solution}
    \begin{align*}
\mbox{ We choose }       u & = t &\mbox{ and } dv & = e^{5t} ~dt \\
\mbox{ so } \frac{du}{dt} & = 1 & \mbox{ and } v & = \int e^{5t}~dt \\
\mbox{ or } du & = 1 ~dt& v & = e^{5t}/5 
    \end{align*}
Using the integration by parts formula, 
\begin{align*}
  \int\ub{ t}_u \ub{e^{5t}~dt}_{dv} & = \ub{t}_u \ub{(e^{5t}/5)}_v 
- \int \ub{(e^{5t}/5)}_v \ub{dt}_{du} \\
& = \frac{t e^{5t}}{5} - \frac{1}{5}\int e^{5t}~dt \\
& = \frac{t e^{5t}}{5} - \frac{1}{5}\left(e^{5t}/5\right)  + C \\
& = \frac{t e^{5t}}{5} - \frac{1}{25} e^{5t}+ C 
\end{align*}
As always, we can check our integral is correct by differentiating:
\begin{align*}
  \ddt{} \left( \frac{t e^{5t}}{5} - \frac{e^{5t}}{25} + C\right)
& = \frac{1}{5} ( e^{5t} + t(5e^{5t})) - \frac{5e^{5t}}{25} \\
& = t e^{5t} + \frac{e^{5t}}{5}- \frac{e^{5t}}{5} \\
& = t e^{5t}  \\
& = \mbox{ original integrand.~~\CheckmarkBold}
\end{align*}
  \end{Solution}

\begin{Solution}
  From now on, for brevity, we won't show quite as many steps in the
  solution, nor will we check the answer by differentiating.  However,
  you should always remember that you {\bf can} check your
  antiderivatives/integrals by differentiation.
\end{Solution}
%*****************
\item
  \begin{Question}
   $\ds \int p e^{-0.1p}~dp$
    
  \end{Question}

\begin{Solution}
    \begin{align*}
\mbox{ We choose }       u & = p &\mbox{ and } dv & = e^{-0.1p} ~dp \\
\mbox{ so } du & = dp& \mbox{ and } v & = e^{-0.1p}/(-0.1) 
    \end{align*}
Using the integration by parts formula, 
\begin{align*}
  \int\ub{ p}_u \ub{e^{-0.1p}~dp}_{dv} & = \ub{p}_u \ub{e^{-0.1p}/(-0.1)}_v 
- \int \ub{e^{-0.1p}/(-0.1)}_v \ub{dp}_{du} \\
& = \frac{p e^{-0.1p}}{-0.1} + 10 \int e^{-0.1p}~dp \\
& = -10 p e^{-0.1p} + 10\left(e^{-0.1p}/(-0.1)\right)  + C \\
& = -10 p e^{-0.1p} - 100 e^{-0.1p}+ C 
\end{align*}

    
  \end{Solution}
%*****************
\item
  \begin{Question}
   $\ds \int (z+1)e^{2z}~dz$
  \end{Question}

  \begin{Solution}
    \begin{align*}
\mbox{ We choose }       u & = (z+1)&\mbox{ and } dv & = e^{2z} ~dz \\
\mbox{ so } du & = dz& \mbox{ and } v & = e^{2z}/2 
    \end{align*}
Using the integration by parts formula, 
\begin{align*}
  \int (z+1) e^{2z}~dz & = (z+1) e^{2z}/2
- \int e^{2z}/2 dz \\
& = \frac{(z+1) e^{2z}}{2} - \frac{1}{2} \int e^{2z}~dz \\
& = \frac{(z+1)e^{2z}}{2} - \frac{1}{2} \left(e^{2z}/2\right)  + C \\
& = \frac{(z+1) e^{2z}}{2} - \frac{1}{4} e^{2z}+ C 
\end{align*}
There is no need for further simplifications.
    
  \end{Solution}
%*****************
%*****************
\item \label{q:lnx}
  \begin{Question}
   $\ds \int \ln x~dx$
  \end{Question}

  \begin{Solution}
    In this problem, we recall that the only simple calculus formula
    related to $\ln x$ is that its {\bf derivative} is known:
    $\frac{d}{dx} \ln(x) = 1/x$.  This means that we have to select $u = \ln x$ so that
it will be differentiated.
    \begin{align*}
\mbox{ We choose }       u & = \ln x&\mbox{ and } dv & = dx \\
\mbox{ so } du & = \frac{1}{x} ~dx& \mbox{ and } v & = x 
    \end{align*}
Using the integration by parts formula, 
\begin{align*}
  \int \ln x~dx & = x (\ln x) 
- \int x \frac{1}{x}~ dx \\
& = x \ln x - \int 1 ~dx \\
& = x \ln x - x + C  
\end{align*}
    
  \end{Solution}
%*****************
\item \label{q:xlnx}
  \begin{Question}
   $\ds \int y \ln y~dy$
  \end{Question}

  \begin{Solution}
    In this problem, we recall that the only simple calculus formula
    related to $\ln y$ is that its {\bf derivative} is known:
    $\frac{d}{dy} \ln(y) = 1/y$.  While it might be tempting to keep
    with our earlier pattern of choosing $u = y$ and $dv = \ln y ~dy$,
    that won't work because we won't be able to integrate $dv$. As a
    result,
    \begin{align*}
\mbox{ we choose }       u & = \ln y&\mbox{ and } dv & = y ~dy \\
\mbox{ so } du & = \frac{1}{y} ~dy& \mbox{ and } v & = y^2/2 
    \end{align*}
Using the integration by parts formula, 
\begin{align*}
  \int y \ln y~dy & = (\ln y) (y^2/2) 
- \int (y^2/2) \frac{1}{y}~ dy \\
& = \frac{y^2 \ln y}{2} - \frac{1}{2} \int y ~dy \\
& = \frac{y^2 \ln y}{2} - \frac{1}{2} (y^2/2)  + C \\
& = \frac{y^2 \ln y}{2} - \frac{y^2}{4}  + C 
\end{align*}
    
  \end{Solution}
%*****************
%*****************
\item
  \begin{Question}
   $\ds \int x^3 \ln x~dx$
  \end{Question}

  \begin{Solution}
    
    \begin{align*}
\mbox{ We choose }       u & = \ln x&\mbox{ and } dv & = x^3 ~dx \\
\mbox{ so } du & = \frac{1}{x} ~ dx& \mbox{ and } v & = x^4/4 
    \end{align*}
Using the integration by parts formula, 
\begin{align*}
  \int x^3 \ln x~dx & = (\ln x) x^4/4 
- \int (x^4/4) \left(\frac{1}{x} ~dx\right) \\
& = \frac{x^4 \ln x}{4} - \frac{1}{4} \int x^3~dx \\
& = \frac{x^4 \ln x}{4} - \frac{1}{4} (x^4/4) + C \\
& = \frac{x^4 \ln x}{4} - \frac{x^4}{16}  + C
\end{align*}
  \end{Solution}
%*****************
%*****************
\item
  \begin{Question}
   $\ds \int q^5 \ln(5q)~dq$
  \end{Question}

  \begin{Solution}
    \begin{align*}
\mbox{ We choose }       u & = \ln(5q)&\mbox{ and } dv & = q^5 ~dq \\
\mbox{ so } du & = \frac{1}{5q} (5)~dq& \mbox{ and } v & = q^6/6 
    \end{align*}
Using the integration by parts formula, 
\begin{align*}
  \int q^5 \ln(5q)~dq & = (\ln(5q)) (q^6/6) 
- \int(q^6/6) \left(\frac{1}{q}~ dq\right) \\
& = \frac{q^6 \ln(5q)}{6} - \frac{1}{6} \int q^5~dq \\
& = \frac{q^6 \ln(5q)}{6} - \frac{1}{6} (q^6/6) + C\\
& = \frac{q^6 \ln(5q)}{6} - \frac{1}{36} q^6 + C
\end{align*}
    
  \end{Solution}
%*****************
%*****************
\item
  \begin{Question}
   $\ds \int t^2 \sin t~dt$ 
  \end{Question}

  \begin{Solution}
    \begin{align*}
\mbox{ We choose }       u & = t^2&\mbox{ and } dv & = \sin t ~dt \\
\mbox{ so } du & = 2 t~dt& \mbox{ and } v & = -\cos t
    \end{align*}
Using the integration by parts formula, 
\begin{align*}
  \int t^2 \sin t~dt & = -t^2 \cos t 
- \int (-\cos t)~ (2 t~ dt) \\
& = - t^2 \cos t + 2 \int t \cos t~dt \\
\end{align*}
While we have traded our original integral for a slightly simpler one,
it is still not simple enough to evaluate by finding an obvious
antiderivative.  In fact, it is of the form of one of our earlier
examples of integration by parts, so here we must apply integration by
parts {\em again} to finally evaluate the original integral.

We focus on just the new integral part, $\ds \int t \cos t~dt$:
    \begin{align*}
\mbox{ We choose }       u & = t&\mbox{ and } dv & = \cos t ~dt \\
\mbox{ so } du & = ~dt& \mbox{ and } v & = \sin t
    \end{align*}
Using the integration by parts formula, 
\begin{align*}
  \int t \cos t~dt & = t \sin t 
- \int (\sin t)~ dt \\
& =  t \sin t - \int \sin t~dt \\
& =  t \sin t - (-\cos t) + C \\
& = t \sin t + \cos t + C
\end{align*}
Subbing this result back into the original integral, 
\begin{align*}
  \int t^2 \sin t~dt 
& =\ub{ - t^2 \cos t + 2 \ub{\int t \cos t~dt}_{\mbox{Second by parts step}}}_{\mbox{First by parts step}} \\
& = - t^2 \cos t + 2 \left( t \sin t + \cos t\right) + C
\end{align*}
No further simplification is necessary for a pure `evaluate the integral' question.
    
  \end{Solution}
%*****************
\item
  \begin{Question}
   $\ds \int x^2 \cos(3x)~dx$
  \end{Question}

  \begin{Solution}
    \begin{align*}
\mbox{ We choose }       u & = x^2&\mbox{ and } dv & = \cos(3x) ~dx \\
\mbox{ so } du & = 2 x~dx& \mbox{ and } v & = \sin(3x)/3 
    \end{align*}
Using the integration by parts formula, 
\begin{align*}
  \int x^2 \cos(3x)~dx & = x^2 \sin(3 x)/3
- \int (\sin(3 x)/3)~ (2 x~ dx) \\
& = \frac{x^2 \sin(3x)}{3} - \frac{2}{3} \int x \sin(3 x)~dx \\
\end{align*}
While we have traded our original integral for a slightly simpler one,
it is still not simple enough to evaluate by finding an obvious
antiderivative.  In fact, it is of the form of one of our earlier
examples of integration by parts, so here we must apply integration by
parts {\em again} to finally evaluate the original integral.

We focus on just the new integral part, $\ds \int x \sin(3x)~dx$:
    \begin{align*}
\mbox{ We choose }       u & = x&\mbox{ and } dv & = \sin(3 x) ~dx \\
\mbox{ so } du & = ~dx& \mbox{ and } v & = -\cos(3 x)/3
    \end{align*}
Using the integration by parts formula, 
\begin{align*}
  \int x \sin(3x)~dx & = -x \cos(3x )/3
- \int (-\cos(3 x)/3)~ dx \\
& =  -\frac{x \cos(3x)}{3} + \frac{1}{3}\int \cos(3 x)~dx \\
& =  -\frac{x \cos(3x)}{3} + \frac{1}{3} (\sin(3 x)/3) + C \\
& =  -\frac{x \cos(3x)}{3} + \frac{1}{9} \sin(3 x) + C 
\end{align*}
Subbing this result back into the original integral, 
\begin{align*}
&   \int x^2 \cos(3 x)~dx  \\
 =&\ub{ \frac{x^2 \sin (3x)}{3} - \frac{2}{3} \ub{\int x \sin(3x) ~dx}_{\mbox{Second by parts step}}}_{\mbox{First by parts step}} \\
 =& \frac{ x^2 \sin(3x)}{3}  -\frac{2}{3} \left( -\frac{x \cos(3x)}{3} + \frac{\sin(3x)}{9} + C \right) \\
 =& \frac{ x^2 \sin(3x)}{3}  +\frac{2}{9} x \cos(3x) - \frac{2\sin(3x)}{27} + C_2  
\end{align*}
where $C_2 = -(2/3) C$ is a new integration constant.
    
  \end{Solution}
%*****************
%*****************
\item
  \begin{Question}
   $\ds \int (\ln t)^2 ~dt$
  \end{Question}

  \begin{Solution}
We only know how to differentiate $(\ln t)^2$, so we have to choose it as $u$.
    \begin{align*}
\mbox{ We choose }       u & = (\ln t)^2&\mbox{ and } dv & =  ~dt \\
\mbox{ so } du & = \frac{2 \ln t}{t}~ dt& \mbox{ and } v & = t 
    \end{align*}
Using the integration by parts formula, 
\begin{align*}
  \int (\ln t)^2 ~dt & = (\ln t)^2 t
- \int (t) \left( \frac{2 \ln t}{t}\right)~dt \\
& = t (\ln t)^2  - 2 \int \ln t~dt
\end{align*}
We are left with a simpler integral, but not an easy one (unless you
look at the examples from the course notes!)

We focus on $\ds I_2 = \int \ln t~dt$, and apply integration by parts once more.
    \begin{align*}
\mbox{ We choose }       u & = \ln t&\mbox{ and } dv & =  ~dt \\
\mbox{ so } du & = \frac{1}{t}~ dt& \mbox{ and } v & = t 
    \end{align*}
Using the integration by parts formula, 
\begin{align*}
  \int \ln t ~dt & = (\ln t) t
- \int (t) \left( \frac{1 }{t}\right)~dt \\
& = t (\ln t)  - \int 1 ~dt \\ 
& = t \ln t - t + C
\end{align*}
Thus, going back to our original integral, 
\begin{align*}
  \int (\ln t)^2 ~dt & = t (\ln t)^2  - 2 \ub{\int \ln t~dt}_{I_2} \\
& = t (\ln t)^2  - 2 (t \ln t - t + C) 
\end{align*}


    
  \end{Solution}
%*****************
%*****************
\item
  \begin{Question}
   $\ds \int t^2 e^{5t}~dt$
  \end{Question}

  \begin{Solution}
    \begin{align*}
\mbox{ We choose }       u & = t^2&\mbox{ and } dv & = e^{5t} ~dt \\
\mbox{ so } du & = 2t ~dt& \mbox{ and } v & = e^{5t}/5 
    \end{align*}
Using the integration by parts formula, 
\begin{align*}
  \int t^2 e^{5t}~dt & = t^2 e^{5t}/5
- \int (e^{5t}/5) (2t~dt) \\
& = \frac{t^2 e^{5t}}{5} - \frac{2}{5} \ub{\int t e^{5t}~dt}_{I_2} 
\end{align*}
Applying integration by parts again to the integral marked $I_2$ will lead to
\begin{align*}
  I_2 = \frac{t e^{5t}}{5} - \frac{e^{5t}}{25} + C
\end{align*}
so the overall integral will be
\begin{align*}
  \int t^2 e^{5t}~dt 
& = \frac{t^2 e^{5t}}{5} - \frac{2}{5} \ub{\int t e^{5t}~dt}_{I_2} \\
& = \frac{t^2e^{5t}}{5} - \frac{2}{5} \left(\frac{t e^{5t}}{5} - \frac{e^{5t}}{25} + C\right) \\
& = \frac{t^2 e^{5t}}{5} - \frac{2 t e^{5t}}{25} + 2 \frac{e^{5t}}{125} + C_2
\end{align*}
where $C_2$ is a multiple of the original $C$.
    
  \end{Solution}
%*****************
\item
  \begin{Question}
   $\ds \int y \sqrt{y+3}~dy$
  \end{Question}

  \begin{Solution}
    We choose $u = y$ and $dv = (y + 3)^{1/2}~dx$,\\
 so $du = dx$ and
    $\ds v = \frac{2}{3} (y + 3)^{3/2}$:
    \begin{align*}
\int y \sqrt{ y + 3}~ dy & = \frac{2}{ 3} y(y + 3)^{3/2} -
\int \frac{2}{3} (y + 3)^{3/2} ~dy \\
& = \frac{2}{3} y(y + 3)^{3/2} - \frac{2}{3}\frac{ (y + 3)^{5/2}}{5/2} + C \\
& = \frac{2}{3} y(y + 3)^{3/2} - \frac{4}{15} (y + 3)^{5/2} + C.
    \end{align*}
  \end{Solution}
%*****************
%*****************
\item
  \begin{Question}
   $\ds \int (t+2)\sqrt{2+3t}~dt$
  \end{Question}

  \begin{Solution}
    Let $u = t + 2$ and $dv = \sqrt{2 + 3t}$,\\
 so $du = dt$ and $v =
    \frac{2}{ 9} (2 + 3t)^{3/2}$.

Then
\begin{align*}
\int (t + 2)\sqrt{1 + 3t}~ dt & =
\frac{2}{9} (t + 2)(2 + 3t)^{3/2} -
\frac{2}{9} \int
(2 + 3t)^{3/2} ~dt \\
& = \frac{2}{9} (t + 2)(2 + 3t)^{3/2}  \\
& - \frac{4}{ 135} (2 + 3t)^{5/2} + C.
\end{align*}
  \end{Solution}
%*****************
%*****************
\item
  \begin{Question}
   $\ds \int (p+1)\sin(p+1)~dp$
  \end{Question}

  \begin{Solution}
    Let $u = p + 1$ and $dv = \sin(p + 1)$, \\
so $du = dx$ and $v = -\cos(p+ 1)$.
    \begin{align*}
& \int (p + 1) \sin(p + 1) ~dp \\
 = &-(p + 1) \cos(p + 1) +
\int \cos(p + 1) ~dp \\
 =&  -(p + 1) \cos(p + 1) + \sin(p+ 1) + C
    \end{align*}
  \end{Solution}
%*****************
%*****************
\item \label{q:zexpz}
  \begin{Question}
   $\ds \int \frac{z}{e^z}~dz$
  \end{Question}

  \begin{Solution}
Rewriting the integral, 
   $$\ds \int \frac{z}{e^z}~dz = \ds \int z e^{-z}~dz $$
Let $u = z$, $dv = e^{-z}~dx$. \\
Thus $du = dz$ and $v = -e^{-z}$. Integration by parts gives:
\begin{align*}
\int ze^{-z} ~dz  & = -ze^{-z} - \int (-e^{-z}) ~dz \\
& = -ze^{-z} + \int e^{-z} ~dz \\
& = -ze^{-z} - e^{-z} + C 
\end{align*}
 \end{Solution}
%*****************
%*****************
\item
  \begin{Question}
   $\ds \int \frac{\ln x}{x^2}~dx$
  \end{Question}

  \begin{Solution}
    Let $u = \ln x$, $dv = \frac{1}{x^2}~dx$. \\
Then $du = \frac{1}{x}~dx$ and $v = -\frac{1}{x}$. 

 Integrating by parts, we get:
 \begin{align*}
\int \frac{\ln x}{x^2} ~dx 
& = -\frac{1}{x} \ln x - \int \left(-\frac{1}{x}\right) \frac{1}{x}~dx \\ 
& = -\frac{1}{x} \ln x + \int \frac{1}{x^2}~dx  \\
& = -\frac{1}{x} \ln x -  \frac{1}{x} + C  \\
 \end{align*}
  \end{Solution}
%*****************
%*****************
\item
  \begin{Question}
   $\ds \int \frac{y}{\sqrt{5-y}}~dy$
  \end{Question}

  \begin{Solution}
    Let $u = y$ and $dv = \frac{1}{ \sqrt{5-y}}$, \\
so $du = dy$ and $v = -2(5 - y)^{1/2}$
\begin{align*}
 \int y \sqrt{5 - y} ~dy 
& = -2y(5 - y)^{1/2} - \int (-2)(5 - y)^{1/2} dy \\
& = -2y(5 - y)^{1/2} + 2\int (5 - y)^{1/2} dy \\
& = -2y(5 - y)^{1/2} + 2\frac{2}{3} (5 - y)^{3/2}(-1) + C \\
& = -2y(5 - y)^{1/2} - \frac{4}{3} (5 - y)^{3/2}+ C \\
\end{align*}
  \end{Solution}
%*****************
%*****************
\item
  \begin{Question}
   $\ds \int \frac{t+7}{\sqrt{5-t}}~dt$
  \end{Question}

  \begin{Solution}
Since we have a fraction in the numerator, we can split the integral into a sum, and then
evaluate each term separately.
   $$\ds \int \frac{t+7}{\sqrt{5-t}}~dt = \underbrace{\int \frac{t}{\sqrt{5-t}}~dt}_{I_1} + \underbrace{\int \frac{7}{\sqrt{5-t}}~dt}_{I_2}  $$
$I_1$ can be evaluated using integration by parts. \\
Let $u = t$ and $dv = \frac{1}{\sqrt{5-t}}~dt$, \\
so $du = dx$ and $v = -2(5-t)^{1/2}$.
\begin{align*}
\int \frac{t}{\sqrt{5 - t}} ~dt  \\
&= -2t(5 - t)^{1/2} + 2 \int (5 - t)^{1/2} ~dt  \\
&= -2t(5 - t)^{1/2} - \frac{4}{3}(5 - t)^{3/2} + C.
\end{align*}
$I_2$ can be integrated directly:
\begin{align*}
  \int \frac{7}{\sqrt{5-t}}~dt 
& = 7 (2) (5-t)^{1/2}(-1) + C_1 \\ 
& = -14 (5-t)^{1/2} + C_1
\end{align*}
Adding the two integrals back together, we obtain
\begin{align*}
 \int \frac{t+7}{\sqrt{5-t}}~dt 
& = \ub{-2t(5 - t)^{1/2} - \frac{4}{3}(5 - t)^{3/2} + C}_{I_1} \\
& + \ub{-14 (5-t)^{1/2} + C_1}_{I_2}
\end{align*}
  \end{Solution}
%*****************
\item
  \begin{Question}
   $\ds \int x (\ln x)^2~dx$
  \end{Question}

  \begin{Solution}
Select $u = (\ln x)^2$ and $dv = x~dx$, \\
so $\ds du = 2 \ln x \left(\frac{1}{x}\right) ~dx$ and $v = x^2/2$. \\
Using the integration by parts formula, 
\begin{align*}
  \int x (\ln x)^2~dx 
& = (\ln x)^2\left(\frac{x^2}{2} \right)
- \int \left(\frac{x^2}{2}\right) \left(\frac{2 \ln x}{x}\right)~dx  \\
& = \frac{1}{2} x^2 (\ln x)^2 - \ub{\int x \ln x~dx}_{I_2}
\end{align*}
This second integral, $I_2$, can be evaluated with a second
application of integration by parts. This
was done earlier in Question \#\ref{q:xlnx} (using $y$ instead of $x$ though):  \\
\begin{align*}
 \ub{\int x \ln x~dx}_{I_2}
& = \frac{x^2 \ln x}{2} - \frac{x^2}{4}  + C  \\
\end{align*}
Going back to the original integral
\begin{align*}
 & \int x (\ln x)^2~dx \\
& = \frac{1}{2} x^2 (\ln x)^2 - 
\ub{\left(\frac{x^2 \ln x}{2} - \frac{x^2}{4}\right)}_{I_2} + C \\
& =\frac{1}{2} x^2 (\ln x)^2 - \frac{1}{2} x^2 \ln x + \frac{x^2}{4}  + C
\end{align*}
  \end{Solution}
%*****************
%*****************
\item \label{q:arcsinx}
  \begin{Question}
   $\ds \int \arcsin(w)~dw$
  \end{Question}

  \begin{Solution}
   We don't know the integral of $\arcsin$, but we do know its derivative. Therefore we pick  \\
$u = \arcsin(w)$ and $dv = dw$, \\
so $\ds du = \frac{1}{\sqrt{1 - w^2}} ~dw$ and $v = w$. \\
Using the integration by parts formula,

\begin{align*}
\int \arcsin(w)~dw & = w~\arcsin(w) - \ub{\int \frac{w}{\sqrt{1 - w^2}}~dw}_{I_2}
\end{align*}
The new integral $I_2$ can be evaluate using a substitution. \\
Let $z = 1 - w^2$, so $\frac{dz}{dw} = -2w$ or $\ds \frac{-1}{2w} ~dz = dw$:
\begin{align*}
\ub{\int \frac{w}{\sqrt{1 - w^2}}~dw}_{I_2} & = \int \frac{w}{\sqrt{z}} \left(\frac{-1}{2w}~dz\right) \\
& = \frac{-1}{2} \frac{1}{\sqrt{z}} ~dz \\
& =  \frac{-1}{2} (2z^{1/2}) +C \\
& = -\sqrt{1 - w^2} + C  
  \end{align*}
Going back to the original integral,
\begin{align*}
\int \arcsin(w)~dw & = w~\arcsin(w) - \ub{\int \frac{w}{\sqrt{1 - w^2}}~dw}_{I_2} \\
& =w~\arcsin(w) - \left(-\sqrt{1 - w^2} + C\right)   \\
& =w~\arcsin(w) + \sqrt{1 - w^2} + C_2
\end{align*}
  \end{Solution}
%*****************
%*****************
\item \label{q:arctan}
  \begin{Question}
   $\ds \int \arctan(7x) ~dx$
  \end{Question}

  \begin{Solution}
   We don't know the integral of $\arctan$, but we do know its derivative. Therefore we pick  \\
$u = \arctan(7x)$ and $dv = dx$, \\
so $\ds du = \frac{7}{1 + (7x)^2} ~dx$ and $v = x$. \\
Using the integration by parts formula,
\begin{align*}
\int \arctan(7x)~dx & = x~\arctan(7x) - \ub{\int \frac{7x}{1 - 49x^2}~dx}_{I_2}
\end{align*}
The new integral $I_2$ can be evaluate using a substitution. \\
Let $w = 1 + 49x^2$, so $\frac{dw}{dx} = 98 x$ or $\ds \frac{1}{98x} ~dw = dx$:
\begin{align*}
\ub{\int \frac{7x}{1 + 49x^2}~dx}_{I_2} & = \int \frac{7x}{w} \left(\frac{1}{98x}~dw\right) \\
& = \frac{1}{14} \frac{1}{w}~dw \\
& = \frac{1}{14} \ln|w| + C \\
& = \frac{1}{14} \ln|1 + 49x^2| + C
  \end{align*}
Going back to the original integral,
\begin{align*}
\int \arctan(x)~dx & = x~\arctan(x) - \ub{\int \frac{7x}{1 - 49x^2}~dx}_{I_2} \\
& =x~\arctan(x) - \frac{1}{14} \ln|1 + 49x^2| + C_2
\end{align*}
    
  \end{Solution}
%*****************
%*****************
\item
  \begin{Question}
   $\ds \int x \arctan(x^2)~dx$
  \end{Question}

  \begin{Solution}
This question starts off best with a substitution, due to the $x^2$ inside the $\arctan$, and the $x$ outside: \\
$  \mbox{ Let } w = x^2 \mbox{, so} \frac{1}{2x} ~dw = dx $ \\
\begin{align*}
\int x \arctan(x^2)~dx & = \int x \arctan(w) \left(\frac{1}{2x}~dw\right) \\
& = \frac{1}{2} \ub{\int \arctan(w)~dw   }_{I_2}
\end{align*}
Evaluating $I_2$, we use by parts, following the same approach as Question \#\ref{q:arctan}, but without the `7' factor, 
\begin{align*}
\ub{\int \arctan(w)~dw   }_{I_2} & =w~\arctan(w) - \frac{1}{2} \ln|1 + w^2| + C
\end{align*}
Going back to the original integral, and using $w = x^2$, 
\begin{align*}
& \int x \arctan(x^2)~dx \\
& = \frac{1}{2} \ub{\int \arctan(w)~dw   }_{I_2} \\
& = \frac{1}{2} \left(w~\arctan(w) - \frac{1}{2} \ln|1 + w^2| + C\right) \\
& = \frac{1}{2} \left(x^2~\arctan(x^2) - \frac{1}{2} \ln|1 + (x^2)^2| + C\right)\\
& = \frac{1}{2} x^2~\arctan(x^2) - \frac{1}{4} \ln|1 + x^4| + C_2
\end{align*}
  \end{Solution}
%*****************
%*****************
\item\label{q:parts1_end}
  \begin{Question}
   $\ds \int x^3 e^{x^2}~dx$
  \end{Question}

  \begin{Solution}
    In this problem, we note that we can't integrate $e^{x^2}$ by
    itself (no closed-form anti-derivative).  However, if we package
    it with one of the $x$'s from the $x^3$, we'll get $x e^{x^2}$,
    and that {\em can} be integrated, using substitution ($w = x^2$).

    Let $u = x^2$ and $dv = xe^{x^2}$, \\
so $du = 2x ~dx$ and $v = \frac{1}{2} e^{x^2}$.  Using that, 
\begin{align*}
\int x^3e^{x^2}~dx = \frac{1}{2} x^2e^{x^2} - \ub{\int xe^{x^2}~dx}_{I_2} &
= \frac{1}{2} x^2e^{x^2} - \frac{1}{2} e^{x^2} + C
\end{align*}
where $I_2$ is evaluated using the same substitution $w = x^2$.
  \end{Solution}
%*****************
%*****************
%\item 
%  \begin{Question}
%   $\ds \int x^3 \cos(x^3)~dx$
%  \end{Question}
%
%  \begin{Solution}
%    
%  \end{Solution}
%*****************
\item
  \begin{Question}
$\ds \int_1^5 \ln t~dt$    
  \end{Question}

  \begin{Solution}
We integrate by parts, as done in Question \#\ref{q:lnx}:
    \begin{align*}
    \int_1^5 \ln t~dt 
& = \ub{t \ln t - t }_{\mbox{ from \#\ref{q:lnx}}} \Big|_1^5 \\
& = \left(5 \ln 5 - 5\right)   - \left(1 \ln 1 - 1\right) \\
& = 5 \ln 5 - 4.
    \end{align*}

  \end{Solution}

%*****************
\item
  \begin{Question}
$\ds \int_3^5 x \cos x~dx$
  \end{Question}

  \begin{Solution}
   Integrating by parts with $u = x$ and $dv = \cos x ~dx$ gives 
   \begin{align*}
\int_3^5 x \cos x~dx & = x \sin(x) + \cos(x) \Big|_3^5 \\
& = 5 \sin(5) + cos(5) - (3 \sin(3) + \cos(3)) \\
& \approx -3.944
   \end{align*}
  \end{Solution}

%*****************
\item
  \begin{Question}
$\ds \int_0^{10} ze^{-z}~dz$
  \end{Question}

  \begin{Solution}
   The integral is the same as in Question \#\ref{q:zexpz}. We use
by parts with $u = z$ and $dv = e^{-z}~dz$, giving
\begin{align*}
  \int_0^{10} ze^{-z}~dz & = \ub{-ze^{-z} - e^{-z}}_{\mbox{from \#\ref{q:zexpz}}} \Big|_0^{10} \\
& = -10 e^{-10} - e^{-10} - (0 - e^0) \\
& = -11 e^{-10} + 1  \approx 0.9995 \\
\end{align*}
  \end{Solution}

%*****************
\item
  \begin{Question}
$\ds \int_1^3 t\ln(t)~dt$
  \end{Question}

  \begin{Solution}
    The integral is the same as in Question \#\ref{q:xlnx}.  We use by
    parts, with $u = \ln(t)$ and $dv = t ~dt$.
    \begin{align*}
      \int_1^3 t\ln(t)~dt & = \ub{ \frac{t^2 \ln t}{2} 
- \frac{t^2}{4}}_{\mbox{from \#\ref{q:xlnx}}} \Big|_1^3 \\
& = \left(\frac{9 \ln 3}{2} - \frac{9}{4}\right)
- \left(\frac{1 \ln 1}{2} - \frac{1}{4}\right) \\
& = \frac{9 \ln 3}{2} - 2 \approx 2.944\\
    \end{align*}
    
  \end{Solution}

%*****************
\item
  \begin{Question}
$\ds \int_0^1\arctan(y)~dy$
  \end{Question}

  \begin{Solution}
    We follow the work from Question \#\ref{q:arctan}, but without the `7' factor, using 
$u = \arctan(y)$ and $dv = dy$.
\begin{align*}
  \int_0^1\arctan(y)~dy  
& =y~\arctan(y) - \frac{1}{2} \ln|1 + y^2| \Big|_0^1  \\
& = 
\left(1\cdot \arctan(1) - \frac{1}{2} \ln(2)\right)
- \left(0\cdot \arctan(0) - \frac{1}{2} \ln(1)\right) \\
& = \frac{\pi}{4} -  \frac{1}{2} \ln(2) \approx 0.439.
\end{align*}
  \end{Solution}

%*****************
\item
  \begin{Question}
$\ds \int_0^5\ln(1+t)~dt$
  \end{Question}

  \begin{Solution}
    We use the solution to Question \#\ref{q:lnx}, or applying by
    parts with $u = \ln(1+t)$ and $dv = ~dt$.
    \begin{align*}
      \int_0^5\ln(1+t)~dt 
      & = \ub{(1+t) \ln (1+t) - (1+t) }_{\mbox{from \#\ref{q:lnx}, with } x = 1+t} \Big|_0^5 \\
& =
 \left(6 \cdot \ln 6 - 6\right) 
 -\left(1 \cdot \ln 1 - 1\right)  \\
& = 6 \ln 6 - 5 \approx 5.751
    \end{align*}

    
  \end{Solution}

%*****************
\item
  \begin{Question}
$\ds \int_0^1\arcsin z ~dz$
  \end{Question}

  \begin{Solution}
Using the solution from Question \#\ref{q:arcsinx}, or $u = \arcsin z$ and $dv = dz$, 
\begin{align*}
  \int_0^1\arcsin z ~dz 
& =z~\arcsin(z) + \sqrt{1 - z^2} \Big|_0^1  \\
& = 
\left( \arcsin(1)  + \sqrt{0}\right) 
- \left( 0 \cdot \arcsin(0)  + \sqrt{1}\right)  \\
& = \frac{\pi }{2} - 1 \approx 0.571
\end{align*}
    
  \end{Solution}

% *****************
\item
  \begin{Question}
$\ds \int_0^1 x \arcsin(x^2)~dx$
  \end{Question}

  \begin{Solution}
We first simplify the integral with the substitution $w = x^2$, which leads
to the new limits \\
 $x = 0 \to w = 0^2 = 0$ and \\
 $x = 1 \to w = 1^2 = 1$.
\begin{align*}
  \int_{x=0}^{x=1} x \arcsin(x^2)~dx & =
\ub{ \frac{1}{2} \int_{w=0}^{w=1} \arcsin(w)~dw  }_{\mbox{after substitution}}  \\
\end{align*}
At this point, we have returned to the integral in Question
\#\ref{q:arcsinx}, which can be evaluated using by parts, with $u =
\arcsin(w)$ and $dv = dw$.
\begin{align*}
&  \int_{x=0}^{x=1} x \arcsin(x^2)~dx \\
& = \frac{1}{2} \int_{w=0}^{w=1} \arcsin(w)~dw  \\
& = \frac{1}{2} \ub{\left(w~\arcsin(w) + \sqrt{1 - w^2}\right)}_{\mbox{by parts}} \Big|_0^1  \\ 
& = \frac{1}{2} \left[
\left( \arcsin(1)  + \sqrt{0}\right) 
- \left( 0 \cdot \arcsin(0)  + \sqrt{1}\right) 
\right]  \\
& = \frac{\pi}{4} - \frac{1}{2} \approx 0.285
\end{align*}
  \end{Solution}

% *****************
\item
  \begin{Question}
Find the area under the curve $y = t e^{-t}$ on the interval $0 \le t \le 2$.
  \end{Question}

  \begin{Solution}
The function $t e^{-t}$ is always positive on the interval $0 \le t \le 2$ so
the area under the curve is equal to the integral
$$\int_0^2 t e^{-t}~dt$$
   Proceeding in the same way as Question \#\ref{q:zexpz}, using $u = t$ and $dv = e^{-t}~dt$,
   \begin{align*}
   \int_0^2 t e^{-t}~dt 
& = \ub{-te^{-t} - e^{-t} }_{\mbox{from \#\ref{q:zexpz}}} \Big|_0^2\\ 
& = 
\left(-2e^{-2} - e^{-2}\right) 
- \left(0e^{0} - e^{0}\right)  \\
& = -3 e^{-2} + 1 
   \end{align*}
  \end{Solution}

% *****************
\item
  \begin{Question}
Find the area under the curve $f(z) = \arctan z$ on the interval $[0, 2]$.
  \end{Question}

  \begin{Solution}
    On the interval $t \in [0, 2]$, the function $\arctan(z)$ is
    always positive, so the area equals the integral $\ds \int_0^2 \arctan(z)~dz$.

    To evaluate the integral, we follow the work from Question
    \#\ref{q:arctan}, but without the `7' factor, using $u =
    \arctan(z)$ and $dv = dz$.
\begin{align*}
  \int_0^2 \arctan(z)~dz & = z~\arctan(z) - \frac{1}{2} \ln|1 + z^2|  \Big|_0^2\\
& = 
\left(2 \arctan(2) - \frac{1}{2} \ln 5\right)
- \left(0 \arctan(0) - \frac{1}{2} \ln 1\right) \\
& = 2 \arctan(2) - \frac{\ln 5}{2} 
\end{align*}
  \end{Solution}

% *****************
%\item
%  \begin{Question}
%    Find the area under the curve $f(y) = \arcsin y$ for $0 \le y \le 1$. 
%  \end{Question}
%
%  \begin{Solution}
%    
%  \end{Solution}


% *****************
\item
  \begin{Question}
    Use integration by parts twice to find $\ds \int  e^x \sin(x)~dx$.
  \end{Question}

  \begin{Solution}
There are several ways to evaluate this integral; we'll show just one here. \\
Let $u = \sin(x)$ and $dv = e^x ~dx$, \\
so $du = \cos(x)~dx$ and $v = e^x$. 
\begin{align*}
  \ub{\int  e^x \sin(x)~dx}_{\mbox{ our goal}, ~I} & =
\sin(x) e^x - \ub{\int \cos(x) e^x~dx}_{I_2} \\
\end{align*}
To evaluate $I_2$, we select the trig function again as $u$ and the exponential as $dv$:
Let $u = \cos(x)$ and $dv = e^x ~dx$, \\
so $du = -\sin(x)~dx$ and $v = e^x$. 
\begin{align*}
&   \ub{\int  e^x \sin(x)~dx}_{\mbox{ our goal}, ~I}  \\
& = \sin(x) e^x - \ub{\int \cos(x) e^x~dx}_{I_2} \\
& = \sin(x) e^x - \left( \cos(x) e^x - \int (-\sin(x)) e^x~dx\right) \\
& = \sin(x) e^x - \left( \cos(x) e^x - \int (-\sin(x)) e^x~dx\right) \\
\end{align*}
Tidying, we obtain
\begin{align*}
  \ub{\int  e^x \sin(x)~dx}_{~I}  
& = \sin(x) e^x - \cos(x) e^x - \ub{\int \sin(x) e^x~dx }_I
\end{align*}
Grouping the integrals $I$, which are what we are looking for,
\begin{align*}
  2 \int e^x \sin(x)~dx & = 
 \sin(x) e^x - \cos(x) e^x  \\
\mbox{ or } 
  \int e^x \sin(x)~dx & = 
 \frac{1}{2} \left(\sin(x) e^x - \cos(x) e^x \right) \\
\end{align*}

  \end{Solution}
% *****************
\item
  \begin{Question}
Use integration by parts twice to find $\ds \int  e^y \cos(y)~dy$.
  \end{Question}

  \begin{Solution}
    This question is done in the same manner as the previous one.  For
    variety, and to show it works as well, we will select the
    exponential function as $u$ and the trig functions as $dv$. (Both
    choices work, so long as you are consistent in both integration by
    parts steps.)
    
Let $u = e^y$ and $dv = \cos(y) ~dy$, \\
so $du = e^y~dy$ and $v = \sin(y) $. 
\begin{align*}
  \ub{\int  e^y \cos(y)~dy}_{\mbox{ our goal}, ~I} & =
\sin(y) e^y - \ub{\int \sin(y) e^y~dy}_{I_2} \\
\end{align*}
To evaluate $I_2$, we select the exponential function again as $u$ and the trig as $dv$: \\
Let $u = e^y$ and $dv = \sin(y) ~dy$, \\
so $du = e^y~dy$ and $v = -\cos(y)$. 
\begin{align*}
&   \ub{\int  e^y \cos(y)~dy}_{\mbox{ our goal}, ~I}  \\
& = \sin(y) e^y - \ub{\int \sin(y) e^y~dy}_{I_2} \\
& = \sin(y) e^y - \left( (-\cos(y)) e^y - \int (-\cos(y)) e^y~dy\right) \\
& = \sin(y) e^y + \cos(y) e^y - \int \cos(y) e^y~dy
\end{align*}
Tidying, we obtain
\begin{align*}
  \ub{\int  e^y \cos(y)~dy}_{~I}  
& = \sin(y) e^y + \cos(y) e^y - \ub{\int \cos(y) e^y~dy }_I
\end{align*}
Grouping the integrals $I$, which are what we are looking for,
\begin{align*}
  2 \int e^y \cos(y)~dy & = 
 \sin(y) e^y + \cos(y) e^y  \\
\mbox{ or } 
  \int e^y \cos(y)~dy & = 
 \frac{1}{2} \left(\sin(y) e^y + \cos(y) e^y \right) \\
\end{align*}
  \end{Solution}

% *****************
%\item
%  \begin{Question}
%Use integration by parts twice to find $\ds \int  \sin^2x~dx$.
%  \end{Question}
%
%  \begin{Solution}
%    
%  \end{Solution}

% *****************
\item
  \begin{Question}
    Use integration by parts to show that $$\int x^n \cos(ax)~dx =
    \frac{1}{a} x^n \sin(ax) - \frac{n}{a} \int x^{n-1} \sin(ax) ~dx$$
  \end{Question}

  \begin{Solution}
We are simply asked to change one integral into another, which can be done here 
directly with integration by parts. \\
Let $u = x^n$ and $dv = \cos(ax)~dx$, \\
so $du = n x^{n-1} ~dx$ and $\ds v = \frac{1}{a} \sin(ax)$  \\
Applying the by parts formula,
\begin{align*}
&   x^n \cos(ax)~dx \\
& =
x^n \left(\frac{1}{a} \sin(ax)\right) 
- \int \frac{1}{a} \sin(ax) \cdot n \cdot x^{n-1} ~dx \\
& = x^n \left(\frac{1}{a} \sin(ax)\right) 
-\frac{n}{a} \int x^{n-1} \sin(ax) ~dx
\end{align*}
which is the desired formula.
  \end{Solution}
    

% *****************
\item
  \begin{Question}

    The concentration, $C$, in ng/ml, of a drug in the blood as a
    function of the time, $t$, in hours since the drug was administered
    is given by $$C = 15te^{ - 0.2t}.$$ The area under the concentration
    curve is a measure of the overall effect of the drug on the body,
    called the {\em bioavailability}. Find the bioavailability of the drug
    between $t = 0$ and $t = 3$.
  \end{Question}

  \begin{Solution}
    We have
$$Bioavailability = \int^3_0 15te^{-0.2t}~dt.$$
We first use integration by parts to evaluate the indefinite integral of this function. \\
Let $u = 15t$ and $dv = e^{-0.2t}~dt$,\\
so $du = 15~dt$ and $v = -5e^{-0.2t}$. Then,
\begin{align*}
 \int 15te^{-0.2t}~dt & = (15t)(-5e^{-0.2t}) - \int (-5e^{-0.2t})(15~dt) \\
& = -75te^{-0.2t} + 75 \int e^{-0.2t}~dt \\
& = -75te^{-0.2t}- 375e^{-0.2t} + C.
\end{align*}
  
\begin{align*}
\mbox{Thus, } \int^3_0 15te^{-0.2t}~dt & = (-75te^{-0.2t} - 375e^{-0.2t}) \Big|^3_0 \\
& = -329.29 + 375 = 45.71. \\
\end{align*}
The bioavailability of the drug over this time interval is 45.71 (ng/ml)-hours
  \end{Solution}

%*****************
\item
  \begin{Question}
    During a surge in the demand for electricity, the rate, $r$, at
    which energy is used can be approximated by $$r = te^{ -at}$$  where $t$
    is the time in hours and $a$ is a positive constant. 
    \begin{enumerate}[(a)]
    \item Find the total energy, $E$, used in the first $T$ hours. Give
      your answer as a function of $a$.
    \item What happens to $E$ as $T \to \infty$?
    \end{enumerate}
  \end{Question}

  \begin{Solution}
   We know that $\ddt{E} = r$, so the total energy $E$ used in the first $T$ hours is given by 
$$E = \int^T_0 te^{-at} dt.$$ 
We use integration by parts. \\
Let $u = t$, $dv = e^{-at}~dt$. \\
Then $du  = dt$, and $v = (-1/a) e^{-at}$.
\begin{align*}
E & = \int^T_0 te^{-at} dt  \\
& = - (t/a) e^{-at} \Big|^T_0 - \int^T_0 -(1/a) e^{-at} ~dt \\
& = - (1/a) Te^{-aT} + (1/a) \int^T_0 e^{-at} dt \\
& = - (1/a) Te^{-aT} + (1/a^2) (1 - e^{-aT} ).
\end{align*}
  
(b)
 \begin{align*}
 \lim_{T \to \infty}  E & = -(1/a) \lim_{T\to\infty}  \left( \frac{T }{e^{aT}} \right) + (1/a^2) \left( 1 - \lim_{T \to\infty} \frac{1}{ e^{aT}} \right) 
 \end{align*}
 Since $a > 0$, the second limit on the right hand side in the above
 expression is 0. In the first limit, although both the numerator and
 the denominator go to infinity, the denominator $e^{aT}$ goes to
 infinity more quickly than $T$ does (can verify with l'Hopital's
 rule). So in the end the denominator $e^{aT}$ is much greater than
 the numerator $T$. Hence
$ \ds \lim_{T\to \infty} \frac{T}{ e^{aT}} = 0$. 

Thus $\lim_{T\to\infty} E = \frac{1}{a^2}$.  

In words this means that the total amount of energy in the surge,
accounting for over all time $(T \to \infty)$ is $\ds \frac{1}{a^2}$
Joules.
 \end{Solution}
%*****************

% \item
%   \begin{Question}
%     In describing the behavior of an electron, we use wave functions
%     $\Psi_1$, $\Psi_2$, $\Psi_3$, $\ldots$\footnote{The symbol $\Psi$ is the capital greek letter ``Psi'', pronoused the same way as `sigh'.  It is the most common symbol for probability functions and wave functions in chemistry and physics.} of the form $$\Psi_n( x) =
%     C_n \sin( n\pi x) \mbox{~~~ for }n = 1, 2, 3, \ldots$$ where $x$
%     is the distance from a fixed point and $C_n$ is a positive
%     constant.
% \begin{enumerate}[(a)]
% \item  Find $C_1$ so that $\Psi_1$ satisfies 
% $$\int^1_0 ( \Psi_1( x))^2~dx = 1.$$ This is called {\em normalizing} the wave function $\Psi_1$. 
% \item  For any integer $n$, find $C_n$ so that $Psi_n$ is normalized.
% \end{enumerate}
%   \end{Question}

%   \begin{Solution}
    
%   \end{Solution}
\end{multicols}

\hrulefill

\end{enumerate}
\end{document}

