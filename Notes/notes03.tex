\input{PreambleCommon}
\newcommand{\WeekTitleOne}{Derivatives - Foundations}
\newcommand{\WeekTitleTwo}{Derivatives - Linearization and Applications}
\newcommand{\WeekTitleThree}{Derivatives - Applications}
\newcommand{\WeekTitleFour}{Integrals - Foundations}
\newcommand{\WeekTitleFive}{Integrals - Techniques}
\newcommand{\WeekTitleSix}{Integrals - Modeling}
\newcommand{\WeekTitleSeven}{Differential Equations - }
\newcommand{\WeekTitleEight}{Differential Equations - }
\newcommand{\WeekTitleNine}{Differential Equations - }
\newcommand{\WeekTitleTen}{Linear Algebra - }
\newcommand{\WeekTitleEleven}{Linear Algebra - }
\newcommand{\WeekTitleTwelve}{Linear Algebra - }



\begin{document}
\setfont
\pagestyle{fancy}
\renewcommand{\Week}{3 }
\renewcommand{\WeekTitle}{\WeekTitleThree }

\fancyhead[LE,RO]{Week \Week}  % default, usually only for first page
\fancyfoot{}
\sectionbox{Week \#\Week: \WeekTitle}


\vspace{5mm}
\goals
\begin{itemize}
\item Calculate and interpret the first and second derivatives, as
  well as higher order derivatives.
\item Define and calculate Taylor Polynomials.
\item Use MATLAB to graph and compare functions with their Taylor
  polynomial approximations.
\item Find and use critical points for global and local optimization
  problems.
\item Use MATLAB optimizers and equation solvers to identify optimal
  values and critical points.
\end{itemize}

\vspace{5mm}


\newpage
\topic{Second and Higher Derivatives}
\subsection*{Second and Higher Derivatives}


The information about the graph of a function $f$ provided by the sign
of $f'(x)$ and $f''(x)$ on an interval $(a,b)$ is expressed in the
following table. ($a$ and $b$ are assumed to be finite.)

{\huge
\vsc
\begin{center}
\begin{tabular}{|l|l|} \hline
\qquad \qquad \qquad \qquad \qquad  \qquad \qquad & \qquad  \qquad \qquad \qquad \qquad  \qquad \qquad \\
\qquad $f'(x) > 0 {\mbox{ on }} (a,b)$  &  \qquad $f$ increasing on $[a,b]$  \\
\qquad \qquad \qquad \qquad \qquad & \qquad \qquad \qquad  \qquad \qquad \\
 \hline
\qquad \qquad \qquad \qquad \qquad & \qquad \qquad \qquad  \qquad \qquad \\
\qquad $f'(x) < 0 {\mbox{ on }} (a,b)$  &  \qquad $f$ decreasing on $[a,b]$  \\
\qquad \qquad \qquad \qquad \qquad & \qquad \qquad \qquad  \qquad \qquad \\
 \hline
\qquad \qquad \qquad \qquad \qquad & \qquad \qquad \qquad  \qquad \qquad \\
\qquad $f''(x) > 0 {\mbox{ on }} (a,b)$  &  \qquad $f$ concave up on $[a,b]$  \\ \qquad \qquad \qquad \qquad \qquad & \qquad \qquad \qquad  \qquad \qquad \\
\hline
\qquad \qquad \qquad \qquad \qquad & \qquad \qquad \qquad  \qquad \qquad \\
\qquad $f''(x) < 0 {\mbox{ on }} (a,b)$  &  \qquad $f$ concave down on $[a,b]$ \qquad  \qquad  \\ 
\qquad \qquad \qquad \qquad \qquad & \qquad \qquad \qquad  \qquad \qquad \\
\hline
\end{tabular}
\end{center}
}
\setfont

\newpage 

\problem Sketch the possible graphs combining different signs of
positive and negative first and second derivatives.

\vfill
\vfill


\problem Sketch graphs where the {\bf first} derivative has a zero value. 

\vfill \problem Sketch graphs where the {\bf second} derivative has a
zero value.

\vfill

\newpage

Aside from their graphical interpretation, second derivatives
frequently have important physical interpretations in kinematics
problems.

\problem If $x(t) = 4 \sin(2t)$ gives the position of a particle at
time $t$, what is particle's {\bf speed} at $\ds t=\frac{\pi}{6}$?

\vfill

For the same particle, what is its {\bf acceleration} at
$\ds t=\frac{\pi}{6}$?

\vfill

\newpage



While their interpretations are not as immediately obvious, it is
possible to compute 3rd, 4th, or higher derivatives of function if we
want.

\problem Find the first four derivatives of the function
$$f(x) = 7 (2^x) + \ln(x).$$


\newpage
\topic{Taylor Polynomials}
\subsection*{Taylor Polynomials}

One application of higher derivative information is to help us build
{\bf polynomial approximations} to complicated functions.

Previously we found a formula for linear approximations to functions
$f(x)$ around a point $x=a$:

\vspace{1.5in}

This linear approximation, or tangent line formula, can also be called
the {\bf Taylor polynomial of degree 1 approximating $f(x)$ near
  $x=a$.}

\newpage

\problem Sketch the graph of $\cos(x)$ around $x=0$, and add its
tangent line based at $x=0$.

\includegraphics[width=3in]{graphics/empty_graph_square_12}


The linearization or tangent line is clearly a very
  limited approximation to this function.  What might be a {\em
    slightly} more complex form of function that would work better in
  this case?

\vfill

\newpage

\begin{boxnote}

{\bf Taylor Polynomial of Degree 2}
\vspace{1in}

$$ f(x) \approx ~~~~~f(a) ~~~~~+ f'(a) (x-a) ~~~~~+ \frac{f''(a)}{2} (x-a)^2$$  

\vspace{1in}

is a {\em quadratic} approximation to $f(x)$ near $x=a$.
\end{boxnote}


\problem For values of $x$ close to $a$ do you think this quadratic
approximation will be a better or worse approximation than the tangent
line?  Why?

\vfill

\newpage
\topic{Taylor Polynomials - Examples}

\problem Find the quadratic Taylor approximation to \\$f(x) = \cos(x)$
  near $x=0$.

\vfill

\newpage

\problem Use MATLAB to draw the graph of $\cos(x)$ around $x=0$, and
add both its 1st and 2nd degree Taylor polynomial approximations for
$x$ near 0.


\newpage

There is a very good reason for the particular form of the Taylor
polynomial.

\problem What mathematical features will the original $f(x)$ share
with its 2nd degree Taylor approximation at the point $x=a$?

\vfill

\newpage
\topic{Taylor Polynomials of Higher Degree}
\subsection{Taylor Polynomials of Higher Degree}

\problem If we wanted a still-better approximation for a function
$f(x)$ near a specific point $x=a$, how could we generalize our
earlier 1st and 2nd degree Taylor polynomials?

\vfill

\newpage 

Below is the general formula for the terms in a Taylor polynomial, up
to degree $n$.  \\

$$ f(x) \approx ~~f(a) ~~+ f'(a) (x-a) ~~+ \frac{f''(a)}{2} (x-a)^2 + \ldots 
+ \frac{f^{(n)}(a)}{n!} (x-a)^n$$   \\

\begin{itemize}
\item $f^{(n)}$ means ``the $n$-th derivative of $f$''. \\
\item $n!$ means ``$n$ factorial''
\end{itemize}

\newpage
\subsection*{Higher Degree Taylor Polynomials - Example 1}

Consider the function $f(x) = \sin(x)$.

\problem Find the first five derivatives of $f(x)$, and evaluate them at $x=0$.


\newpage

\problem Write out the Taylor poylnomial of degree 5 for
$f(x) = \sin(x)$.

\vfill

\problem Write out the general form of the Taylor poylnomial of degree
$n$ for $f(x) = \sin(x)$.

\vfill

\newpage

\problem Use MATLAB to plot the graph of $f(x) = \sin(x)$ and the
Taylor polynomial approximations up to degree 5.

\newpage

MATLAB Demo of increasing higher degrees.


\subsection*{Higher Degree Taylor Polynomials - Example 2}

Consider the function $g(x) = x e^{-x}$.

\problem Find the first three derivatives of $g(x)$, and evaluate them
at $x=1$.


\newpage

\problem Write out the Taylor poylnomial of degree 3 for
$g(x) = x e^{-x}$ centered at $x=1$.

\vfill

\newpage

\problem Use MATLAB to plot the graph of $g(x) = x e^{-x}$ and the
Taylor polynomial approximation from degree 1, 2 and 3.

\newpage
\topic{Critical Points}

\subsection*{Critical Points} 

Aside from understanding the shape of functions, derivative
information can help us identify and classify interesting points of a
function, like the highest and lowest values.

\problem Sketch graphs which have high and low points.  

\vfill 

What do those extreme values have in common?


\newpage

\begin{boxnote}

  If $f(x)$ is defined on the interval $(a,b)$, then we call a point
  $c$ in the interval a {\bf critical point} if:
\begin{itemize}
	\item $f'(c) = 0$, or
	\item $f'(c)$ does not exist.
\end{itemize}
We will also refer to the point $(c,f(c))$ on the graph of $f(x)$ as a
critical point.  We call the function value $f(c)$ at a critical point
$c$ a {\bf critical value}.

\vsc
\end{boxnote}

\newpage

Technical Notes:
\begin{enumerate} 
\item By this definition, $f(c)$ must be {\bf defined} for
  $c$ to be a critical point.  

\problem   Sketch $f(x) = 1/x$, and decide whether $x=0$ is a
    critical point.

\vfill
  Sketch $g(x) = |x|$, and decide whether $x=0$ is a
    critical point.

\vfill

\newpage

\item By the definition, if a function is defined on a closed
  interval, the endpoints of interval {\bf cannot} be critical points.

\problem   Sketch the graph of $f(x) = \sqrt{x}$ and decide
    whether $x=0$ is a critical point.

\vfill

\newpage

\problem Sketch the graph of $g(x) = \sqrt[3]{x}$ and decide
  whether $x=0$ is a critical point.

\vfill

\end{enumerate}

\newpage

% \problem Identify all the critical points on the graph below, and
%   characterize any other interesting points by continuity, limits, or
%   other properties.

\problem Identify all the critical points on the graph below.

\includegraphics[width=8in]{graphics/notes03_crit_pnt_oddities}

\newpage

\topic{Classifying Critical Points}

\section*{Classifying Critical Points}

We will now formalize two ways to determine if a critical point is a
local min, max, or neither. This avoids the need for a sketch of the
graph.

\begin{boxnote}
{\bf First Derivative Test}

One way to decide whether at a critical point there is a local maximum
or minimum is to examine the sign of the derivative on opposite sides
of the critical point.  This method is called the {\bf{first
    derivative test}}.  {\em{Complete this table:}}

%\def\arraystretch{1.5}
\vspace{1mm}
\begin{center}
\begin{tabular}{|l|c|c|p{1in}|} \hline
\qquad \qquad \qquad \qquad \qquad &  $f'$ sign left of $c$ & $f'$ sign right of $c$ & Sketch\hspace{0.5in}  \\ \hline
local minimum at $c$ & & &\\[1.0in] \hline
local maximum at $c$ & & &\\[1.0in] \hline
neither local max nor min & & &\\[1.0in] \hline
\end{tabular}
\end{center}

%\def\arraystretch{1}
\vspace{1.5mm}

\end{boxnote}

\newpage

\problem Find the critical points of the function $f(x) = 2 x^3 - 9
  x^2 + 12 x +3$.  Use the first derivative test to show whether each
  critical point is a local maximum or a local minimum.

\vfill
\vfill

\newpage 

Using your answer to the preceding question, determine
  the number of real solutions of the equation $2 x^3 - 9 x^2 + 12 x +
  3 = 0$.

\vfill

\newpage 
\begin{boxnote}
{\bf Second Derivative Test}

You may also use the Second Derivative Test to determine if a critical
point is a local minimum or maximum.  
\begin{itemize}
\item The first derivative test uses the {\bf first} derivative {\bf
    around} the critical point.
\item The second derivative test uses the {\bf second} derivative {\bf at} the critical point.
\end{itemize}
\begin{itemize}
\item If $f'(c)$ = 0 and $f''(c) > 0$ then $f$ has a local minimum at $c$. \notesonly{\\[1.0in]}\largeall{\\[0.8in]}
\item If $f'(c)$ = 0 and $f''(c) < 0$ then $f$ has a local maximum at $c$. \notesonly{\\[1.0in]}\largeall{\\[0.8in]}
\item If $f'(c)$ = 0 and $f''(c) = 0$ then the test is inconclusive. \\[0.3in] 
\end{itemize}
\end{boxnote}

\vsc
\newpage
\problem A function $f$ has derivative $f'(x) = \cos(x^2) + 2x - 1$.
  Does it have a local maximum, a local minimum, or neither at its
  critical point $x=0$?

\newpage 

\topic{Global vs. Local Optimization}
\section*{Global vs. Local Optimization}

\reading{Section 4.2}

The first and second derivative tests only give us {\em local}
information in most cases.  However, if there are multiple local
maxima or minima, we usually want the {\bf global} max or min.  The
ease of determining when we have found the global max or min of a
function depends strongly on the properties of the question.

\begin{boxnote}

  \subsection*{Local vs Global Extrema}

  A {\bf local max} occurs at $x=c$ when $f(c) > f(x)$ for $x$ values
  near $c$.

  A {\bf global max} occurs at $x=c$ if $f(c) \ge f(x)$ for {\bf all}
  values of $x$ in the domain.  It is possible to have several global
  maxima if the function reaches its peak value at more than one
  point.

  Corresponding definitions apply for local and global minima.

\end{boxnote}

\newpage

\problem Give an example of a simple function with multiple global
  maxima.

\vfill

\problem Give an example of a simple function with a single global
  maximum, but no global minimum.

\vfill

\problem Give an example of a simple function with neither a global
  maximum nor a global minimum.

\vfill


\newpage \problem Earlier we worked with the function $f(x) = 2 x^3 -
  9 x^2 + 12 x +3$.  If we limit the function to the interval $x \in
  [0, 2.5]$, what are the {\bf global max} and {\bf global minimum}
  values on that interval?  



\vfill
\vfill

\newpage

\topic{Global Extrema on Closed and Open Intervals}

\begin{boxnote}
\subsection*{Global Extrema on Closed Intervals}

A continuous function on a closed interval will {\bf always} have a
global max and a global min value.  These values will occur at either
\begin{itemize}
\item a critical point {\em or}
\item an end point of the interval.
\end{itemize}
To find which value is the global extrema, you can compute the
original function's values at all the critical points and end points,
and select the point with the highest/lowest value of the function.

\end{boxnote}


\newpage
\begin{boxnote}
\subsection*{Global Extrema on Open Intervals}

A function defined on an open interval may or may not have global
maxima or minima.  
\vsc

If you are trying to demonstrate that a point is a global max or min,
and you are working with an open interval, including the possible
interval $(-\infty, \infty)$, proving that a particular point is a global
max or min requires a careful argument.  A recommendation is to look
at either:
\begin{itemize}
\item values of $f$ when $x$ approaches the endpoints of the interval,
  or $\pm \infty$, as appropriate; or
\item if there is only one critical point, look at the sign of $f'$ on
  either side of the critical point.
\end{itemize}
With that information, you can often construct an argument about a
particular point being a global max or min.

\end{boxnote}

\newpage

\problem Determine whether the function $f(x) = 2 x^3 - 9 x^2 + 12 x
  +3$ has a global max and/or min.

\vfill


\newpage
\problem Determine whether the function $f(x) = (x-2)^4$ has a global
  max/and or min.

\vfill

\newpage

\topic{Optimization - Fencing Example}
\section*{Optimization}

\reading{Section 4.4}
\vsc

An optimization problem is one in which we have to find the maximum or
minimum value of some quantity.  In principle, we already know how to
find the maximum and minimum values of a function if we are given a
formula for the function and the interval on which the maximum or
minimum is sought.  Usually the hard part in an optimization problem
is interpreting the word problem in order to find the formula of the
function to be optimized.

\vsc 

\newpage 

\problem A farmer wants to build a rectangular enclosure to contain
  livestock.  The farmer has 120 meters of wire fencing with which to
  build a fence, and one side of the enclosure will be part of the
  side an already existing building (so there is no need to put up
  fence on that side). What should the dimensions of each side be to
  maximize the area of the enclosure? 

\vsc


What is the quantity to be maximized in this example?

\vspace{2cm}

\newpage

What are the variables in this question, and how are they
  related?  You may want to draw a picture.

\vfill

Express the quantity to be optimized in terms of the
  variables.  Try to eliminate all but one of the variables.

  \vfill

  What is the domain on which the one remaining variable
    makes sense?

\vspace{2cm} 
  \newpage 

Use the techniques learned earlier in the course to
  maximized the function on this domain.  Give reasons explaining why
  the answer you found is the {\bf global} maximum.

\vspace{5cm}

\newpage

\hfill (continued)


\newpage

\topic{Optimization - Storage Example}
\problem Storage Container)\\ A rectangular storage container with an
  open top is to have a volume of 10 m$^3$.  The length of its base is
  to be twice its width.  Material for the base costs \$10.00 per
  m$^2$, and material for the sides costs \$6.00 per m$^2$.  Determine
  the cost of the material for the cheapest such container.
~ \hfill \includegraphics[width=1.5in]{graphics/week07_storage_container}

\newpage
~ \hfill \includegraphics[width=1.5in]{graphics/week07_storage_container}

\newpage
~ \hfill \includegraphics[width=1.5in]{graphics/week07_storage_container}

\newpage 

\topic{Optimization - Fisher Example}
{\bf Example} (Taken from 2004 Dec Exam)

A fisher is in a boat at point A, which is 2 km from the nearest
point on the shoreline.  He is to go to a lighthouse at point B, which
is 3 km down the coast (see figure below).

\hfill \includegraphics[height=\notesonly{3in}\largeall{4in}]{graphics/notes08_graph01}
\newpage If the fisher can row at 4 km per hour, and walk
  at 5 km per hour, find an expression for $T(x)$, the travel time if
  the fisher lands the boat $x$ km down the shore from the nearest
  approach.

\hfill \includegraphics[height=\notesonly{3in}\largeall{4in}]{graphics/notes08_graph01}

\newpage 

If the fisher can row at 4 km per hour, and walk at 5 km
  per hour, how far from the point B should he land the boat to
  minimize the time it takes to get to the lighthouse?  Make sure to
  indicate how you know your answer is the global minimum.

\hfill \includegraphics[height=3in]{graphics/notes08_graph01}
\vsc

\newpage

\hfill \includegraphics[height=3in]{graphics/notes08_graph01}

\newpage

\topic{Optimization - Fisher General Case}

Often, the numerical values in an optimization problem are somewhat
arbitrary, or estimated using best guesses.  It is often more
important to discover the response in the solution to a {\em range} of
possible problem values.  In that vein, we now suppose the fisher
has a motor that will drive his boat at a speed of $v$ km per hour.

If the fisher's walking speed is still 5 km per hour,
  for what values of $v$ will it be fastest to simply drive the boat
  directly to the lighthouse (i.e. do no walking)?

\hfill \includegraphics[height=3in]{graphics/notes08_graph01}

\newpage

\hfill \includegraphics[height=3in]{graphics/notes08_graph01}



\end{document}

