\documentclass[12pt]{amsart}
\documentclass[12pt,twoside]{article}

% *** Set page dimensions ***
\raggedbottom
\parindent=0in
%\setlength{\topmargin}{-0.5in}
%\setlength{\oddsidemargin}{0.1875in}
%\setlength{\evensidemargin}{0in}
%\setlength{\textheight}{8.5in}
%\setlength{\textwidth}{6.225in}
%\addtolength{\oddsidemargin}{-0.7in}
%\addtolength{\evensidemargin}{-1.2in}
%\setlength{\oddsidemargin}{-0.2in}
%\setlength{\evensidemargin}{-0.2in}
%\addtolength{\textwidth}{1.4in}
%\addtolength{\topmargin}{-.875in}
%\addtolength{\textheight}{2.00in}

% *** Packages ***
\usepackage{alltt}
\usepackage{tocloft}
\usepackage{graphicx}
\usepackage{lscape}
\usepackage{amssymb}
\usepackage{float}
\usepackage{amsmath}
\usepackage{gensymb}
%\usepackage{subfigure}
\usepackage{lscape}
\usepackage{epsfig}
\usepackage{enumerate}
\usepackage{multicol}
\usepackage{fancyhdr}
\usepackage{epstopdf}
\usepackage{hyperref}
\usepackage{listings}

% *** Table of contents and Sectioning *** 
\setcounter{secnumdepth}{0}
\setcounter{tocdepth}{5}

% *** Table of contents and Sectioning *** 
\newcommand{\next}{\addtocounter{enumi}{9} \item}
\newcommand{\now}[1]{\setcounter{enumi}{#1}}
\newcommand{\Z}{\mbox{\sf Z\hspace{-1.5mm}Z}}
\newcommand{\R}{\mbox{\rm I\hspace{-0.75mm}R}}
\columnsep=0.75in

% *** Shortcuts for syntax ***
\newcommand{\ds}{\displaystyle }
\newcommand{\vsc}{\vspace{4mm}}
\newcommand{\dd}[1]{\frac{d}{d{#1}} \,} 
\newcommand{\ddx}{\frac{d}{dx} \,} 
\newcommand{\ddy}{\frac{d}{dy} \,} 
\newcommand{\ddz}{\frac{d}{dz} \,} 
\newcommand{\dydx}{\frac{dy}{dx} \,} 
\newcommand{\dydt}{\frac{dy}{dt} \,} 
\newcommand{\dfdx}{\frac{df}{dx} \,} 
\newcommand{\ddt}[1]{  \frac{d{#1}}{dt} }
\newcommand{\pp}[2]{  \frac{\partial{#1}}{\partial {#2}} }
\newcommand{\zx}{\frac{\partial z}{\partial x} \,}
\newcommand{\zy}{\frac{\partial z}{\partial y} \,}
\newcommand{\limh}{\lim_{h \rightarrow 0} \;}
\newcommand{\diff}{\frac{d}{dx} \,}
\newcommand{\de}{\Delta}
\renewcommand{\thesection}{\Roman{section}}
\newcommand{\bfr}{\begin{flushright}}
\newcommand{\efr}{\end{flushright}}
\newcommand{\dx}{\frac{\partial f}{\partial x} \,}
\newcommand{\dy}{\frac{\partial f}{\partial y} \,}
\newcommand{\p}{\partial}
\newcommand{\vi}{\vec{i}}
\newcommand{\vj}{\vec{j}}
\newcommand{\vk}{\vec{k}}
\newcommand{\lan}{\left\langle}
\newcommand{\ran}{\right\rangle}
\newcommand{\reading}[1] { {\em Reading: #1}}
\renewcommand{\Pr}{ \mbox{Pr}}

% *** Commands related to textbook references
\newcommand{\problem}{{\bf Problem.} }

% *** Footnoting with symbols ***
\long\def\symbolfootnote[#1]#2{\begingroup%
\def\thefootnote{\fnsymbol{footnote}}\footnote[#1]{#2}\endgroup}

% *** Defining a boxed note ***
\floatstyle{boxed}
\newfloat{noteinbox}{htb}{loa}
\newenvironment{boxnote}{\begin{noteinbox}[H]}{\end{noteinbox}}

\newcommand{\Question}{ {\bf Question: }  }
\newcommand{\Example}[1]{ {\bf Example: } {\em #1} }
\newcommand{\ExampleCont}[1]{ {\em #1} }

% *** Define the boxed Week #/summary at the beginning/end of every chapter ***
\newcommand{\sectionbox}[1]{% 
\begin{tabular}{|p{6in}|}%
\hline%
\ \\ %
{\Large {\bf {#1}}}  \\%
\ \\%
\hline%
\end{tabular}}

% *** Shortcuts *** 
\newcommand\goals{\large {\bf {Goals:}}}
\newcommand\setfont{ }

% *** Week commands: overwritten in each notes file
\newcommand{\Week}{Null-InPreambleCommon}
\newcommand{\WeekTitle}{Null-InPreambleCommon}
\newcommand{\Course}{MNTC P04}
\newcommand{\SetNum}{1 }
\newcommand{\topic}[1]{
\newpage
\setcounter{page}{1}
\fancyhead[LE,RO]{#1 - \thepage}
}

% *** Setup Latex for the large version of the files ***
%\usepackage[landscape]{geometry}
\usepackage[letterpaper,landscape,hmargin={.8in,.8in},vmargin={1in,0.2in}]{geometry}

% Remove paragraph indents
\setlength{\parindent}{0pt}

% Spacing at the top for the header is too large by default
\setlength{\voffset}{-5ex}

% **** RENEW SCALING COMMANDS HERE ****
% *** Text in boxes ***
\renewenvironment{boxnote}{\begin{noteinbox}[H] \huge}{\end{noteinbox}} 

% *** Chapter lead in/summary boxes ***
\renewcommand{\sectionbox}[1]{% 
\begin{tabular}{|p{9.5in}|}%
\hline%
\ \\ %
{\huge {\bf {#1}}}  \\%
\ \\%
\hline%
\end{tabular}}

% *** 'Section'' commands, which are sometimes used for spacing
% From http://zoonek.free.fr/LaTeX/LaTeX_samples_section/0.html
\makeatletter
 \renewcommand\section{\@startsection {section}{1}{\z@}%
                                    {-3.5ex \@plus -1ex \@minus -.2ex}%
                                    {0.3ex \@plus.2ex}%
                                    {\setfont\bf}}

 \renewcommand\subsection{\@startsection {subsection}{1}{\z@}%
                                    {-3.5ex \@plus -1ex \@minus -.2ex}%
                                    {0.3ex \@plus.2ex}%
                                    {\setfont\bf}}

% *** 'Goals' should be larger in the overheads ***
\renewcommand\goals{\huge {\bf {Goals:}}}
\renewcommand\setfont{\huge }

\thispagestyle{empty}

\setfont 

\newcommand{\WeekTitleOne}{Derivatives - Foundations}
\newcommand{\WeekTitleTwo}{Derivatives - Linearization and Applications}
\newcommand{\WeekTitleThree}{Derivatives - Modeling}
\newcommand{\WeekTitleFour}{Integrals - Foundations}
\newcommand{\WeekTitleFive}{Integrals - Techniques}
\newcommand{\WeekTitleSix}{Integrals - Modeling}
\newcommand{\WeekTitleSeven}{Differential Equations - }
\newcommand{\WeekTitleEight}{Differential Equations - }
\newcommand{\WeekTitleNine}{Differential Equations - }
\newcommand{\WeekTitleTen}{Linear Algebra - }
\newcommand{\WeekTitleEleven}{Linear Algebra - }
\newcommand{\WeekTitleTwelve}{Linear Algebra - }


\begin{document}

\newpage
\setfont
\pagestyle{fancy}
\renewcommand{\Week}{10}
\renewcommand{\WeekTitle}{\WeekTitleTen}

\fancyhead[LE,RO]{Week \Week}  % default, usually only for first page
\fancyfoot{}
\sectionbox{Week \#\Week: \WeekTitle}

\vspace{5mm}
\goals 
\begin{itemize}
\item Express vectors, linear combinations, and compute dot products.
\item Write information in matrix form in the context of engineering applications.
\item Understand the definition of the transpose, and use MATLAB to compute it.
\item Use MATLAB to compute the inverse of a matrix.
\end{itemize}

\newpage
\topic{Linear Algebra - Introduction}
\subsection*{Linear Algebra}

As pointed out on several other occasions in the course, most laws
of nature and science, once they are translated into mathematics,
take the form of a {\bf differential equation}.

A differential equation is quite unlike the other equations we have
studied. For one thing, it involves derivatives, often even second
or higher derivatives.  But this is not the most important
difference between a differential equation and the kinds of
equations we have been used to and continue to study in our courses. 
\begin{itemize}
\item Solutions to {\bf algebraic} equations are {\bf real values} (or sets of values) \\[2ex]

\item Solutions to {\bf differential} equations are {\bf functions} (or sets of functions) \\[2ex]

\end{itemize}

\newpage
\begin{problem}
Which of the following {\bf functions} is a solution to the
{\bf differential} equation $x^{\prime\prime} (t) = - 36 ~x(t)$? \\[2ex]
\begin{enumerate}[A.]
\item $x(t)=-6 t^3$ \\[2ex]
\item $x(t)=\cos(6 t)$ \\[2ex]
\item $x(t)=e^{-6t}$ \\[2ex]
\item $x(t)=-e^{-6t}$
\end{enumerate}
\end{problem}
\bigskip

\newpage

\begin{problem}
Confirm your answer.
\end{problem}
$x^{\prime\prime} (t) = - 36 ~x(t)$ 
\hfill\begin{minipage}[t]{0.4\linewidth}
\begin{enumerate}[A.]
\item $x(t)=-6 t^3$ 
\item $x(t)=\cos(6 t)$ 
\item $x(t)=e^{-6t}$ 
\item $x(t)=-e^{-6t}$
\end{enumerate}
\end{minipage}

\newpage
\topic{Differential Equations - Harmonic Motion}
\subsection*{Modeling with Differential Equations}

\subsection*{Harmonic Motion} 
We will begin by studying the possible
solutions of a particular differential equation that will be very
important in your Physics course next term. Thinking about these
solutions will help us understand what we mean when we say a function
is a solution of a differential equation.  The differential equation
we saw in the previous concept question arises when we study the
motion of a block at the end of a spring, as in the next diagram:
\begin{center}
\includegraphics[width=0.5\linewidth]{graphics/w11_block_pstricks}
\end{center}

 In this system, how would you describe $x$ in words?


\newpage
\begin{problem}
  
 Draw a free-body diagram for the mass.  Indicate the magnitude of the forces, assuming 
\begin{itemize}
\item the mass of the block is $m$ kg, and
\item the spring constant (in $N/m$) is
  given by the constant $k$.
\end{itemize}
\end{problem}


\newpage
Let us work with our intuition about this system before beginning the mathematics.

\begin{problem}
 If the spring is very stiff, is $k$ large or small?
\end{problem}
\vfill

 If we exchange a soft spring for a stiff spring, do you
  expect the \emph{period} of the oscillations to increase or
  decrease? Why?
\vfill


\newpage
\begin{problem}
 If we replace a small mass with a heavier mass, do you expect
the \emph{period} of the oscillations to increase or decrease? Why?
\end{problem}

\vfill

 If we know $k$ and $m$, and assume that friction is
  negligible, should we be able to determine the exact period of the
  oscillations?
\vfill

 Does anyone know the formula for the period of oscillations of a spring system?
\vfill

\newpage

The spring system is an excellent introduction to differential equations because 
\begin{itemize}
\item we all know how it \emph{should} work physically, \\[1ex]
\item the mathematics and physics are simple, {\bf but} \\[1ex]
\item there's no obvious way to predict critical features (e.g. the
  period) from the given information. \\[1ex]
\end{itemize}
We clearly need some new tools!

\newpage
\topic{Analysis of Mass/Spring System}
\subsection*{Analysis of Mass/Spring System}
\begin{problem}
 Use Newton's second law, $F = ma$, to construct an
  equation involving the position $x(t)$.
\end{problem}
\vfill

 What are we solving for in this equation?  I.e., what is
  the unknown?
\vfill

\newpage
\begin{problem}
  
  To simplify matters temporarily, let us assume that both
  $k = 1 $ N/m and $m = 1$ kg. Rewrite the previous differential
  equation.
\end{problem}
\vfill

 This differential equation invites us to find a function
$x(t)$ whose second derivative is its own negative.  Try to think of
such a function.
\vfill

\newpage

We have found a set of solutions for the differential equation
$$\displaystyle\frac{d^2x}{dt^2}= - x.$$ Does this help us discover
solutions for the differential equation 
$$m\frac{d^2x}{dt^2}= - kx?$$  

\begin{problem}
  
 Find one or more solutions to this second differential equation.
\end{problem}
\vfill

\newpage
~\hfill$m\frac{d^2x}{dt^2}= - kx$

\newpage
\begin{problem}
  Find {\bf the most general family} of solutions to this
  differential equation.
\end{problem}


% \includesoln{\solntest}{}{%%%%%%%%%%%%%%%%%%%%%%%%%%%%%%%%%%%%%%
% After some experimentation, you will notice that if we let
% $\omega=\sqrt{k/m}$, then $x(t)= \cos(\omega t)$ and
% $x(t)=\sin(\omega t)$ work, and then that any function of the form
% $x(t)= B\cos(\omega t) + C\sin(\omega t)$ is a solution to this
% differential equation.
% }%%%%%%%%%%%%%%%%%%%%%%%%%%%%%%%%%%%%%%%%%%%%%%%%%%%%%%%%%%%%%%%%%%%%%%%%%%%
\newpage
\begin{problem}
 If we define $\ds \omega = \sqrt{\frac{k}{m}}$, can we simplify our solutions?
\end{problem}
\vfill

 Why is the variable $\omega$ (usually associated with circular motion) an appropriate name for this constant?
\vfill

\newpage
\topic{Two Formulas for Oscillations}
\subsection*{Two Formulas for Oscillations}

From our solution $x(t) = A \cos(\omega t) + B \sin(\omega t)$, we get
the sense that the position of the mass attached to a spring will
oscillate at frequency $\omega$ rad/s. But imagining what the sum of a
sine and cosine curve together looks like is not easy: it is a
simple-looking curve or not?  
\\ 
Fortunately, there is another way to
write the solution that is much more easily sketched.

\newpage
A less obvious solution to the DE
$$\frac{d^2x}{dt^2}= - \omega^2\, x $$
is 
$$x(t)= C\cos(\omega t - \phi)$$

Through the use of the trig identity 
$$\cos(x-y) = \cos(x)\cos(y) + \sin(x)\sin(y)$$ 
we can derive the following relations:

\begin{boxnote}
$$A \cos(\omega t) + B \sin(\omega t) = C \cos(\omega t - \phi) $$
$$\text{where} \quad C = \sqrt{A^2 + B^2} \quad \text{and} \quad \tan(\phi) = \frac{B}{A}$$
\end{boxnote}

\newpage
$$A \cos(\omega t) + B \sin(\omega t) = C \cos(\omega t - \phi) $$
$$\text{where} \quad C = \sqrt{A^2 + B^2} \quad \text{and} \quad \tan(\phi) = \frac{B}{A}$$

\begin{problem}
 Prove the equivalency of the two forms, and draw a
  triangle that represents the relationship between all the
  constants.
\end{problem}

\newpage
\hfill
\begin{minipage}{0.7\linewidth}
$$A \cos(\omega t) + B \sin(\omega t) = C \cos(\omega t - \phi) $$
$$\text{where} \quad C = \sqrt{A^2 + B^2} \quad \text{and} \quad \tan(\phi) = \frac{B}{A}$$
\end{minipage}

%  Confirm that this is in fact a solution to the spring
%   DE.

%  Are the functions of this form actually different than our earlier
% $x(t) = A \cos(\omega t) + B \sin(\omega t)$ solutions?

% No: 

% In fact it does not, for we will
% now show that these two only look like different functions, and
% that, using appropriate trigonometric identities, we can show that
% each $B\cos(\omega t) + C\sin(\omega t)$ is identically equal to
% $A\cos(\omega t + \phi)$ for suitable $A$ and $\phi$, and
% conversely.

% First of all, you should note that allowing $A$ to be negative in
% the expression $x(t)= A\cos(\omega t + \phi)$ amounts to unnecessary duplication, for if
% we add $\pi$ to $\phi$ we change the sign of the expression:
% $$ A\cos(\omega t + \phi +\pi) = -A\cos(\omega t + \phi)$$
% The reason for this is that $\cos(x+\pi)=-\cos(x)$ for any $x$.
% Thus, we may as well assume that $A$ is chosen to be positive.

% Now there is a trigonometric identity for the cosine of the sum of two angles; it is proved in Appendix D in the textbook if you have not seen it before: $$\cos(a+b) = \cos a \cos b - \sin a \sin b\,
% . $$ If we apply this with $a=\omega t$ and with $b=\phi$, we obtain
% $$A\cos(\omega t + \phi)= A\left[ \cos(\omega t)\cos(\phi)-\sin(\omega
% t)\sin(\phi)\right]$$
% $$ = [A\cos(\phi)]\cos(\omega t) + [-A\sin(\phi)]\sin(\omega
% t)\, . $$ Notice that $t$ is the only variable in this expression;
% the other letters represent constants.  The question before us is:
% Can $B$ and $C$ be chosen so that this is exactly the same as
% $B\cos(\omega t) + C\sin(\omega t)$ {\bf for all $t$}?  Clearly we can,
% by simply letting %
% \[\begin{array}{lclcl}
% B & = & A\cos(\phi) & = & A\cos(-\phi) \\
% C & = & -A\sin(\phi) & = & A\sin(-\phi)\\
% \end{array}\]

% Conversely, we could ask whether we can always solve for $A$ and
% $\phi$ in terms of $B$ and $C$.  We are not interested in knowing an
% actual formula so much as knowing that it is possible, so that we
% can say with confidence that the two families of solutions we found
% are really one and the same - that for each expression in one family
% there is a corresponding one (defining the same function) in the
% other. To see that this can always be done, suppose a number $\phi$
% and a (positive) number $A$ are given. The we can imagine a line
% segment of length $A$ emanating from the origin and making the angle
% $-\phi$ with the horizontal axis:

% \includesoln{\solntest}{
% \begin{center}
% \begin{pspicture}(-3,-2.5)(3,2.5)
% \psdot(0,0)\psline(0,0)(3,-2)
% \psline[linestyle=dashed]{->}(-3,0)(3,0)
% \psline[linestyle=dashed]{->}(0,-2.5)(0,2.5)\rput(.85,-.25){$-\phi$}
% \psdot(3,-2) \rput(1.4,-1.2){$A$} \psline{->}(1.2,-1.1)(-.1,-.2)
% \psline{->}(1.6,-1.3)(2.8,-2.1)
% \end{pspicture}
% \end{center}
% }{%%%%%%%%%%%%%%%%%%%%%%%%%%%%%%%%%%%%%%
% \begin{center}
% \begin{pspicture}(-3,-2.5)(3,2.5)
% \psdot(0,0)\psline(0,0)(3,-2)
% \psline[linestyle=dashed]{->}(-3,0)(3,0)
% \psline[linestyle=dashed]{->}(0,-2.5)(0,2.5)\rput(.85,-.25){$-\phi$}
% \psdot(3,-2) \rput(1.4,-1.2){$A$} \psline{->}(1.2,-1.1)(-.1,-.2)
% \psline{->}(1.6,-1.3)(2.8,-2.1) \rput(3.7,-2){$(B,C)$}
% \end{pspicture}
% \end{center}
% }%%%%%%%%%%%%%%%%%%%%%%%%%%%%%%%%%%%%%%%%%%%%%%%%%%%%%%%%%%%%%%%%%%%%%%%%%%%

% \ques{What are the coordinates of the endpoint of the line segment?}
% \includesoln{\solntest}{\vspace{2cm}}{%%%%%%%%%%%%%%%%%%%%%%%%%%%%%%%%%%%%%%
% If we drop a perpendicular from the endpoint to the horizontal axis,
% or think of the segment as a vector from the origin to its endpoint,
% it is easy to see that the coordinates of the endpoint are
% $(A\cos(-\phi),A\sin(-\phi))=(A\cos(\phi),-A\sin(\phi))$.

% }%%%%%%%%%%%%%%%%%%%%%%%%%%%%%%%%%%%%%%%%%%%%%%%%%%%%%%%%%%%%%%%%%%%%%%%%%%%

%  This shows that for any $A$ and any
% $\phi$ the associated numbers $B$ and $C$ are just the coordinates
% of the endpoint of the line segment we drew; and, conversely, if $B$
% and $C$ are given, then we let $A$ be the length of the segment from
% the origin to the point $(B,C)$ and $\phi$ the angle it makes with
% the horizontal axis.  Thus we have shown geometrically that we can
% always solve for $B$ and $C$ if we know $A$ and $\phi$, and
% conversely.


% The upshot of this discussion is that\\

%  \thmbox{The family of functions
% $$B\sin(\omega t) + C\cos(\omega t)\, ,$$
% where $B$ and $C$ are arbitrary constants, and the family of
% functions
% $$A\cos(\omega t + \phi)\, ,$$
% where $\phi$ is an arbitrary constant and $A$ an arbitrary positive
% constant, are really the same family.  Each member of this family of
% functions is a solution to the differential equation
% $$ x^{\prime\prime} (t) = - \omega^2 x(t)$$
% arising from Hooke's Law.}

% By another, rather long, argument it can be shown that there are no
% solutions arising from Hooke's Law, other than the ones in this
% family of functions. We will simply assume this.


% \paragraph{What do these solutions look like?}  We should now describe the functions
% in this family of
% solution, for they tell us how the block will move under the
% influence of the spring.

% To do this it is easiest to work with the expression
% $$A\cos(\omega t + \phi)\, .$$  We begin by observing that these
% functions are all periodic:

% \conques{$x(t) = A\cos(\omega t + \phi)$ is periodic with period
% \begin{enumerate}
% \item[A. ] $2\pi$
% \item[B. ] $\omega$
% \item[C. ] $\phi/\omega$
% \item[D. ] $2\pi/\omega$
% \end{enumerate}
% }

% \includesoln{\solntest}{\vspace{6cm}}{%%%%%%%%%%%%%%%%%%%%%%%%%%%%%%%%%%%%%%%%%%%
% The key question is: what fixed number $a$ is such that $x(t+a) =
% x(t)$ for all $t$?  That is, we want $\cos(\omega (t+a) + \phi) =
% \cos(\omega t + \phi)$.  That is, we want
% $$\cos(\omega t + \phi + \omega a ) = \cos(\omega t + \phi)\, .$$
% We know that the cosine function is periodic with period $2\pi$, so
% we know that
% $$\cos(\omega t + \phi + 2\pi ) = \cos(\omega t + \phi)\, .$$
% Thus we want $2\pi = \omega a$.  That is, we want $a = 2\pi/\omega$.
% In other words, D is the correct answer.
% }%%%%%%%%%%%%%%%%%%%%%%%%%%%%%%%%%%%%%%%%%%%%%%%%%%%%%%%%%%%%%%%%%%%%


\newpage
\topic{Graphs of Shifted Cosines}
\subsection*{Graphs of $C \cos(\omega t - \phi)$ }

We have now shown that the functions $x(t) = A \cos(\omega t) + B \sin(\omega t)$
and $x(t) = C \cos(\omega t - \phi)$ are identical.  What do they look like?
\vfill
Stage 1: $y = \cos(t)$
\begin{center}
\includegraphics[width=0.8\linewidth]{graphics/week11_cos1}
\end{center}
\vfill

Stage 2: $y = \cos(t - \phi)$
\vfill
\begin{center}
\includegraphics[width=0.8\linewidth]{graphics/week11_cos2}
\end{center}

\newpage 

Stage 3: $y = \cos(\omega t - \phi)$
\begin{center}
\includegraphics[width=0.8\linewidth]{graphics/week11_cos3}
\end{center}
\vfill

Stage 4: $y = C \cos(\omega t - \phi)$
\begin{center}
\includegraphics[width=0.8\linewidth]{graphics/week11_cos4}
\end{center}
\vfill

\newpage
The upshot of this discussion is that $C\cos(\omega t - \phi)$ 
\begin{itemize} 
\item is a``sinusoidal'' function (that is, it is shaped like a sine or
cosine function); \\[1ex]
\item has {\bf amplitude} $C$; \\[1ex]
\item is periodic with {\bf period} $2\pi/\omega$; \\[1ex]
\item has its first peak at $\ds t = \frac{\phi}{\omega}$  (or when $(\omega t - \phi) = 0$). \\[1ex]
\end{itemize}
The motion described by the function $C\cos(\omega t - \phi)$ is known
as {\bf simple harmonic motion}.

\newpage
\begin{problem}
  Recall that in an application of spring motion
  $\displaystyle\omega = \sqrt{\frac{k}{m}}$, where $k$ is the
  stiffness of the spring and $m$ is the mass of the sliding block.
  Suppose the motion of the block is started by releasing the block
  from a non-equilibrium position.  Which of the following will change
  the {\bf time it takes} for the block to go through {\bf one cycle} of its
  periodic motion? \\[1ex]
\begin{enumerate}[A.]
\item The block is given a small push in the direction away from the equilibrium position. \\[1ex]
\item The block is released closer to the equilibrium position. \\[1ex]
\item A weight is attached to the top of the block.
\end{enumerate}

\end{problem}


\newpage
\begin{problem}
For a block attached to a spring (with spring constant $k$) on a
frictionless table, the potential energy is 0 when the block is in
the equilibrium position. What is the potential energy when the
extension of the spring is $x$?

\end{problem}

\newpage
 \begin{problem}
If an object undergoes harmonic motion, so that its
position function is a solution of the differential equation
$$ x^{\prime\prime} (t) = - \frac{k}{m} x(t)\, ,$$ prove using the
differential equation that
$$ m(x^\prime(t))^2 + k(x(t))^2 = \mbox{a constant}$$
\end{problem}

\newpage
$$ m(x^\prime(t))^2 + k(x(t))^2 = \mbox{a constant}$$
\begin{problem}
What does this equation tell you physically?
\end{problem}




\newpage

\topic{First-Order Differential Equations}
\subsection*{First-Order Differential Equations}
We have studied the differential equation for harmonic motion and
found its solutions.  This differential equation 
$$ m x'' = - k x$$
was a {\bf second order} differential equation, by which we mean that
it involved {\bf second derivatives} of the unknown function. 

Our methods for finding solutions were ad hoc: there is no indication
how the technique we used to find these solutions might be used to
solve other differential equations.  In second year, you will have a
full course on differential equations where you will learn a more
systematic approach for solving second order DEs.

In this class, we will limit ourselves to methods for solving {\bf
  first order} DEs (equations which involves first derivatives only).

\newpage
 Consider the form
$$\frac{dy}{dx} = f(x,y) $$
where $f(x,y)$ indicates an expression involving $x$ and $y$ but no
derivatives. 

\begin{problem}
What would we be trying to solve for in this equation?
\end{problem}
\vfill

\begin{problem}
 If we picked an arbitrary $(x,y)$ point on the graph of a
  solution, $y(x)$, what graphical interpretation could we make about
  the shape of the graph of $y(x)$ there?
\end{problem}
\vfill

\newpage

% This type of equation has a very nice visualization.
% Think of the unknown function $y$ as measured along the dependent
% axis, and $x$ as independent variable in the usual way.  We are
% looking for a solution function $y(x)$.  The differential equation
% tells you that when $x$ has a given value and $y$ has a particular
% value, then the slope of the function you are trying to find is
% equal to the value of $f(x,y)$.  For example, suppose the
% differential equation is
% and that a solution $y(x)$ of this equation has the value 3 when
% $x=2$, then the slope of the graph of $y(x)$ at $x=2$ is 6.  This
% observation allows us to think of finding a solution to this kind of
% differential equation in terms of a picture.

\begin{problem}
On the following grid, draw the slopes for the differential equation
equation 
 $$\frac{dy}{dx} = xy$$
 at a number of points.  The result is know as the {\bf slope field}
 of the differential equation. 


\hfill \includegraphics[width=0.4\linewidth]{graphics/w11_axes_pstricks}

Once you have the slope field, use your picture to 
sketch the graphs
of some {\bf solutions} to this differential equation.
\end{problem}

\newpage
Here is a computer-generated version of that slope field:
\begin{center}
\includegraphics[width=7in]{graphics/w11_slopefield}
\end{center}
\begin{problem}
Sketch the graphs
of some {\bf solutions} to this differential equation.
\end{problem}

\newpage
\topic{Constant Solutions}
\subsection*{Constant Solutions}
$$ y' = xy$$
\begin{problem}
There was one solution that was perfectly horizontal.  Identify this {\bf constant solution}
based only on the differential equation. 
\end{problem}

\vfill

\begin{problem}
Show that there are no other possible constant solutions to $y' = xy$. 
\end{problem}

\vfill
\newpage
\topic{Separable Differential Equations}
\subsection*{Separable Differential Equations}
A {\bf separable differential equation} is a differential equation
in which the expression for the derivative can be isolated and
written as a product of two functions, one involving only the
independent variable and the other involving only the dependent
variable.  That is, it is a separable differential equation if it
can be written in the form
$$ \frac{dy}{dx} = g(x) f(y)\, .$$

\newpage
\begin{problem}
Which of the following are separable?
\begin{center}\begin{tabular}{rlrl}
1. & $\ds \frac{du}{dt} = u\cos(t)$ &
\qquad \qquad 2. & $\ds x^\prime (t) = k t$ \\[2.5ex]
3. & $\ds 4\frac{dy}{dx} = xy$  &
4. & $\ds T^\prime (t) = T(t)-20$ \\[1ex]
\end{tabular}\end{center}
~\\[1ex]
\begin{enumerate}[A.]
\item All of them. \\[1ex]
\item Number 3 only. \\[1ex]
\item Numbers 1 and 3.
\end{enumerate}

\end{problem}

\newpage

Give examples of other DEs which are {\bf not} separable. 

\newpage
\topic{Separation of Variables - Example 1}
\subsection*{Separation of Variables - Example 1}
\begin{problem}
Use separation of variables to find a solution to the differential equation $y' = xy$.
\end{problem}


\newpage
\hfill $y' = xy$.

\newpage

\begin{problem}
  Sketch the solutions you found on the earlier slope field for $y' = xy$.
\begin{center}
\includegraphics[width=7in]{graphics/w11_slopefield}
\end{center}
\end{problem}

% For such values of $y$ it is convenient to define $h(y)
% = 1/f(y)$.  The differential equation then becomes
% $$\frac{dy}{dx} = \frac{g(x)}{h(y)}\, . $$
% The method for solving separable equations hinges on our ability to
% {\bf separate the variables} as follows:
% $$ h(y) dy = g(x) dx\, . $$
% Writing it this way, invites integrating both sides:
% $$ \int h(y) dy = \int g(x) dx\, . $$
% Before we go on, we should justify this formal pushing around of
% symbols.  Do we have any justification for believing that doing this
% will lead to a solution for the differential equation?  An
% indefinite integral is an anti-derivative (and always involves an
% arbitrary constant), so the last equation can be thought of as
% follows:  Suppose we have found anti-derivatives $H(y) =  \int h(y)
% dy $ and $G(x) = \int g(x)\,dx$ (each hiding an arbitrary constant
% of course).  Now put these equal to each other, $H(y)=G(x)$, and
% solve this for $y$ in terms of $x$.  Do we know that $y=y(x)$ will
% then be a solution to the differential equation?

% \ques{Use implicit differentiation on $H(y)=G(x)+C$ to show that $y$
% is a solution to the differential equation $$\frac{dy}{dx} =
% \frac{g(x)}{h(y)}\, . $$}
% \includesoln{\solntest}{\mbox{}\\[8cm]}{%%%%%%%%%%%%%%%%%%%%%%%%%%%%%%%%%%%%%%%%%%%%%%%%%
% Differentiating with respect to $x$ we get $$H^\prime (y)
% \frac{dy}{dx} = G^\prime (x)$$ and thus
% $$h(y)\frac{dy}{dx} = g(x)\, .$$
% That is,
% $$\frac{dy}{dx} = \frac{g(x)}{h(y)}\, . $$
% }%%%%%%%%%%%%%%%%%%%%%%%%%%%%%%%%%%%%%%%%%%%%%%%%%%%%%%%%%%%%%%%%%%%%%%%%%%

\newpage
\topic{Separation of Variables - Method}
The method for separation of variables can be summarized as follows:

\begin{boxnote}
{\bf Separable differential equations}
\begin{center}
\begin{minipage}{0.9\linewidth}
 Separate the
 variables to produce an equality of differential expressions
 $$ h(y) dy =  g(x) dx\, , $$
 Find anti-derivatives of both sides (thus introducing an arbitrary
 constant) giving $$H(y)= G(x)$$ and solve for $y$ in terms of $x$.
\end{minipage}
\end{center}
\end{boxnote}

\newpage
\topic{Separation of Variables - Example 2}
\begin{problem}
Find the solution to 
\[ \frac{dC}{dt} =  r - kC.
\]
 
\end{problem}

\newpage
\hfill
$\ds \frac{dC}{dt} =  r - kC$


\newpage

The differential equation in the previous example, $\ds \frac{dC}{dt} = r - kC$, shows up in many applications.
\begin{itemize}
\item This model can describe the concentration of a drug in a patient; if
it is injected at a rate of $r$ g/hour, and metabolized at a rate
proportional to the amount present, then the net rate of change is
$$\mbox{net rate} = \mbox{ rate in} - \mbox{rate out} = r - kC$$
\item The solutions to this DE are typical in that the ``+ D'' from
  integration appears, but {\bf not as an additive term} in the final
  solution.

\end{itemize}

\newpage
$$\ds \frac{dC}{dt} = r - kC$$
\begin{problem}
If $C$ represents a concentration, what will the concentration value converge to in the long run? 

\begin{enumerate}[A.]
\item $0$ \\[1ex]
\item $r$ \\[1ex]
\item $k$ \\[1ex]
\item $r-k$ \\[1ex]
\item $r/k$
\end{enumerate}
\end{problem}

\newpage
$$\ds \frac{dC}{dt} = r - kC$$
\begin{problem}
Support your answer by looking for the constant solution(s) to the DE.
\end{problem}


\end{document}
