\input{PreambleCommon}
\newcommand{\WeekTitleOne}{Derivatives - Foundations}
\newcommand{\WeekTitleTwo}{Derivatives - Linearization and Applications}
\newcommand{\WeekTitleThree}{Derivatives - Applications}
\newcommand{\WeekTitleFour}{Integrals - Foundations}
\newcommand{\WeekTitleFive}{Integrals - Techniques}
\newcommand{\WeekTitleSix}{Integrals - Modeling}
\newcommand{\WeekTitleSeven}{Differential Equations - }
\newcommand{\WeekTitleEight}{Differential Equations - }
\newcommand{\WeekTitleNine}{Differential Equations - }
\newcommand{\WeekTitleTen}{Linear Algebra - }
\newcommand{\WeekTitleEleven}{Linear Algebra - }
\newcommand{\WeekTitleTwelve}{Linear Algebra - }


\begin{document}

\newpage
\setfont
\pagestyle{fancy}
\renewcommand{\Week}{10}
\renewcommand{\WeekTitle}{\WeekTitleTen}

\fancyhead[LE,RO]{Week \Week}  % default, usually only for first page
\fancyfoot{}
\sectionbox{Week \#\Week: \WeekTitle}

\vspace{5mm}
\goals 
\begin{itemize}
\item Express vectors, linear combinations, and compute dot products.
\item Write information in matrix form in the context of engineering applications.
\item Understand the definition of the transpose, and use MATLAB to compute it.
\item Use MATLAB to compute the inverse of a matrix.
\end{itemize}

\newpage
\topic{Linear Algebra - Introduction}
\subsection*{Introduction to Vectors}
In a physical context we often need to describe measurable quantities such as pressure, mass, and speed, and these objects can be completely described by a single number known as the \textbf{magnitude}. However there are other quantities which help us describe the world around us such as force, velocity and acceleration that require more than magnitude to describe them. They also need a \textbf{direction}. A \textbf{vector} is a magnitude (a number describing how much, how fast, etc) in combination with a direction. Vectors are denoted by boldface letters, such as $\overrightarrow{u}, \overrightarrow{v}, \overrightarrow{x}, \overrightarrow{y}, \overrightarrow{z}$.

\newpage

When discussing direction, we need to decide on a frame of reference. We will be using the Cartesian coordinate system. A \textit{point} in the two-dimensional Cartesian plane, denoted as $\mathbb{R}^2$ is denoted by an ordered pair (x,y) of real numbers, which we call \textit{coordinates}. So for example, (2,1) can be represented as a point on a grid like so: 



\newpage

Consider the point (x,y) in $\mathbb{R}^2$. If we draw a directed line segment from the \textbf{origin} (the point (0,0)) to (x,y), we get the following picture:
``insert picture here"

Notice there is an arrow pointing at (x,y). We call this the \textbf{head} of the vector and the \textbf{tail} is at the origin, indicating a direction. The \textbf{magnitude} of the vector is its length from (0,0) to (x,y). 


\newpage

To find the magnitude of a vector, we first find its \textbf{components}. If the head of the vector is at the point (x', y') and the tail of the vector is at the point (x,y), then the components of the vector are $x'-x$ and $y'-y$, and we represent them in the following way:

$$ \begin{bmatrix}x'-x\\y'-y\end{bmatrix} $$

Once you have found the components, the formula for finding the magnitude of the vector is:
$$\text{magnitude} = \sqrt{(x'-x)^2 + (y'-y)^2} $$ 

\newpage

Draw and find the magnitude of the following vectors:$ \overrightarrow{a}$ has the tail at (-1,1) and the head at (4,3), $\overrightarrow{b}$ has the tail at (0,4) and the head at (-2,-2), and $\overrightarrow{c}$ has the tail at (3,0) and the head at (0,3). We use the notation $||\overrightarrow{a}||$ to represent the magnitude of $\overrightarrow{a}$.

``Insert picture here"

\newpage

An interesting fact to note is that two vectors are \textbf{equal} if their components are equal. Consider the vector $\overrightarrow{A}$ going from (-1,4) to \\ \noindent (-3,1) (tail to head), $\overrightarrow{B}$ going from (2,3) to (0,0) and $\overrightarrow{C}$ going from (5,5) to (3,2). All of these vectors occupy the different regions of $\mathbb{R}^2$, yet we can define them as equal since they have the same components. A \textbf{position vector} is a vector whose tail starts at the origin, and if a vector is given with just its components, you may assume it is a position vector.
\newpage
\subsection*{Vector Addition and Scalar Multiplication}


If we have two vectors in $\mathbb{R}^2$ with components
$$ \overrightarrow{x} = \begin{bmatrix}x_1\\x_2\end{bmatrix} \text{ and } \overrightarrow{y} = \begin{bmatrix}y_1\\y_2\end{bmatrix}$$

then the \textbf{sum} of the vectors \textbf{x} and \textbf{y} is

$$ \overrightarrow{x} + \overrightarrow{y} = \begin{bmatrix}x_1+y_1\\x_2+y_2\end{bmatrix} $$

\newpage

So if we have vectors 

$$ \overrightarrow{x} = \begin{bmatrix}1\\1\end{bmatrix}, \text{ } \overrightarrow{y} = \begin{bmatrix}-2\\5\end{bmatrix}, \text{ and } \overrightarrow{z} = \begin{bmatrix}4\\-1\end{bmatrix}$$

Then $$ \overrightarrow{x} + \overrightarrow{y} = \begin{bmatrix}1+(-2)\\1+5  \end{bmatrix} = \begin{bmatrix}
-1\\6
\end{bmatrix} $$

Or we could have

$$ \overrightarrow{y} + \overrightarrow{z} = \begin{bmatrix}-2+4\\5+(-1)  \end{bmatrix} = \begin{bmatrix}
2\\4
\end{bmatrix} $$

\newpage

We can interpret vector addition geometrically.

``Insert picture here''


\newpage

If $\overrightarrow{x} = \begin{bmatrix}
x_1\\x_2
\end{bmatrix}$ is a vector and $c$ is a scalar (meaning a real number), then the \textbf{scalar} multiple $c\overrightarrow{x}$, meaning every component of $\overrightarrow{x}$ is multiplied by $c$, is $c\overrightarrow{x} = \begin{bmatrix}
cx_1\\cx_2
\end{bmatrix}$. If $c>0$, then $c\overrightarrow{x}$ is in the same direction as $\overrightarrow{x}$. If $c<0$, then $c\overrightarrow{x}$ is in the opposite direction as $\overrightarrow{x}$.

``Insert picture here"


\newpage

For example, if $c = -1$, $d = 2$ and $\overrightarrow{y} = \begin{bmatrix}
4\\-1
\end{bmatrix}$ (Recall that $\overrightarrow{y}$ is a position vector, whose tail starts at the origin), then $c\overrightarrow{y} =\begin{bmatrix}(-1)\cdot(4)\\(-1)\cdot(-1)\end{bmatrix} = \begin{bmatrix}
-4\\1
\end{bmatrix}$ and $d\overrightarrow{y}= \begin{bmatrix}(2)\cdot(4)\\(2)\cdot(-1)\end{bmatrix} = \begin{bmatrix} 8\\-2
\end{bmatrix}$

``insert picture here"

\newpage
\subsection*{3-Dimensional Vectors}

3-dimensional (also known as 3-space) vectors exist in $\mathbb{R}^3$, meaning that we are now dealing another axis, the z-axis. Just like in $\mathbb{R}^2$, there is an origin where all of the axes meet, (0,0,0). Points in $\mathbb{R}^3$ are represented by an ordered triplet (x, y, z). The points (2,1,3), (4, -2, -2), and (1,0,5) would be drawn like so:

"Insert Picture Here"



\newpage

Components for 3-dimensional vectors are defined in the same way as 2-dimensional vectors, except that now there are three of them. So for a vector $\overrightarrow{u}$ whose tail starts at the point (x,y,z) and has its head at the point (x',y',z'), the components of $\overrightarrow{u}$ are:

$$\begin{bmatrix}
x'-x\\y'-y\\z'-z
\end{bmatrix} $$

A 3-dimensional vector whose components are given without information about the head or the tail is a position vector, and you can assume its tail starts at the origin.

\newpage

All of the rules for vector addition and scalar multiplication we presented for 2-dimensional vectors are the same for the 3-dimensional versions. If we have position vector $\overrightarrow{a} = \begin{bmatrix}
a_1\\a_2\\a_3
\end{bmatrix}$ then the magnitude of $\overrightarrow{a}$ is
$$||\overrightarrow{a}|| = \sqrt{a_1^2 + a_2^2 + a_3^2}$$

Keep in mind that we get this formula from the fact that the tail of $\overrightarrow{a}$ is at the origin (0,0,0).

\newpage



\end{document}
