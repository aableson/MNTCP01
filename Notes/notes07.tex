\documentclass[12pt,twoside]{article}

% *** Set page dimensions ***
\raggedbottom
\parindent=0in
%\setlength{\topmargin}{-0.5in}
%\setlength{\oddsidemargin}{0.1875in}
%\setlength{\evensidemargin}{0in}
%\setlength{\textheight}{8.5in}
%\setlength{\textwidth}{6.225in}
%\addtolength{\oddsidemargin}{-0.7in}
%\addtolength{\evensidemargin}{-1.2in}
%\setlength{\oddsidemargin}{-0.2in}
%\setlength{\evensidemargin}{-0.2in}
%\addtolength{\textwidth}{1.4in}
%\addtolength{\topmargin}{-.875in}
%\addtolength{\textheight}{2.00in}

% *** Packages ***
\usepackage{alltt}
\usepackage{tocloft}
\usepackage{graphicx}
\usepackage{lscape}
\usepackage{amssymb}
\usepackage{float}
\usepackage{amsmath}
\usepackage{gensymb}
%\usepackage{subfigure}
\usepackage{lscape}
\usepackage{epsfig}
\usepackage{enumerate}
\usepackage{multicol}
\usepackage{fancyhdr}
\usepackage{epstopdf}
\usepackage{hyperref}
\usepackage{listings}

% *** Table of contents and Sectioning *** 
\setcounter{secnumdepth}{0}
\setcounter{tocdepth}{5}

% *** Table of contents and Sectioning *** 
\newcommand{\next}{\addtocounter{enumi}{9} \item}
\newcommand{\now}[1]{\setcounter{enumi}{#1}}
\newcommand{\Z}{\mbox{\sf Z\hspace{-1.5mm}Z}}
\newcommand{\R}{\mbox{\rm I\hspace{-0.75mm}R}}
\columnsep=0.75in

% *** Shortcuts for syntax ***
\newcommand{\ds}{\displaystyle }
\newcommand{\vsc}{\vspace{4mm}}
\newcommand{\dd}[1]{\frac{d}{d{#1}} \,} 
\newcommand{\ddx}{\frac{d}{dx} \,} 
\newcommand{\ddy}{\frac{d}{dy} \,} 
\newcommand{\ddz}{\frac{d}{dz} \,} 
\newcommand{\dydx}{\frac{dy}{dx} \,} 
\newcommand{\dydt}{\frac{dy}{dt} \,} 
\newcommand{\dfdx}{\frac{df}{dx} \,} 
\newcommand{\ddt}[1]{  \frac{d{#1}}{dt} }
\newcommand{\pp}[2]{  \frac{\partial{#1}}{\partial {#2}} }
\newcommand{\zx}{\frac{\partial z}{\partial x} \,}
\newcommand{\zy}{\frac{\partial z}{\partial y} \,}
\newcommand{\limh}{\lim_{h \rightarrow 0} \;}
\newcommand{\diff}{\frac{d}{dx} \,}
\newcommand{\de}{\Delta}
\renewcommand{\thesection}{\Roman{section}}
\newcommand{\bfr}{\begin{flushright}}
\newcommand{\efr}{\end{flushright}}
\newcommand{\dx}{\frac{\partial f}{\partial x} \,}
\newcommand{\dy}{\frac{\partial f}{\partial y} \,}
\newcommand{\p}{\partial}
\newcommand{\vi}{\vec{i}}
\newcommand{\vj}{\vec{j}}
\newcommand{\vk}{\vec{k}}
\newcommand{\lan}{\left\langle}
\newcommand{\ran}{\right\rangle}
\newcommand{\reading}[1] { {\em Reading: #1}}
\renewcommand{\Pr}{ \mbox{Pr}}

% *** Commands related to textbook references
\newcommand{\problem}{{\bf Problem.} }

% *** Footnoting with symbols ***
\long\def\symbolfootnote[#1]#2{\begingroup%
\def\thefootnote{\fnsymbol{footnote}}\footnote[#1]{#2}\endgroup}

% *** Defining a boxed note ***
\floatstyle{boxed}
\newfloat{noteinbox}{htb}{loa}
\newenvironment{boxnote}{\begin{noteinbox}[H]}{\end{noteinbox}}

\newcommand{\Question}{ {\bf Question: }  }
\newcommand{\Example}[1]{ {\bf Example: } {\em #1} }
\newcommand{\ExampleCont}[1]{ {\em #1} }

% *** Define the boxed Week #/summary at the beginning/end of every chapter ***
\newcommand{\sectionbox}[1]{% 
\begin{tabular}{|p{6in}|}%
\hline%
\ \\ %
{\Large {\bf {#1}}}  \\%
\ \\%
\hline%
\end{tabular}}

% *** Shortcuts *** 
\newcommand\goals{\large {\bf {Goals:}}}
\newcommand\setfont{ }

% *** Week commands: overwritten in each notes file
\newcommand{\Week}{Null-InPreambleCommon}
\newcommand{\WeekTitle}{Null-InPreambleCommon}
\newcommand{\Course}{MNTC P04}
\newcommand{\SetNum}{1 }
\newcommand{\topic}[1]{
\newpage
\setcounter{page}{1}
\fancyhead[LE,RO]{#1 - \thepage}
}

% *** Setup Latex for the large version of the files ***
%\usepackage[landscape]{geometry}
\usepackage[letterpaper,landscape,hmargin={.8in,.8in},vmargin={1in,0.2in}]{geometry}

% Remove paragraph indents
\setlength{\parindent}{0pt}

% Spacing at the top for the header is too large by default
\setlength{\voffset}{-5ex}

% **** RENEW SCALING COMMANDS HERE ****
% *** Text in boxes ***
\renewenvironment{boxnote}{\begin{noteinbox}[H] \huge}{\end{noteinbox}} 

% *** Chapter lead in/summary boxes ***
\renewcommand{\sectionbox}[1]{% 
\begin{tabular}{|p{9.5in}|}%
\hline%
\ \\ %
{\huge {\bf {#1}}}  \\%
\ \\%
\hline%
\end{tabular}}

% *** 'Section'' commands, which are sometimes used for spacing
% From http://zoonek.free.fr/LaTeX/LaTeX_samples_section/0.html
\makeatletter
 \renewcommand\section{\@startsection {section}{1}{\z@}%
                                    {-3.5ex \@plus -1ex \@minus -.2ex}%
                                    {0.3ex \@plus.2ex}%
                                    {\setfont\bf}}

 \renewcommand\subsection{\@startsection {subsection}{1}{\z@}%
                                    {-3.5ex \@plus -1ex \@minus -.2ex}%
                                    {0.3ex \@plus.2ex}%
                                    {\setfont\bf}}

% *** 'Goals' should be larger in the overheads ***
\renewcommand\goals{\huge {\bf {Goals:}}}
\renewcommand\setfont{\huge }

\thispagestyle{empty}

\setfont 

\newcommand{\WeekTitleOne}{Derivatives - Foundations}
\newcommand{\WeekTitleTwo}{Derivatives - Linearization and Applications}
\newcommand{\WeekTitleThree}{Derivatives - Modeling}
\newcommand{\WeekTitleFour}{Integrals - Foundations}
\newcommand{\WeekTitleFive}{Integrals - Techniques}
\newcommand{\WeekTitleSix}{Integrals - Modeling}
\newcommand{\WeekTitleSeven}{Differential Equations - }
\newcommand{\WeekTitleEight}{Differential Equations - }
\newcommand{\WeekTitleNine}{Differential Equations - }
\newcommand{\WeekTitleTen}{Linear Algebra - }
\newcommand{\WeekTitleEleven}{Linear Algebra - }
\newcommand{\WeekTitleTwelve}{Linear Algebra - }



\begin{document}
\setfont
\pagestyle{fancy}
\renewcommand{\Week}{7 }
\renewcommand{\WeekTitle}{\WeekTitleSeven }

\fancyhead[LE,RO]{Week \Week}  % default, usually only for first page
\fancyfoot{}
\sectionbox{Week \#\Week: \WeekTitle}

\vspace{5mm}
\noindent

\goals
\begin{itemize}
\item Determine how to calculate the area described by a function.
\item Define the {\bf definite integral}.
\item Explore the relationship between the definite integral and area.
\item Explore ways to estimate the definite integral.
\end{itemize}

\newpage

\topic{Integration} 
\subsection*{Integration} 
If we had to summarize the first half of
the course, we would have to say that the focus was on {\bf
  differentiation}. 

All differentiation problems ask the same basic question: {\bf At what
  rate does a process change}, and how does that rate of change relate
to other characteristics of the process?

The key observation was that {\bf at small scales, rates of change look linear}.

\newpage

In the remainder of the course we will study {\bf integration}.
Again, the analysis will be made possible by the observation that on a very
small scale all processes look linear.  This time, though, we will use
this fact to see how regarding a process as an {\bf accumulation} of
infinitely many small linear steps allows us to calculate the
accumulated total even when the rate of accumulation is far from
linear.  {\bf Integration is always in some way about finding the
  total at the end of a process of accumulation.}

\newpage


 \topic{Distance and Velocity}
 \subsection*{Distance and Velocity}

 Recall that if we measure distance $x$ as a function of time $t$, the
 velocity is determined by differentiating $x(t)$: 
 \begin{center}
{\bf Velocity} is the {\bf slope} on the {\bf position graph}.
 \end{center}

 But now suppose we begin with a {\bf graph of the velocity} with
 respect to time.  How can we determine what {\bf distance} will be
 traveled?  Does distance also ``appear'' in the velocity graph
 somehow?

\newpage

\problem 
 Consider the graph for the velocity of a particle shown below. 
\begin{center}
\includegraphics[width=0.25\linewidth]{graphics/w07_const_velocity}
\end{center}
How far did the particle travel between $t=0$ and $t=5$ seconds?
\begin{enumerate}[(a)]
\item 5 m\\[1ex]
\item 10 m\\[1ex]
\item 15 m\\[1ex]
\item 20 m
\end{enumerate}

\newpage
\problem 
~
\begin{center}
\includegraphics[width=0.25\linewidth]{graphics/w07_const_velocity}
\end{center}

Where does the distance travelled between $t=0$ and $t=5$ ``appear''
on this velocity graph?
\begin{enumerate}[(a)]
\item The distance travelled is the {\bf slope} between $t=0$ and $t=5$  \\[1ex]
\item The distance travelled is the {\bf average height} of the graph between $t=0$ and $t=5$  \\[1ex]
\item The distance travelled is the {\bf area} under the graph between $t=0$
  and $t=5$.
\end{enumerate}

\newpage

When the velocity is {\bf constant}, we have the equivalency:
\vspace{0.2in}
\begin{center}
dist = vel $\times$ time ~~~~ $\Longleftrightarrow$ dist = area under the velocity graph
\end{center}
\vspace{0.2in}

\problem 
What about when the velocity is {\bf not} constant though?

\begin{center}
\includegraphics[width=0.35\linewidth]{graphics/w07_variable_velocity}
\end{center}

Do the units of the ``area'' under this graph still make sense as a
distance value?


\vfill
\newpage
 \topic{Calculating Areas}
\subsection*{Calculating Areas}

It appears that the distance traveled is the area under the graph of
velocity, even when the velocity is changing. We'll see exactly why
this is true very soon.

If we are simply interested in the area under a graph, without any
physical interpretation, we can already do so if the graph creates a
shape that we recognize.
\newpage

\problem Calculate the area between the $x$-axis and the graph of $y = x+1$ from $x=-1$ and $x=1$.

\hfill \includegraphics[width=0.25\linewidth]{graphics/w07_linear_area}
\newpage

\problem Calculate the area between the $x$-axis 
and the graph of $y = \sqrt{1 - x^{2}}$ from $x = -1$ to $x = 1$.

\hfill \includegraphics[width=0.3\linewidth]{graphics/w07_semi_circle_area}

\vfill

\problem What shapes do you know, right now, for which you can
calculate the exact area?

\vfill

\newpage

\topic{Estimating Areas}
\subsection*{Estimating Areas}

Unfortunately, many or most arbitrary areas are essentially impossible
to find the area of when the shape isn't a simple composition of
triangles, rectangles, or circles.

In these cases, we must use less direct methods.  We will start by
making an {\em estimate} of the area under the graph using shapes
whose area is easier to calculate.

\newpage

\problem Suppose we are trying to find the area underneath the graph of the
function $f(x)$ given below between $x = 1$ and $x=4$.  Shade in that
region, and call that area $A$.

\begin{center}
\includegraphics[width=4in]{graphics/w07_graph06}
\end{center}


\newpage
We can make a rough estimation of the area by drawing a rectangle that
completely contains the area, or a rectangle that is completely
contained by the area.


\problem 
{Calculate this overestimate and underestimate for the
  area $A$.}
  
\hfill \includegraphics[width=4in]{graphics/w07_graph06}


\newpage The next step is to use smaller rectangles to improve our
estimate the area.  We can divide the interval from $x = 1$ to $x = 4$
into 3 intervals of width 1, and use different size rectangles on each
interval.



\problem Estimate the area $A$ by using 3 rectangles of width 1.
  Use the function value at the {\em left} edge of the interval as the
  height of each rectangle.

\hfill \includegraphics[width=0.43\linewidth]{graphics/w07_graph06}

\newpage

We can repeat this process for any number of rectangles, and we expect
that our estimation of the area will get better the more rectangles we
use.  The method we used above, choosing for the height of the
rectangles the function at the left edge, is called the {\bf left hand
  sum}, and is denoted {\em LEFT(n)} if we use $n$ rectangles.

\newpage

\topic{Generalizing Area Calculation with Summation Notation}
\subsection*{Generalizing the Area Calculation}

Suppose we are trying to estimate the area under the function $f(x)$
from $x = a$ to $x = b$ via the left hand sum with $n$ rectangles.
\begin{itemize}
\item the {\bf width} of each rectangle will be $\Delta x =
\displaystyle\frac{b-a}{n}$.  
\item If we label the endpoints of the
intervals to be $a = x_{0} < x_{1} < \cdots < x_{n-1} < x_{n} = b$,
then the {\bf heights} of the rectangles will be $f(x_{i})$'s, and
\item the formula for the left hand sum/area will be
\end{itemize}

\begin{align*}
  LEFT(n) = & f(x_{0}) \Delta x+ 
 f(x_{1}) \Delta x + \ldots + 
 f(x_{n-1}) \Delta x 
\\  = & \displaystyle\sum_{i=1}^{n} f(x_{i-1})\Delta x.
\end{align*}

\newpage

\subsection*{Aside: Summation Notation}

The capital greek letter sigma, $\sum$, is used as a shortform notation for long sums.
E.g.
$$\ds \sum_{i=1}^n x_i$$

\newpage
\problem 
Translate the following into more traditional sums:

$\ds \sum_{i=1}^{100} i= $ \vfill 

$\ds \sum_{n=1}^{10} n^2 + n= $  \vfill

$\ds \sum_{i=1}^{n}  f(x_{i-1}) ~\Delta x= $
\vfill

\newpage
\topic{Back to Areas!} 
\subsection*{Back to Areas!} 

Thinking back to our LEFT sum to estimate areas, we have a similar
definition for the {\bf right hand sum}, or {\em RIGHT(n)}. This is
calculated by taking the height of each rectangle to be the height of
the function at the {\bf right hand endpoint} of the interval.

\begin{align*}
RIGHT(n) = & f(x_{1}) \Delta x+ 
f(x_{2}) \Delta x + \ldots + 
 f(x_{n}) \Delta x \\ 
= &  \displaystyle\sum_{i=1}^{n} f(x_{i})\Delta x.
\end{align*}

\bigskip

\newpage 

{Calculate LEFT(6) and RIGHT(6) for the function shown,
  between $x=1$ and $x=4$.  You will need to estimate some rectangle
  heights from the graph.}

\includegraphics[width=0.4\linewidth]{graphics/w07_graph06}
\hfill \includegraphics[width=0.4\linewidth]{graphics/w07_graph06}
\newpage

%{In general, when will LEFT(n) be greater than RIGHT(n)? }
%\vfill 
%
%{ When will LEFT(n) be an overestimate for the
%  area?}  \vfill 
%
%{ When will LEFT(n) be an underestimate?}
%
%\vfill
%
%\newpage

\topic{Riemann Sums}
\subsection*{Riemann Sums}

Area estimations like $LEFT(n)$ and $RIGHT(n)$ are often called {\bf
  Riemann sums}, after the mathematician Bernahrd Riemann (1826-1866)
who formalized many of the techniques of calculus.  The general form
for a Riemann Sum is

\begin{align*}
 f(x_{1}^{*}) \Delta x+ f(x_{2}^{*}) \Delta x + \ldots + f(x_{n}^{*}) \Delta x 
 \\ = \displaystyle\sum_{i=1}^{n} f(x_{i}^{*})\Delta x
\end{align*}

where each $x_{i}^{*}$ is some point in the interval $[x_{i-1},
x_{i}]$.  For $LEFT(n)$, we choose the left hand endpoint of the
interval, so $x_{i}^* = x_{i-1}$; for $RIGHT(n)$, we choose the right
hand endpoint, so $x_{i}^* = x_{i}$.  \newpage 

The common property of all these
approximations is that they involve
\begin{itemize}
\item a sum of rectangular areas, with \\[1ex]
\item widths ($\Delta x$), and \\[1ex]
\item heights ($f(x_i^*)$)
\end{itemize}



There are other Riemann Sums that give slightly better estimates of
the area underneath a graph, but they often require extra computation.
We will examine some of these other calculations a little later.

\newpage

\topic{The Definite Integral}
\subsection*{The Definite Integral}

We observed that as we increase the number of rectangles used to
approximate the area under a curve, our estimate of the area under the
graph becomes more accurate.  This implies that to obtain the {\bf
  exact area}, we should use a {\bf limit} on our Riemann sums.

\medskip
\begin{center}
The area underneath the graph of $f(x)$ between $x=a$ and $x=b$ is equal to $\displaystyle\lim_{n \to \infty} LEFT(n) = 
\displaystyle\lim_{n \to \infty} \displaystyle\sum_{i=1}^{n} f(x_{i-1})\Delta x$, ~~~where $\Delta x = \displaystyle\frac{b-a}{n}$.
\end{center}

This limit is called the {\em definite integral} of $f(x)$ from $a$ to $b$, and is equal to the area under curve whenever $f(x)$ is a non-negative continuous function.  The definite integral is written with some special notation.

\newpage

  {\bf{Notation for the Definite Integral}} 
  
  The definite integral of $f(x)$ between $x=a$ and
  $x=b$ is denoted by the symbol $$\int_{a}^{b} f(x) \,dx$$
  We call $a$ and $b$ the {\bf{limits of integration}} and $f(x)$ the
  {\bf{integrand}}.  The $dx$ denotes which variable we are using; this will become important for using some techniques for calculating definite integrals.
Note that this notation shares the same common structure with Riemann sums: 
\begin{itemize}
\item a sum ($\ds \int $ sign) \\[1ex]
\item widths ($dx$), and \\[1ex]
\item heights ($f(x)$)
\end{itemize}

\newpage 

\problem Write the definite integral representing the area underneath
  the graph of $f(x) = x + \cos x$ between $x=2$ and $x=4$.

\vfill

\newpage

\problem 
  Use a right hand sum with $n=4$ to estimate $\displaystyle
  \int_{0}^{2}x^2\,dx$.  

\newpage

\vfill Sketch how the $n=$ right hand sum calculation would be
represented graphically, and how it differs from the exact value of
$\ds \int_0^2 x^2~dx$.

\includegraphics[width=0.3\linewidth]{graphics/w07_parabola_for_areas}
\hfill
\includegraphics[width=0.3\linewidth]{graphics/w07_parabola_for_areas}

\newpage

% You can get MuPAD to draw the diagram associated with the midpoint
% rule for this integral by first creating the plot object for the graph:

% [\, {\bf GRPH := plot::Function2d(x$\wedge$2, x=0..2)}

% and then use this graph to produce the plot object for the midpoint rule with 4 intervals:

% [\, {\bf RECTS := plot::Integral(GRPH, IntMethod=RiemannMiddle, Nodes=[4])}

% You can then show them together by typing

% [\, {\bf plot(GRPH, RECTS)}

% The values of the Riemann sum and the integral are automatically printed on the diagram.

%\newpage

\topic{Negative Integral Values}
\subsection*{Negative Integral Values}

So far we have only dealt with the areas under/intergrals of {\bf
  positive} functions.  Will the definite integral still be equal to
the area underneath the graph if $f(x)$ is always negative?  What
happens if $f(x)$ crosses the $x$-axis several times?



\problem Suppose that $f(x)$ has the graph shown below, and that A, B,
  C, D, and E are the areas of the regions shown.

\begin{center}
\includegraphics[width=0.6\linewidth]{graphics/w07_negative_area_example}
\end{center}

\newpage
\begin{center}
\includegraphics[width=0.4\linewidth]{graphics/w07_negative_area_example}
\end{center}

If we were to partition $[a,b]$ into small subintervals and construct a corresponding Riemann sum, then the first few terms in the Riemann sum would correspond to the region with area A, the next few to B, etc. 

\problem 
Which of these sets of terms have positive values?

\vfill

Which of these sets have negative values?
\vfill

\newpage

\begin{center}
\includegraphics[width=0.3\linewidth]{graphics/w07_negative_area_example}
\end{center}
\problem 
{Express the integral $\displaystyle \int_a^b f(x) dx$ in
  terms of the (positive) areas A, B, C, D, and E.  }

\vfill
\vfill

{If $f$ were to represent velocity over time, what would the ``negative
  areas'' represent?}

\vfill
\newpage


%\newpage
% \problem 
% The answer we found in the preceding question is
% \begin{enumerate}[A.]
% \item An underestimate because the first rectangle is very small.
% \item An overestimate because this is an increasing function.
% \item An underestimate because the graph of $f$ is concave upward.
% \item An overestimate because we used the midpoints.
% \end{enumerate}
% 

\newpage
\begin{boxnote}
{\bf The Role of Riemann sums}
\begin{enumerate}
\item They are needed to say what we mean by an integral. \\[2ex]
\item They enable us to decide which integral is appropriate in a word
problem. \\[2ex]
\item They can also be used to give an approximate value of the
integral. 
\end{enumerate}
\end{boxnote}

\vspace{0.5in}

We will explore how to avoid needing Riemann sums for calculations next week, through the Fundamental Theorem of Calculus! 

\end{document}

