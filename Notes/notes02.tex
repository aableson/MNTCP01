\input{PreambleCommon}
\newcommand{\WeekTitleOne}{Derivatives - Foundations}
\newcommand{\WeekTitleTwo}{Derivatives - Linearization and Applications}
\newcommand{\WeekTitleThree}{Derivatives - Applications}
\newcommand{\WeekTitleFour}{Integrals - Foundations}
\newcommand{\WeekTitleFive}{Integrals - Techniques}
\newcommand{\WeekTitleSix}{Integrals - Modeling}
\newcommand{\WeekTitleSeven}{Differential Equations - }
\newcommand{\WeekTitleEight}{Differential Equations - }
\newcommand{\WeekTitleNine}{Differential Equations - }
\newcommand{\WeekTitleTen}{Linear Algebra - }
\newcommand{\WeekTitleEleven}{Linear Algebra - }
\newcommand{\WeekTitleTwelve}{Linear Algebra - }



\begin{document}
\setfont
\pagestyle{fancy}
\renewcommand{\Week}{2 }
\renewcommand{\WeekTitle}{\WeekTitleTwo }

\fancyhead[LE,RO]{Week \Week}  % default, usually only for first page
\fancyfoot{}
\sectionbox{Week \#\Week: \WeekTitle}


\vspace{5mm}
\goals
\begin{itemize}
\item Describe the meaning and value of linearization
\item Apply the technique of linearization to solve a variety of
  nonlinear equations
\item Use MATLAB to graph and compare functions with their
  linearizations
\item Use MATLAB to implement Newton's method
\item Calculate and interpret the first and second derivatives, as
  well as higher order derivatives
 \end{itemize}

\vspace{5mm}

\newpage
\topic{Linear Approximations}
\subsection*{Linear Approximations}

You should now feel comfortable in finding the derivative of a wide
variety of functions with formulas. \\[1ex]

In this section, we will explore how the derivatives you can compute
can be tied back to understanding the behaviour of the original
function. \\[1ex]

We will start by returning to the definition of the derivative, based
on the $\ds \frac{\mbox{rise}}{\mbox{run}}$ formula for slopes: 
\begin{align*} 
  f'(x) = \frac{d}{dx} f = \dfdx = \lim_{\Delta x \to 0} \frac{\Delta f}{\Delta x} = \lim_{h \to 0} \frac{f(x+h) - f(x)}{h}
\end{align*} 



\newpage 
For example, if $y = f(x)$, then 
\[ \lim_{\Delta x \to 0} \frac{\Delta y}{\Delta x} = f'(x) 
\]
\begin{problem}
 What is the relationship between $f'$ and $\Delta y$,
	$\Delta x$ for merely {\em small} delta values? 
\end{problem}
\vfill
         Now sketch a graph, and label the two points
        in the $\Delta x$ difference `$x$' and `$a$'.  What expression
        do we obtain for $f(x)$?

\newpage
We want to have a name for the RHS of this approximation:
\begin{boxnote}
\[ L(x)=f(a) + f'(a) (x-a) 
\]
\end{boxnote}
\begin{problem}
 What are some names for this linear function?
\end{problem}


\newpage
\begin{problem}
Consider the tangent line approximation to the graph
of $f(x)=e^x$ at $(0,1)$, where we know that $f^\prime (0) = 1$. 

A good approximation for $e^{0.5}$ is therefore: \\[2ex]
\begin{enumerate}[A.]
\item $e^{0.5 }\approx 0.5$ \\[2ex]
\item $e^{0.5 }\approx 1+e^{.5}$ \\[2ex]
\item $e^{0.5 }\approx 1+.5$ \\[2ex]
\end{enumerate}

\end{problem}

\newpage
\topic{Linear Approximations for Error Estimation}
\begin{problem}
  A rock formation with high density ores was identified using gravity
  measurements; the formation is roughly cubic in shape. The edge each
  side of the cube was found to be 370 m, with a possible error in
  measurement of 10 m.  Use the ideas of linear approximations to
  estimate the maximum possible error (positive or negative) in
  computing the {\bf volume} of the formation.
\end{problem}
	\vfill
	\vfill

\newpage
\problem What are the trade-offs of using the linear approximation to
obtain the above error estimate, compared to a direct calculation of
the possible volumes with the error measurements?  \vfill

\newpage

%\conques{%
% A function $f$ and its derivative have the following values at 0 and
% 10:%
% \begin{center}
% \begin{tabular}{cc}
% $f(0) = 2$ & $f(10) = 5$ \\
% $f^\prime (0) = -1$ & $f^\prime (10) = 1$
% \end{tabular}
% \end{center}
% Which of the following calculations produces the best estimate for
% $f(0.2)$?
% \begin{enumerate}
% \item[A. ] $(0.02)\times (5-2) + 2 = 2.06$
% \item[B. ] 2
% \item[C. ] $(0.2)\times (-1) = -0.2$
% \item[D. ] $(-1)\times (0.2) + 2 = 1.8$
% \end{enumerate}
% }%

\topic{Tangent Lines to Graphs}

\begin{problem}
Sketch the function $\ds f(x)=\frac{1}{x}$.
\end{problem}

\includegraphics[width=4in]{graphics/empty_graph_square_12}

\newpage

\begin{problem}
If we drew a tangent line to $f(x)$ at $x=4$, what range we would expect for the slope there?

\begin{enumerate}[A.]
\item Slope above 1. \\[2ex]
\item Slope between 0 and 1.  \\[2ex]
\item Slope between 0 and -1.\\[2ex] 
\item Slope below -1.
\end{enumerate}
\end{problem}



\newpage
\begin{problem}
 Find the linearization of $\ds f(x)=\frac{1}{x}$ at $a=4$.
\end{problem}
\vfill
\vfill

 Find the equation of the tangent line to the hyperbola $\ds y=\frac{1}{x} $ at $x=4$. 
\vfill

\newpage

\problem Sketch the graph of $\ds y=\frac{1}{x}$, and its tangent line at $x=4$. 

\includegraphics[width=4in]{graphics/empty_graph_square_12}




% \noindent \ques{Find the linearization of $f(x)=1/x$ at $a=4$.}
% %%%%% begin solution %%%%%%%%%%%%%%%%%%%%%%%%%%%%%%%%%%%%%%%%%%%
% \includesoln{\solntest}{\mbox{}\\[11cm]}{
% We can use our ``recipe'' to find the linearization of $f$:
% $$L(x)=f(a)+f'(a)(x-a)$$
% When $x=a$,
% \begin{align*}
% f(a) & = f(4)\\[0.1cm]
% & =\frac{1}{4}
% \end{align*}
% and
% \begin{align*}
% f'(a) & =-\frac{1}{a^2}\\[0.1cm]
% & = -\frac{1}{4^2}\\[0.1cm]
% & = -\frac{1}{16}
% \end{align*}
% So the linearization of $f(x)$ at $x=a$ is
% $$L(x)=\frac{1}{4}-\frac{1}{16}(x-4).\vi{1}$$}
% %%%%% end solution %%%%%%%%%%%%%%%%%%%%%%%%%%%%%%%%%%%%%%%%%%%%%
% \ques{Find the equation of the tangent line to the hyperbola $y=1/x$
% at $x=4$.}
% %%%%% begin solution %%%%%%%%%%%%%%%%%%%%%%%%%%%%%%%%%%%%%%%%%%%
% \includesoln{\solntest}{}{
% The tangent line of $y=\frac{1}{x}$ at $x=4$ and the line
% approximating the hyperbola $y=\frac{1}{x}$ at $x=4$ (linearization
% of the hyperbola) are the same line!  So using the previous example,
% we can write that the tangent line to the hyperbola at $x=4$ is
% $$y=\frac{1}{4}-\frac{1}{16}(x-4).$$}
%%%%% end solution %%%%%%%%%%%%%%%%%%%%%%%%%%%%%%%%%%%%%%%%%%%%%

\newpage
\topic{The $\sin(x)$ Approximation}
\subsection*{The $\sin(x)$ Approximation}
One of the most commonly-used approximation in physics
is the relationship
\[ \sin(x) \approx x
\]
\begin{problem}
 Derive this relationship using linearization. \vfill
\end{problem}

 What is the fine-print that should {\bf always} be associated with this approximation? \vspace{1in}

\newpage

Sketch the graphs of $y = \sin(x) $ and $y = x$. Focus on the domain $\ds \frac{-\pi}{2} \le x \le  \frac{\pi}{2}$.

\includegraphics[width=4in]{graphics/empty_graph_square_12}


\newpage  Below are more detailed calculations relating
$\sin(x)$ and $x$.  

By filling in some or all of the missing values, determine the range
of angles which the {\em relative error} in the approximation $\sin(x)
\approx x$ is {\bf less than 1\%}. State your answer in degrees and radians.
\vspace{0.2in}

{\LARGE
	\begin{tabular}{||r||c|c|c|c|c|c|c|c||} \hline
	$x$ (degrees) &  & & & & & & & \\ \hline
	$x$ (rad) & 0.0500&          0.1000&          0.1500  & 0.2000  &0.2500 & 0.3000 & 0.3500 & 0.4000\\ \hline
	$\sin(x)$ & 0.0500 &0.0998 &0.1494 &0.1987 &0.2474 &0.2955 &0.3429 &0.3894 \\ \hline
	abs err & 0.0000 &0.0002 &0.0006 &0.0013 &0.0026 &0.0045 &0.0071 &0.0106 \\ \hline
	rel err &  & & & & & & & \\ \hline
	\end{tabular}
}
	

\newpage
 Just for fun, put your calculator into degree mode, and see whether $\sin(x) \approx x$ still holds for small $x$.
	

\newpage
  \end{document}

