\documentclass[12pt,twoside]{article}

% *** Set page dimensions ***
\raggedbottom
\parindent=0in
%\setlength{\topmargin}{-0.5in}
%\setlength{\oddsidemargin}{0.1875in}
%\setlength{\evensidemargin}{0in}
%\setlength{\textheight}{8.5in}
%\setlength{\textwidth}{6.225in}
%\addtolength{\oddsidemargin}{-0.7in}
%\addtolength{\evensidemargin}{-1.2in}
%\setlength{\oddsidemargin}{-0.2in}
%\setlength{\evensidemargin}{-0.2in}
%\addtolength{\textwidth}{1.4in}
%\addtolength{\topmargin}{-.875in}
%\addtolength{\textheight}{2.00in}

% *** Packages ***
\usepackage{alltt}
\usepackage{tocloft}
\usepackage{graphicx}
\usepackage{lscape}
\usepackage{amssymb}
\usepackage{float}
\usepackage{amsmath}
\usepackage{gensymb}
%\usepackage{subfigure}
\usepackage{lscape}
\usepackage{epsfig}
\usepackage{enumerate}
\usepackage{multicol}
\usepackage{fancyhdr}
\usepackage{epstopdf}
\usepackage{hyperref}
\usepackage{listings}

% *** Table of contents and Sectioning *** 
\setcounter{secnumdepth}{0}
\setcounter{tocdepth}{5}

% *** Table of contents and Sectioning *** 
\newcommand{\next}{\addtocounter{enumi}{9} \item}
\newcommand{\now}[1]{\setcounter{enumi}{#1}}
\newcommand{\Z}{\mbox{\sf Z\hspace{-1.5mm}Z}}
\newcommand{\R}{\mbox{\rm I\hspace{-0.75mm}R}}
\columnsep=0.75in

% *** Shortcuts for syntax ***
\newcommand{\ds}{\displaystyle }
\newcommand{\vsc}{\vspace{4mm}}
\newcommand{\dd}[1]{\frac{d}{d{#1}} \,} 
\newcommand{\ddx}{\frac{d}{dx} \,} 
\newcommand{\ddy}{\frac{d}{dy} \,} 
\newcommand{\ddz}{\frac{d}{dz} \,} 
\newcommand{\dydx}{\frac{dy}{dx} \,} 
\newcommand{\dydt}{\frac{dy}{dt} \,} 
\newcommand{\dfdx}{\frac{df}{dx} \,} 
\newcommand{\ddt}[1]{  \frac{d{#1}}{dt} }
\newcommand{\pp}[2]{  \frac{\partial{#1}}{\partial {#2}} }
\newcommand{\zx}{\frac{\partial z}{\partial x} \,}
\newcommand{\zy}{\frac{\partial z}{\partial y} \,}
\newcommand{\limh}{\lim_{h \rightarrow 0} \;}
\newcommand{\diff}{\frac{d}{dx} \,}
\newcommand{\de}{\Delta}
\renewcommand{\thesection}{\Roman{section}}
\newcommand{\bfr}{\begin{flushright}}
\newcommand{\efr}{\end{flushright}}
\newcommand{\dx}{\frac{\partial f}{\partial x} \,}
\newcommand{\dy}{\frac{\partial f}{\partial y} \,}
\newcommand{\p}{\partial}
\newcommand{\vi}{\vec{i}}
\newcommand{\vj}{\vec{j}}
\newcommand{\vk}{\vec{k}}
\newcommand{\lan}{\left\langle}
\newcommand{\ran}{\right\rangle}
\newcommand{\reading}[1] { {\em Reading: #1}}
\renewcommand{\Pr}{ \mbox{Pr}}

% *** Commands related to textbook references
\newcommand{\problem}{{\bf Problem.} }

% *** Footnoting with symbols ***
\long\def\symbolfootnote[#1]#2{\begingroup%
\def\thefootnote{\fnsymbol{footnote}}\footnote[#1]{#2}\endgroup}

% *** Defining a boxed note ***
\floatstyle{boxed}
\newfloat{noteinbox}{htb}{loa}
\newenvironment{boxnote}{\begin{noteinbox}[H]}{\end{noteinbox}}

\newcommand{\Question}{ {\bf Question: }  }
\newcommand{\Example}[1]{ {\bf Example: } {\em #1} }
\newcommand{\ExampleCont}[1]{ {\em #1} }

% *** Define the boxed Week #/summary at the beginning/end of every chapter ***
\newcommand{\sectionbox}[1]{% 
\begin{tabular}{|p{6in}|}%
\hline%
\ \\ %
{\Large {\bf {#1}}}  \\%
\ \\%
\hline%
\end{tabular}}

% *** Shortcuts *** 
\newcommand\goals{\large {\bf {Goals:}}}
\newcommand\setfont{ }

% *** Week commands: overwritten in each notes file
\newcommand{\Week}{Null-InPreambleCommon}
\newcommand{\WeekTitle}{Null-InPreambleCommon}
\newcommand{\Course}{MNTC P04}
\newcommand{\SetNum}{1 }
\newcommand{\topic}[1]{
\newpage
\setcounter{page}{1}
\fancyhead[LE,RO]{#1 - \thepage}
}

% *** Setup Latex for the large version of the files ***
%\usepackage[landscape]{geometry}
\usepackage[letterpaper,landscape,hmargin={.8in,.8in},vmargin={1in,0.2in}]{geometry}

% Remove paragraph indents
\setlength{\parindent}{0pt}

% Spacing at the top for the header is too large by default
\setlength{\voffset}{-5ex}

% **** RENEW SCALING COMMANDS HERE ****
% *** Text in boxes ***
\renewenvironment{boxnote}{\begin{noteinbox}[H] \huge}{\end{noteinbox}} 

% *** Chapter lead in/summary boxes ***
\renewcommand{\sectionbox}[1]{% 
\begin{tabular}{|p{9.5in}|}%
\hline%
\ \\ %
{\huge {\bf {#1}}}  \\%
\ \\%
\hline%
\end{tabular}}

% *** 'Section'' commands, which are sometimes used for spacing
% From http://zoonek.free.fr/LaTeX/LaTeX_samples_section/0.html
\makeatletter
 \renewcommand\section{\@startsection {section}{1}{\z@}%
                                    {-3.5ex \@plus -1ex \@minus -.2ex}%
                                    {0.3ex \@plus.2ex}%
                                    {\setfont\bf}}

 \renewcommand\subsection{\@startsection {subsection}{1}{\z@}%
                                    {-3.5ex \@plus -1ex \@minus -.2ex}%
                                    {0.3ex \@plus.2ex}%
                                    {\setfont\bf}}

% *** 'Goals' should be larger in the overheads ***
\renewcommand\goals{\huge {\bf {Goals:}}}
\renewcommand\setfont{\huge }

\thispagestyle{empty}

\setfont 

\newcommand{\WeekTitleOne}{Derivatives - Foundations}
\newcommand{\WeekTitleTwo}{Derivatives - Linearization and Applications}
\newcommand{\WeekTitleThree}{Derivatives - Modeling}
\newcommand{\WeekTitleFour}{Integrals - Foundations}
\newcommand{\WeekTitleFive}{Integrals - Techniques}
\newcommand{\WeekTitleSix}{Integrals - Modeling}
\newcommand{\WeekTitleSeven}{Differential Equations - }
\newcommand{\WeekTitleEight}{Differential Equations - }
\newcommand{\WeekTitleNine}{Differential Equations - }
\newcommand{\WeekTitleTen}{Linear Algebra - }
\newcommand{\WeekTitleEleven}{Linear Algebra - }
\newcommand{\WeekTitleTwelve}{Linear Algebra - }



\begin{document}
\setfont
\pagestyle{fancy}
\renewcommand{\Week}{5 }
\renewcommand{\WeekTitle}{\WeekTitleFive }

\fancyhead[LE,RO]{Week \Week}  % default, usually only for first page
\fancyfoot{}
\sectionbox{Week \#\Week: \WeekTitle}


\vspace{5mm}
\goals
\begin{itemize}
\item Recognize the family of functions that can be solved with the technique of
integration by substitution.
\item Solve integration problems using the technique of substitution. 
\item Recognize the family of functions that can be solved with the technique of
integration by parts. 
\item Solve integration problems using the technique of integration by parts. 
\end{itemize}
\vspace{5mm}



\topic{Integration Method - Guess and Check}

We now return to the challenge of finding a {\em formula} for an
anti-derivative function.  We saw simple cases last week, and now we
will extend our methods to handle more complex integrals.

\section*{Anti-differentiation by Inspection:\\ The Guess-and-Check Method}


\vsc

Often, even if we do not see an anti-derivative immediately, we can
make an educated guess and eventually arrive at the correct answer.
\begin{flushright} [See also H-H, p. 332-333] \end{flushright}

\vsc 

\newpage
\problem Based on your knowledge of derivatives, what should the
  anti-derivative of $\cos(3x)$, $\ds \int \cos(3x) ~dx$, look like?

\vfill

\vfill

\newpage

\problem Find $\ds \int e^{3x-2} ~dx$.

\vfill

\newpage

\problem Both of our previous examples had {\em linear} `inside'
  functions.  Here is an integral with a {\em quadratic} `inside' function:
$$\ds{\int x e^{-x^2} ~dx}$$
Evaluate the integral.  \vfill \vfill

Why was it important that there be a factor $x$ in front
  of $e^{-x^2}$ in this integral?

\vfill

\newpage
\topic{Integration Method - Substitution}
\section*{Integration by Substitution}

We can formalize the guess-and-check method by defining an {\em
  intermediate variable} the represents the ``inside'' function.

\problem Show that $\ds \int x^3 \sqrt{x^4 + 5} ~dx = {1 \over 6} (x^4
  + 5)^{3/2} + C$.

\vfill
\vfill

\newpage
$$\ds \int x^3 \sqrt{x^4 + 5} ~dx = {1 \over 6} (x^4
  + 5)^{3/2} + C$$

\problem Relate this result to the {\bf chain rule}.

\vfill

\vsc

\newpage

\problem Now use the {\bf method of substitution} to evaluate $\ds
  \int x^3 \sqrt{x^4 + 5}~dx$

\vfill
\vfill
\vfill
\vfill


\newpage

\topic{Substitution Integrals - Example 1}

{\bf Steps in the Method Of Substitution}

\begin{enumerate}[1.]
\item Select a simple function $w(x)$ that appears in the integral.
  \begin{itemize}
  \item Typically, you will also see $w'$ as a {\bf factor} in the
    integrand as well.
  \end{itemize}
\item Find $\ds \frac{dw}{dx}$ by differentiating.  Write it in the form
  $\ldots dw = ~dx$
  \item Rewrite the integral using only $w$ and $dw$ (no $x$ nor
    $dx$).
  \begin{itemize}
  \item If you can now evaluate the integral, the substitution
was effective. 
\item If you cannot remove all the $x$'s, or the integral became
  harder instead of easier, then either try a different substitution,
  or a different integration method.
  \end{itemize}
\end{enumerate}

\newpage


\problem Find $\ds{ \int \tan(x) ~dx}$.

\vsc 

\vfill
\vfill
\vfill
\vfill

\newpage

Though it is not required unless specifically requested, it can be reassuring to check the answer.

\problem Verify that the anti-derivative you found is correct.

\vfill
\newpage

\topic{Substitution Integrals - Example 2}
\problem Find $\ds{\int x^3 e^{x^4 - 3} dx}$.

\newpage 
\topic{Substitution Integrals - Example 3}

\problem For the integral, $$ \int{\frac{e^x - e^{-x}}{(e^x + e^{-x})^2}}
  dx \ \ $$ both $w = e^{x} - e^{-x}$ and $w = e^{x} + e^{-x}$ are
  seemingly reasonable substitutions.  

\Question{Which substitution will change
  the integral into the simpler form?}

  \begin{enumerate}[1.]
  \item  $ w = e^x - e^{-x}$ 
  \item  $ w = e^x + e^{-x}$ 
  \end{enumerate}

\newpage

\problem Compare both substitutions in practice.
$$ \int{\frac{e^x - e^{-x}}{(e^x + e^{-x})^2}}~dx$$
\begin{center}
\begin{tabular}{l|r}
   with $ w = e^x - e^{-x}$ ~~~~~ &~~~~~
   with $ w = e^x + e^{-x}$  \\
~& \\
~& \\
~& \\
~& \\
~& \\
~& \\
~& \\
~& \\
\end{tabular}
\end{center}

\newpage
\topic{Substitution Integrals - Example 4}
\problem Find $\ds \int \frac{\sin(x)}{1 + \cos^2(x)} dx$.

\newpage

\topic{Substitutions and Definite Integrals}


\section*{Using the Method of Substitution for Definite Integrals}

If we are asked to evaluate a {\bf definite} integral such as
$$
\int_0^{\pi/2} \frac{\sin x}{1+\cos x} dx \ \ ,
$$

where a substitution will ease the integration, we have two methods
for handling the limits of integration ($x=0$ and $x=\pi/2$).
\begin{enumerate}[a)]
\item When we make our substitution, convert both the {\em variables}
  $x$ and the {\em limits} (in $x$) to the new variable; or
\item do the integration while keeping the limits explicitly in terms
  of $x$, writing the final integral back in terms of the original $x$
  variable as well, and {\em then} evaluating.
\end{enumerate}

\newpage

\problem Use method a) to evaluate the integral 
$$\int_0^{\pi/2} \frac{\sin x}{1+\cos x} dx$$

\vfill

\newpage
\problem Use method b) method to evaluate 
$$ \int_9^{64} \frac{\sqrt{1 + \sqrt{x}}}{\sqrt{x}} dx \ \ .
$$

\newpage

\topic{Integration Method - By Parts}

\section*{Integration by Parts}

So far in studying integrals we have used 
\begin{itemize}
\item direct anti-differentiation, for relatively simple functions,
  and
\item integration by substitution, for some more complex integrals.
\end{itemize}

However, there are many integrals that can't be evaluated with these
techniques.

\problem Try to find $\ds \int x e^{4x} ~dx$.

\vfill

\newpage

This particular integral can be evaluated with a different integration
technique, {\bf integration by parts.}  This rule is related to the
{\bf product rule} for derivatives.  

\problem Expand
\begin{align*}
\frac{d}{dx} \left(u v\right) = 
\end{align*}

Integrate both sides with respect to $x$ and simplify.

\vfill

Express $\ds \int u \frac{dv}{dx} ~dx$ relative to the other terms.

\vfill


\newpage
\begin{boxnote}
  {\bf Integration by Parts} \\
  For short, we can remember this formula as
\begin{align*}
\ds \int udv = uv - \int vdu
\end{align*}
\end{boxnote}

\vspace{3mm} Integration by parts: 
\begin{itemize}
\item Choose a part of the integral to be $u$, and the remaining part
  to be $dv$.
\item {\bf Differentiate} $u$ to get $du$.
\item {\bf Integrate} $dv$ to get $v$.
\item Replace $\ds \int u ~dv$ with $\ds uv - \int v du$.
\item Hope/check that the new integral is easier to evaluate.
\end{itemize}


\newpage

\problem Use integration by parts to evaluate $\ds \int x e^{4x} ~dx$.

\vfill
\vfill
\vfill

\newpage
\problem Verify that your anti-derivative is correct.

\vfill

\newpage
\topic{Integration By Parts - Examples}
\subsection*{Integration By Parts - Examples}

\vfill
\begin{boxnote}
{\bf Guidelines for selecting $u$ and $dv$}

\begin{itemize}
\item Try to select $u$ and $dv$ so that either 
\begin{itemize}
\item $u'$ is simpler than $u$ or
\item $\int dv$ is simpler than $dv$ 
\end{itemize}
\item Ensure you can actually integrate the $dv$ part by itself
\end{itemize}
\end{boxnote}

\newpage
\problem Find $\ds \int x  \cos x ~ dx$.

\vfill


\newpage
\problem Now evaluate the slightly more challenging integral $$\ds \int x^2\cos x ~ dx$$

\vfill

\newpage

\hfill $\ds \int x^2\cos x ~ dx$

\newpage 

\topic{Integration By Parts - Definite Integrals}
\subsection*{Integration By Parts - Definite Integrals}
When using integration by parts to evaluate {\em definite} integrals,
you need to apply the limits of integration to the {\bf entire}
anti-derivative that you find.

\problem Evaluate $\ds \int_{0}^{\pi} x  \sin 4x ~ dx$

\newpage 

Don't forget that $dv$ does not require any other factors besides
$dx$.  That can help when there is only a single factor in the
integrand.

\problem Find the area under the graph of $\ln x$ between $x=1$ and
$x=2$.

\newpage

General integration advice:
\begin{itemize}
\item Look for a substitution in your integral first - they are the simplest method to use,
and usually the most obvious.
\item Only try integration by parts if substitution fails. 
\item With all methods, you may need to {\bf experiment} with your
  choice of $u, dv$, or your substitution.
\end{itemize}


\end{document}

